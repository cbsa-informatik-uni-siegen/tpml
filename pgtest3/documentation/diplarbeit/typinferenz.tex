%%
%% Typinferenz
%%


\chapter{Typinferenz}
\label{kapitel:Typinferenz}



%%%%%%%%%%%%%%%%%%%%%%%%%%%%%%%%%
%% Die Programmiersprache Loti %%
%%%%%%%%%%%%%%%%%%%%%%%%%%%%%%%%%

\section{Die Programmiersprache \Loti}

Die kontextfreie Grammatik der Sprache \Loti\ erh\"alt man, indem man Syntax der Sprache
\Lot\ durch folgende Produktionen erweitert.
\GRbeg
e \GRis \expAbstr{x}{e_1}
  \GRal \expRec{x}{e_1}
  \GRal \expObject{\self}{r}
\GRend
Bei Abstraktionen, Rekursionen und Objekten darf nun wahlweise ein Typ angegeben oder
weggelassen werden.

Die Typregeln von \Loti\ sind die diejenigen von \Lot\ zusammen mit den drei folgenden
Regeln f\"ur die neuen (typlosen) Ausdr\"ucke. \\[5mm]
\begin{tabular}{ll}
  \RN{Abstr'}  & $\RULE{\Tje{\Gamma\SUB{\tau}{x}}{e}{\tau'}}
                       {\Tje{\Gamma}{\expAbstr{x}{e}}{\typeArrow{\tau}{\tau'}}}$ \\[4mm]
  \RN{Rec'}    & $\RULE{\Tje{\Gamma\SUB{\tau}{x}}{e}{\tau}}
                       {\Tje{\Gamma}{\expRec{x}{e}}{\tau}}$ \\[4mm]
  \RN{Object'} & $\RULE{\Tjr{\Gamma^\star\SUB{\tau}{\self}}{r}{\phi}
                        \quad
                        \tau = \typeObject{\phi}}
                       {\Tje{\Gamma}{\expObject{\self}{r}}{\tau}}$
\end{tabular} \\[7mm]

{\bf TODO:} \"Uberleitung

\begin{proposition}
  F\"ur einen Ausdrucke $e$ in \Loti\ gilt $\Tj{\Gamma}{e}{\tau}$ (in \Loti) genau dann,
  wenn es einen Ausdruck $e'$ in \Lot\ gibt mit $\erase{e} = \erase{e'}$ und
  $\Tj{\Gamma}{e'}{\tau}$ (in \Lot).
\end{proposition}

\begin{beweis}
  Trivial.
\end{beweis}

\begin{korollar}
  Ein Ausdruck $e$ in \Loti\ ist genau dann wohlgetypt und abgeschlossen, wenn
  es einen wohlgetypten, abgeschlossenen Ausdruck $e'$ in \Lot\ gibt mit
  $\erase{e} = \erase{e'}$.
\end{korollar}


%%%%%%%%%%%%%%%%%%%%%%%%%%%%%%%%%%%%%%%%%%%%%%%%%%%
%% Typvariablen, Substitionen und Typgleichungen %%
%%%%%%%%%%%%%%%%%%%%%%%%%%%%%%%%%%%%%%%%%%%%%%%%%%%

\section{Typvariablen, Substitutionen und Typgleichungen}

\begin{definition}[Typvariablen] \
  \begin{itemize}
    \item Vorgegeben seien eine unendliche Menge $\setTVar$, deren Elemente $\alpha$ wir als
          \Define{Typvariablen}{Typvariable} bezeichnen, und eine unendliche Menge $\setRVar$,
          deren Elemente $\varrho$ wir als \Define{Reihenvariablen}{Reihenvariable} bezeichnen.
    \item Die Menge $\setType$ der Typen $\tau$ und die Menge $\setRType$ der Reihentypen
          $\phi$ werden erweitert, indem man die kontextfreien Grammatik um die Produktionen
          \GRbeg
            \tau \GRis \alpha
                 \GRnl
            \phi \GRis \varrho
          \GRend
          erweitert.
  \end{itemize}
\end{definition}
Insbesondere k\"onnen nun in Ausdr\"ucken statt konkreter Typen auch Typvariablen oder
Reihenvariablen verwendet werden. Es besteht somit zum Beispiel die M\"oglichkeit den
Typ des Parameters einer Funktion "`ungenau"' zu spezifizieren, wie das folgende Beispiel
verdeutlicht.

\begin{beispiel} \
  \begin{enumerate}
    \item Die Identit\"atsfunktion kann nun unabh\"angig von einem konkreten Typ durch
          \EQNbeg
            \expAbstr{x:\alpha}{x}
          \EQNend
          definiert werden.
    \item Ebenso wird nun die (h\"ohere) Funktion \name{twice}, die zu einer Funktion
          $f$ eine neue Funktion $f \circ f$ liefert, durch
          \EQNbeg
            \expAbstr{f:\typeArrow{\alpha}{\alpha}}{\expAbstr{x:\alpha}{\expApp{f}{(\expApp{f}{x})}}}
          \EQNend
          typunabh\"angig definiert.
  \end{enumerate}
\end{beispiel}
Das zweite Beispiel veranschaulicht auf einfache Weise den gewonnen Grad an Flexibilit\"at im Typsystem.
Statt f\"ur die Parameter $f$ und $x$ konkrete Typen festzulegen, spezifizieren wir lediglich, dass
$f$ und $x$ "`typm\"a"sig zusammenpassen"' m\"ussen. N\"amlich, dass $f$ auf $x$ anwendbar ist,
und gleichermassen $f$ auf das Ergebnis einer Applikation von $f$ angewendet werden kann.

\begin{definition}[Substitution] \label{definition:Loti:Substitution} \
  \begin{enumerate}
    \item Eine \define{Typsubstitution} ist eine Funktion $s_\tau:\setTVar \to \setType$, f\"ur die gilt:
          Es existieren nur endliche viele Typvariablen $\alpha$ mit $s_\tau(\alpha) \ne \alpha$. 
    \item Eine \define{Reihensubstitution} ist eine Funktion $s_\phi:\setRVar \to \setRType$, f\"ur die
          gilt: Es existieren nur endliche viele Reihenvariablen $\varrho$ mit $s_\phi(\varrho)\ne\varrho$.
    \item Mit $\setSubst$ bezeichnen wir die Menge aller Typsubstitutionen $s_\tau$ und
          Reihensubstitutionen $s_\phi$, wobei wir die Element $s$ der Menge $\setSubst$ zusammenfassend
          als \Define{Substitutionen}{Substitution} bezeichnen.
  \end{enumerate}
\end{definition}
Wir benutzen die Schreibweise $[^{\tau_1}/_{\alpha_1},\ldots,^{\tau_n}/_{\alpha_n}]$ f\"ur die
Typsubstitution $s_\tau:\setTVar \to \setType$ mit 
\begin{itemize}
  \item $\forall i \in \{1,\ldots,n\}:\,s_\tau(\alpha_i) = \tau_i$ und
  \item $\forall \alpha \not\in \{\alpha_1,\ldots,\alpha_n\}:\,s_\tau(\alpha) = \alpha$,
\end{itemize}

% vi:set ts=2 sw=2 et ai syntax=tex:
