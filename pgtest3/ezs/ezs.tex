\documentclass[a4paper,latin1,12pt]{article}

\usepackage{amsmath}
\usepackage{amssymb}
\usepackage{amstext}
\usepackage{german}
\usepackage{ngerman}
\usepackage[latin1]{inputenc}
\usepackage{color}
\usepackage{latexsym}
\usepackage{array}
\usepackage{stmaryrd}
\usepackage{hyperref}
\usepackage{index}
\usepackage{graphicx}

\title{{\Huge Echtzeitsysteme}\\Fragenkatalog}
\author{{\Large Christian Fehler, Benjamin Mies, Michael Oeste}}
\date{\small\today}

\newindex{default}{idx}{ind}{Stichwortverzeichnis}

\newcommand{\question}[3]{\item {\textbf{#1?}}\ (S.\ #2)\\[2mm]#3}
\newcommand{\statement}[3]{\item {\textbf{#1!}}\ (S.\ #2)\\[2mm]#3}
\newcommand{\catchword}[1]{-\ #1\\}
\newcommand{\normaltext}[1]{#1\\}
\newcommand{\page}[1]{#1}
\newcommand{\pages}[2]{#1-#2}

\begin{document}

\maketitle
\newpage
\tableofcontents

\newpage
\section{Einf"uhrung}

\begin{enumerate}

  \question{In welchen Bereichen sind Automaten dem Menschen unterlegen}{\page{1}}
  {
    \catchword{Erkennen von Dingen}
    \catchword{Erkennen von Situationen}
    \catchword{Freiheit der Entscheidung}
  }

  \question{In welchen Bereichen sind Automaten dem Menschen "uberlegen}{\page{1}}
  {
    \catchword{Schnelligkeit}
    \catchword{Pr"azision}
  }

  \question{Wie nennt man den Vorgang zur Einf"uhrung von Automaten}{\page{1}}
  {
   \normaltext{Automatisierung}
  }

  \question{Wie nennt man den Umfang, in dem ein Vorgang oder Prozess automatisiert ist}{\page{1}}
  {
    \normaltext{Automatisierungsgrad}
  }

  \question{Wann ist dieser umso h"oher}{\page{1}}
  {
    \normaltext{Je einfacher ein Vorgang oder Prozess ist}
  }

  \question{Welche zwei Einheiten findet man bei automatisierten Prozessen}{\page{1}}
  {
    \catchword{Technischer Prozess}
    \catchword{Automatische Steuerung}
  }

  \question{Wie lautet die DIN Definition eines Automaten}{\page{2}}
  {
    \normaltext{Ein Automat ist ein k"unstliches System, das selbstst"andig ein Programm
                befolgt. Auf Grund des Programms trifft das System Entscheidungen, die
                auf der Verkn"upfung von Eingaben mit dem jeweiligen Zustand des Systems
                beruhen, und Ausgaben zur Folge haben.}
  }

  \statement{Nennen Sie die zwei wesentlichen Merkmale eines Automaten}{\page{2}}
  {
    \catchword{Grundstruktur aus einer kausalen Folge von Eingabe, Verarbeitung und Ausgabe}
    \catchword{Die Verarbeitung beinhaltet Entscheidungen zwischen verschiedenen M"oglichkeiten}
  }

  \question{Welche Ziele hat die Automatisierung}{\page{2}}
  {
    \catchword{Vereinfachung des Prozesses}
    \catchword{Menschen f"ur Arbeiten mit h"oherem Anspruch freisetzen}
  }

  \statement{Nennen Sie die f"unf typischen Bereiche,
             in denen der Mensch Entlastung erfahren kann}{\pages{2}{4}}
  {
    \catchword{Arbeiten, die regelm"a"sig und monoton wiederkehren}
    \catchword{Arbeiten, die hohe Anforderungen an die Konzentration stellen}
    \catchword{Arbeiten, die eine hohe Arbeitsgeschwindigkeit verlangen}
    \catchword{Arbeiten, die eine gro"se Datenmenge liefern}
    \catchword{Arbeiten, die in gesundheitsgef"ahrdender Umgebung stattfinden}
  }

  \newpage
  \section{Prozessrechner Hardware}

  \question{Warum ist das Fach doof}{\page{x}}
  {
    \normaltext{Weil es so ist!}
  }

  \question{Warum ist das Fach doof}{\page{x}}
  {
    \normaltext{Weil es so ist!}
  }

  \newpage
  \section{Periphere Ger"ate}

  \question{Warum ist das Fach doof}{\page{x}}
  {
    \normaltext{Weil es so ist!}
  }

  \question{Warum ist das Fach doof}{\page{x}}
  {
    \normaltext{Weil es so ist!}
  }

  \newpage
  \section{Prozessrechner-Software}

  \question{Warum ist das Fach doof}{\page{x}}
  {
    \normaltext{Weil es so ist!}
  }

  \question{Warum ist das Fach doof}{\page{x}}
  {
    \normaltext{Weil es so ist!}
  }

  \newpage
  \section{Sonstiges}

  \question{Warum ist das Fach doof}{\page{x}}
  {
    \normaltext{Weil es so ist!}
  }

  \question{Warum ist das Fach doof}{\page{x}}
  {
    \normaltext{Weil es so ist!}
  }

\end{enumerate}


\end{document}