\documentclass[a4paper,12pt]{article}

\usepackage{amsmath}
\usepackage{amssymb}
\usepackage{amstext}
\usepackage{german}
\usepackage{ngerman}
\usepackage[utf8]{inputenc}
\usepackage{color}
\usepackage{latexsym}
\usepackage{array}
\usepackage{stmaryrd}
\usepackage{hyperref}
\usepackage{index}
\usepackage{graphicx}

\title{{\Huge Echtzeitsysteme}\\Fragenkatalog}
\author{{\Large Christian Fehler, Benjamin Mies, Michael Oeste}}
\date{\small\today}

\newindex{default}{idx}{ind}{Stichwortverzeichnis}

\newcommand{\question}[3]{\pagebreak[3]\item {\textbf{#1?}}\ (S.\ #2)#3}
\newcommand{\statement}[3]{\pagebreak[3]\item {\textbf{#1!}}\ (S.\ #2)#3}
\newcommand{\catchword}[1]{\\-\ #1}
\newcommand{\normaltext}[1]{\\#1}
\newcommand{\result}[1]{\\ $\Rightarrow$\ #1}
\newcommand{\resultol}[1]{ $\Rightarrow$\ #1}
\newcommand{\page}[1]{#1}
\newcommand{\pages}[2]{#1\ -\ #2}
\newcommand{\important}[1]{\textbf{#1}}

\begin{document}

\maketitle
\newpage
\tableofcontents


\newpage
\section{Einführung}


\begin{enumerate}

  \question{In welchen Bereichen sind Automaten dem Menschen unterlegen}{\page{1}}
  {
    \catchword{Erkennen von Dingen}
    \catchword{Erkennen von Situationen}
    \catchword{Freiheit der Entscheidung}
  }

  \question{In welchen Bereichen sind Automaten dem Menschen überlegen}{\page{1}}
  {
    \catchword{Schnelligkeit}
    \catchword{Präzision}
  }

  \question{Wie nennt man den Vorgang zur Einführung von Automaten}{\page{1}}
  {
   \normaltext{Automatisierung}
  }

  \question{Wie nennt man den Umfang, in dem ein Vorgang oder Prozess automatisiert ist}{\page{1}}
  {
    \normaltext{Automatisierungsgrad}
  }

  \question{Wann ist dieser umso höher}{\page{1}}
  {
    \normaltext{Je einfacher ein Vorgang oder Prozess ist}
  }

  \question{Welche zwei Einheiten findet man bei automatisierten Prozessen}{\page{1}}
  {
    \catchword{Technischer Prozess}
    \catchword{Automatische Steuerung}
  }

  \question{Wie lautet die DIN Definition eines Automaten}{\page{2}}
  {
    \normaltext{Ein Automat ist ein künstliches System, das selbstständig ein Programm
                befolgt. Auf Grund des Programms trifft das System Entscheidungen, die
                auf der Verknüpfung von Eingaben mit dem jeweiligen Zustand des Systems
                beruhen, und Ausgaben zur Folge haben.}
  }

  \statement{Nennen Sie die zwei wesentlichen Merkmale eines Automaten}{\page{2}}
  {
    \catchword{Grundstruktur aus einer kausalen Folge von Eingabe, Verarbeitung und Ausgabe}
    \catchword{Die Verarbeitung beinhaltet Entscheidungen zwischen verschiedenen Möglichkeiten}
  }

  \question{Welche Ziele hat die Automatisierung}{\page{2}}
  {
    \catchword{Vereinfachung des Prozesses}
    \catchword{Menschen für Arbeiten mit höherem Anspruch freisetzen}
    \catchword{Höhere Produktivität}
  }

  \statement{Nennen Sie die fünf typischen Bereiche,
             in denen der Mensch Entlastung erfahren kann}{\pages{2}{4}}
  {
    \catchword{Arbeiten, die regelmäßig und monoton wiederkehren}
    \catchword{Arbeiten, die hohe Anforderungen an die Konzentration stellen. Mit wechsel von Zeiten
               hoher Aktivität und hoher Anforderung an die Aufmerksamkeit und Zeiten geringer Aktivität,
               die mit Langeweile ausgefüllt sind.}
    \catchword{Arbeiten, die eine hohe Arbeitsgeschwindigkeit verlangen}
    \catchword{Arbeiten, die eine große Datenmenge liefern}
    \catchword{Arbeiten, die in gesundheitsgefährdender Umgebung stattfinden}
  }

  \question{Welche Bereiche umfasst eine höhere Produktivität}{\page{4}}
  {
    \catchword{Größere Menge (Quantität)}
    \catchword{Bessere Qualität}
  }

  \statement{Nennen Sie die Grenzen der Automatisierung}{\page{4}}
  {
    \catchword{Ein automatisches System kann nicht auf alle vorkommenden Ereignisse reagieren}
    \catchword{Kleine Fehler können katastrophale Folgen haben}
  }

  \question{In welchen Bereichen ist die Erkenntnisfähigkeit des Menschen notwendig}{\page{4}}
  {
    \catchword{Spracherkennung}
    \catchword{Bildverarbeitung}
  }

  \question{Was ist die Besonderheit bei der PDV}{\page{5}}
  {
    \normaltext{Die Lenkung eines technischen Prozesses}
  }

  \question{Welche besonderen Anforderungen werden an den Rechner gestellt}{\page{5}}
  {
    \catchword{Einhaltung der zeitlichen Abfolgen}
    \catchword{Einhaltung der vorgegebenen Bearbeitungszeiten}
    \result{Realzeitbedingungen einhalten}
  }

  \question{Wie ist der Begriff Prozess definiert}{\page{6}}
  {
    \normaltext{Der Vorgang zur Umformung, zum Transport oder zur Speicherung von Materie, 
    Enerie oder Information. Ein technischer Prozess ist ein Prozess, dessen Zustandsgrößen
    mit technischen Mitteln erfasst und beeinflusst werden können.}
  }

  \question{Welche Anforderungen stellt das Einhalten der zeitlichen Abfolgen}{\page{5}}
  {
    \catchword{Schritthaltende Verarbeitungsgeschwidigkeit}
    \catchword{Viele Entscheidungsmöglichkeiten in einem Programm}
    \catchword{Sicherheit}
  }

  \question{Welche zwei Einheiten kommunizieren in der PDV}{\page{5}}
  {
    \catchword{Rechner}
    \catchword{Prozess}
  }

  \question{Wie lautet die DIN Definition eines Prozesses}{\page{6}}
  {
    \normaltext{Der Vorgang zur Umformatierung, zum Transport oder zur Speicherung von Materie,
                Energie oder Information. Ein technischer Prozess ist ein Prozess, dessen
                Zustandsgrößen mit technischen Mitteln erfasst und beeinflusst werden können.}
  }

  \question{Was bezeichnet man als Verarbeitungsart}{\page{6}}
  {
    \catchword{Umformung}
    \catchword{Transport}
    \catchword{Speicherung}
  }

  \question{Aus welchen zwei Teilen besteht ein Elementarprozess}{\page{6}}
  {
    \catchword{Verarbeitungsart}
    \catchword{Verarbeitungsgut}
  }

  \question{Welche Kopplungen zwischen den Verarbeitungsarten gibt es}{\page{7}}
  {
    \catchword{Ein Speicherprozess ist immer mit einem Transportprozess gekoppelt}
    \catchword{Ein Umformungsprozess ist immer mit einem Transport- oder Speicherprozess gekoppelt}
  }

  \question{In welche drei Teile lässt sich die Verarbeitungsart unterteilen}{\page{6}}
  {
    \catchword{Transport}
    \catchword{Umformung}
    \catchword{Speicherung}
  }

  \question{In welche drei Teile lässt sich das Verarbeitungsgut unterteilen}{\page{6}}
  {
    \catchword{Materie}
    \catchword{Energie}
    \catchword{Information}
  }

  \question{Mit welchem Prozess ist ein Speicherprozess fast immer gekoppelt}{\page{6}}
  {
    \normaltext{Transportprozess}
  }

  \question{Mit welchem Prozess ist ein Umformunsprozess fast immer gekoppelt}{\page{6}}
  {
    \normaltext{Transport- oder Speicherprozess}
  }

  \question{In welche Arten kann man Prozesse nach ihrer Komlexität einteilen}{\pages{7}{8}}
  {
    \catchword{Elementarprozesse (sind durch genau eine Verarbeitungsart und ein 
    Verarbeitungsgut gekennzeichnet)}
    \catchword{Einzelprozesse (stellen die kleinste geschlossene Prozesseinheit dar.
       Sind aus mehreren Elementarprozessen zusammengesetzt)}
    \catchword{Maschinen (basieren auf mehreren Elementar- oder Einzelprozessen)}
    \catchword{Anlagen (bestehen aus mehreren Maschinen)}
    \catchword{Verbundprozesse (bestehen aus mehreren Einzelprozessen, die zusammenarbeiten)}
    \catchword{Betriebsprozesse (umfassen verschieden Verarbeitungsbereiche eines Betriebs)}
  }

  \question{Wie kann man die Verarbeitungsstruktur unterteilen}{\page{8}}
  {
    \catchword{kontinuierliche Verarbeitung (zB Transportsysteme für Flüssigkeiten und Gase)}
    \catchword{diskontinuierliche oder diskrete Verarbeitung bei Stückproszessen (zB Automobilbau)}
    \catchword{Chargenprozesse (Verarbeitung auf Teilmengen eines sonst kontinuierlichen Gutes)}
  }

  \question{Wie ist ein Rechner laut DIN definiert}{\page{10}}
  {
    \normaltext{Ein Rechensystem, das aus einer Zentraleinheit (mit Prozessor, Zentralspeicher und 
     Ein-/Ausgabewerken) und der notwendigen Peripherie (Ein-/Ausgabegeräte, Massenspeicher und
     Prozessperipherie) besteht}
  }

  \question{Wie unterscheiden sich Prozessdaten und Prozessgrößen}{\page{6}}
  {
    \catchword{Prozessdaten werden zwischen Prozess und Prozessrechner ausgetauscht, und sind 
    dem Rechner zugeordnet}
    \catchword{Prozessgrößen beschreiben einen technischen Prozess unabhängig von seiner Lenkung
    durch einen Rechner}
  }

  \question{Wie kann die Ablaufstruktur sein}{\page{9}}
  {
    \catchword{Deterministisch}
    \catchword{Stochastisch}
  }

  \question{Wie definiert man deterministische Prozesse}{\page{9}}
  {
    \catchword{Zeitliche Abfolge liegt fest}
    \catchword{Sind vorhersagbar}
  }

  \question{Wie definiert man stochastische Prozesse}{\page{9}}
  {
    \catchword{Keine vorhersagbare Abfolge}
    \catchword{Gehorcht den Gestzen der Statistik}
  }

  \statement{Nennen und erklären Sie kurz die Verarbeitungsbereiche}{\page{9}}
  {
    \catchword{Fertigung: Verarbeitung von Gütern}
    \catchword{Verfahrenstechnik: Kontinuierliche oder chargenweise Verarbeitung
               von Gütern oder Energie}
    \catchword{Verteilungsprozesse: Transport und Speicherung von Materie, Energie und Information}
    \catchword{Verwaltung: Verarbeitung von Informationen}
  }

  \question{Wie lautet die DIN Definition eines Prozessrechners}{\page{10}}
  {
    \normaltext{Als Prozessrechner wird ein Rechner bezeichnet, der mittels Prozessperipherie direkt
                an einen Prozess gekoppelt ist.}
  }

  \statement{Geben Sie die Prozessgrößen an}{\page{11}}
  {
    \catchword{Prozess-Kennwerte}
    \catchword{Prozessparameter}
    \catchword{Prozesszustand}
    \catchword{Eingangsgrößen}
    \catchword{Ausgangsgrößen}
  }

  \question{Was sind Prozessgrößen}{\page{11}}
  {
    \normaltext{Technische, physikalische oder chemische Größen, die mit technischen Mitteln
    erfass und beeinflußt werden können, und durch Zahlenwerte (Daten) beschrieben werden können}
  }

  \question{Welche Verarbeitungsarten gibt es, und welche Größen sind charakteristisch}{\page{12}}
  {
    \catchword{Speicherprozess \resultol{die Menge}}
    \catchword{Transport \resultol{der Durchsatz}}
    \catchword{Umformung \resultol{Art und Umfang der Änderung}}
  }
  \question{Was sind Prozesskennwert}{\page{12}}
  {
    \normaltext{Prozesskennwerte oder Anlagedaten sind technische Daten, die durch den Aufbau eines
    Prozesses, seine Struktur, Konstruktion und Anwendung festgelegt sind. Also Festwerte, zu denen
    vor allem auch Grenzwerte gehören.}
  }

  \question{Was sind Prozesszustandswerte}{\page{12}}
  {
    \normaltext{Prozesszustandwerte sind die im Betrieb auftrenden veränderlichen Größen, die den 
    aktuellen Zustand des Prozesses beschreiben, und mit technischen Mitteln erfasst und beeinflusst
    werden können.}
  }

  \question{Was sind Prozessparameter}{\page{12}}
  {
    \normaltext{Prozessparameter stellen eine Zwischenstufe dar, also solche Größen, die in verschiedenen
    Prozessablaufen verschiedene Werte annehmen können.}
  }

  \question{Wonach kann man Betriebsgrößen unterscheiden}{\page{12}}
  {
    \catchword{Flussrichtung}
    \catchword{Eingangsgrößen}
    \catchword{Ausgangsgrößen}
  }

  \question{Welche Arten der Prozesskopplung gibt es}{\pages{14}{16}}
  {
    \catchword{Indirekt gekoppelter Betrieb (engl.: off-line) wobei die Daten zwischen Rechner und Prozess
    mittels Datenträger übermittelt werden.}
    \catchword{Direkt gekoppelter Betrieb (engl.: on-line) mit direktem Datenaustausch
               zwischen Rechner und Prozess}
    \catchword{Offener Prozessbetrieb (engl.: open loop control) wobei der Fluss der Daten
               einseitig ist und noch teilweise die Mitarbeit des Menschen erfordert.}
    \catchword{Geschlossener Prozessbetrieb (engl.: closed loop control) ist ein vollständiger
               Regelkreis, bei dem Istwerte vom Prozess aufgenommen, miet den Sollwerten im
               Rechner vergleichen und daraus wieder Steuerdaten für den Prozess gewonnen werden.}
  }

  \question{Wie kann man den Prozessbetrieb weiter unterteilen}{\page{15}}
  {
    \catchword{Messwerterfassung. Hier übernimmt der Rechner nur Daten vom Prozess,
               um sie weiter zu verarbeiten.}
    \catchword{Prozesssteuerung. Hier werden ausschließlich Daten vom Rechner zum Prozess übertragen. }
  }

  \question{Warum ergeben sich bein geschlossenen Prozessbetrieb geschlossenen Prozessbetrieb
            höhere Anforderungen als im offenen}{\page{16}}
  {
    \normaltext{Da hier 2 gegenläufige Datenströme verarbeitet werden müssen}
    \catchword{Eingabedaten: Sollwerte vom Menschen, und Istwerte vom Prozess}
    \catchword{Verarbeitung: Bestimmung des Prozesszustands, Vergleich von Ist- und Sollwerten,
               Berechnung der notwendigen Aktionen zum Abgleich im Prozess und Umwandlung in Prozessdaten}
    \catchword{Ausgabedaten: Steuerdaten an den Prozess und Zustandsdaten an den Menschen}
  }

  \statement{Beschreiben Sie die Eingangsgrößen eines Prozesses}{\page{12}}
  {
    \catchword{Fließen in den Prozess rein}
    \catchword{Wirken auf den Prozess}
    \catchword{Unabhängig vom Prozesszustand}
    \catchword{Kann kontinuierlich, stückweise oder in Chargen erfolgen}
  }

  \question{Welche Arten von Eigangsgrößen gibt es}{\page{12}}
  {
    \catchword{Stellgrößen: Wirken mehr oder minder direkt auf die Prozesszustandsgrößen}
    \catchword{Störgrößen: Treten meist als unerwünschte und oft schwer kontrollierbare Störungen auf}
  }

  \statement{Beschreiben Sie die Ausgangsgrößen eines Prozesses}{\page{13}}
  {
    \catchword{Fließen aus dem Prozess raus}
    \catchword{Können noch auf den Prozess zurückwirken}
    \catchword{Stets abhängig vom aktuellen Zustand des Prozesses}
    \catchword{Abhängig von den Stellgrößen}
  }

  \question{Welche Anforderungen werden an die Meßgrößen gestellt}{\page{13}}
  {
    \catchword{Aktuelle Zustandsgrößen sollen unabhängig von eineinander sein}
    \catchword{Sollen keine Rückwirkungen auf den Prozess haben}
  }

  \question{Aus welchen Komponenten bestehen Prozessdaten}{\page{14}}
  {
    \normaltext{Eingabe- und Ausgabedaten des Prozessrechners}
  }

  \question{In welcher Form können Eingabedaten vorkommen}{\page{14}}
  {
    \catchword{Digital}
    \catchword{Analog}
    \catchword{Alarm durch nur ein Bit}
  }

  \question{Welche Komponenten müssen in einem Prozessmodell vorhanden sein}{\page{17}}
  {
    \catchword{Eingangsdaten}
    \catchword{Ausgangsdaten}
    \catchword{Zustandsdaten}
    \result{Zusammenhänge zwischen diesen Größen}
  }

  \question{Welche zwei Arten von Modellen gibt es und wie können diese Überprüft werden}{\page{17}}
  {
    \catchword{Gegenständliche Modell}
    \catchword{Mathematisches Modell}
    \result{Simulation mit einem Computer}
  }

  \question{Was muss bei der Prozessbeschreibung gemacht werden}{\page{17}}
  {
    \catchword{Strukturierung des Prozesses in Teilprozesse}
    \catchword{Verfeinerung bis zu Elementarprozessen}
    \catchword{Beschreibung ihrer Zusammenhänge}
  }

  \question{Wie werden statische Strukturen dargestellt}{\page{17}}
  {
    \normaltext{Meistens durch ein, an den jeweiligen Einsatzbereich angepasstes Blockschaltbild}
  }

  \question{Wie werden stationäre Abläufe dargestellt}{\page{17}}
  {
    \normaltext{In Blockdiagramme als Wirkungs- oder Transportwege}
  }

  \question{Welchen Unterschied gibt es bei den Ablaufstrukturen gegenüber
            den bekannten Programmablaufplänen}{\page{17}}
  {
    \normaltext{Parallelverarbeitung}
  }

  \statement{Welche Arten von Bearbeitung gibt es}{\page{18}}
  {
    \catchword{Sequentielle Bearbeitung: Immer nur eine Station aktiv, das Bearbeitungsstück wird
               von einer zur nächsten Station weitergegeben}
    \catchword{Serielle Bearbeitung: Alle Stationen aktiv, die Bearbeitungsstücke werden von einer
               zur nächsten Station weitergegeben}
    \catchword{Parallele Bearbeitung: Entweder identische Vorgänge zur Steigerung des Durchsatzes,
               oder Vorgänge die zu einem Endprodukt gebraucht werden}
  }

  \statement{Erklären Sie die Verteilte Verarbeitung}{\page{20}}
  {
    \catchword{Unterteilung von komplexen Prozessen in Teilbereiche}
    \catchword{Für Teilbereiche einen eigenen hochspezialisierten Prozessrechner einsetzen,
               der einen höheren Wirkungsgrad haben sollte}
  }

  \question{Welche Arten der Prozesserkennung gibt es}{\page{19}}
  {
    \catchword{Empirische Prozesserkennung \resultol der technische Prozess ist bereits realisiert, und soll 
    nachträglich automatisiert werden}
    \catchword{Analytische Modelierung \resultol der Ablauf wird durch bekannte physikalische oder chemische
    Zusammenhänge (Gleichungen) dargestellt. }
  }

  \question{In welchen Ebenen kann die Erstellung eines Modells erfolgen}{\page{19}}
  {
    \catchword{als statisches Modell, in dem die Prozessgrößen, Eingangs-, Ausgangsgrößen und Zustandsgrößen
               definiert und beschrieben werden.}
    \catchword{als stationäres Modell, in dem die Zusammenhänge zwischen den Prozessgrößen für den stationären 
               Zustand beschrieben werden, also für den Fall, dass alle Größen im wesentlichen stabil sind.}
    \catchword{als dynamische Modell, in dem die zeitlichen Veränderungen der Prozessgrößen durch
               innere oder äußere Wirkungen beschrieben werden, insbesondere auch Einschalt- 
               und Einschwingvorgänge.}
  }

  \question{Wodurch entsteht die Unvollkommenheit eines Modells, und wie kann diese 
            verringert werden}{\page{20}}
  {
    \catchword{Sie entsteht dadurch, dass die Störgrößen nicht genügend genau bekannt sind.}
    \catchword{Dies kann durch geeignette Analysen und Simulationen verringert werden}
  }

  \question{Was muss geschehen, um eine gemeinsame Prozesslenkung zu erhalten und
            was ergibt sich daraus}{\page{20}}
  {
    \normaltext{Koppelung der Einzelrechner, die durch einen Leitrechner geführt werden}
    \result{Eine hierachische Baumstruktur entsteht}
  }

  \question{Was bezeichnet man als lernenden Prozessmodell}{\page{20}}
  {
    \catchword{Modelle, die nicht nur passiv die Parameter auf Grund von Prozessdaten bestimmen,
               sondern die optimalen Werte aktiv selber sucht.}
  }

  \question{Welche Komponente sitzt noch über dem Leitrechner}{\page{21}}
  {
    \normaltext{Betriebsrechner}
  }

  \question{Welche Vorteile ergeben sich aus der Verteilung der Aufgaben}{\page{21}}
  {
    \catchword{Leistungssteigerung}
    \catchword{Höhere Verfügbarkeit \resultol{Leitrechner kann einspringen}}
  }

  \question{Welche Lösungsansätze gibt es um ein Beispiel zu lösen}{\page{24}}
  {
    \catchword{Stationäre Lösung: Dauerhafte Erfassung der Zustandgrößen}
    \catchword{Dynamische Lösung: Indirekte Erfassung über Änderungen und einen Anfangszustand}
  }

  \question{Welches Problem ergibt sich bei dynamischen Lösungen, bezogen auf die Zeit}{\page{26}}
  {
    \catchword{Die Erfassung der Eingabedaten erfolgt nicht gleichzeitig}
    \result{Programmzeiten müssen klein im Vergleich zu den prozesszeiten sein}
  }

  \question{Welche Arten von Kommunikation gibt es zwischen Rechner und Peripherie}{\pages{26}{27}}
  {
    \catchword{Abfragebetrieb (Polling)}
    \catchword{Interrupt-Betrieb}
  }

  \question{Welche Probleme ergeben sich bei eintreffenden Ereignissen}{\page{29}}
  {
    \catchword{Kurz hintereinander: Ereignisse können verloren gehen}
    \catchword{Selten: Rechner ist mit unproduktiven Warten beschäftigt}
  }

  \question{Was ist die Grundforderung an die PDV}{\page{30}}
  {
    \normaltext{Der Rechner muss dem Prozessgeschehen, insbesondere seinem wirklichen
                Zeitverhalten angepasst sein. Dies wird als Realzeitbetrieb bezeichnet.}
  }


  \newpage
  \section{Prozessrechner Hardware}


  \question{Welche Werke vereint die CPU}{\page{39}}
  {
    \catchword{Steuerwerk}
    \catchword{Rechenwerk}
    \catchword{evtl. Unterbrechungswerk}
    \catchword{evtl. Rechenwerk für Gleitkommazahlen}
  }

  \question{Welche Arten von Leitungen gibt es in der CPU}{\page{39}}
  {
    \catchword{Adressleitung \resultol{nur zum Senden von Signalen (unidirektional)}}
    \catchword{Datenleitung \resultol{zum Senden und Empfangen (bidirektional)}}
    \catchword{Steuerleitung \resultol{sowohl uni- als auch bidirektional ausgelegt}}
  }

  \statement{Erklären Sie den Aufbau der CPU (schematisch)} {page{40}}
  {
  \normaltext{siehe seite 40 Bild 2.3}
  }

  \question{Wozu dient das Unterbrechungswerk}{\page{39}}
  {
    \catchword{Zur Bearbeitung von Interrupts die hier über ein weiters Bit im Prozessor-
    Status-Wort zugelassen oder verhindert werden können}
  }

  \question{Welche wesentlichen 3 Aufgaben hat das Mikroprogramm}{\page{48}}
  {
    \catchword{Einlesen von Instruktionen und Daten aus dem Hauptspeicher, und das Abspeichern 
               von Ergebnissen}
    \catchword{Interpretieren von Instruktionen, wobei es in eigene Programmteile (Module) verzweigt}
    \catchword{Steuern der CPU d.h. des Datenverkehrs innerhalb der CPU, und ins besondere 
               des Rechenwerks (ALU)}
  }

  \question{Wie bezeichnet man Unterprogramme des Mikroprogramms}{\page{51}}
  {
    \catchword{Nanoprogramme}
  }

  \question{Wofür steht die Abkürzung RISC}{\page{52}}
  {
    \normaltext{Reduced Instruction Set Computing}
  }

  \question{Wofür steht die Abkürzung CISC}{\page{52}}
  {
    \normaltext{Complex Instruction Set Computing}
  }

  \question{Worin unterscheiden sich CISC und RISC}{\pages{52}{53}}
  {
    \catchword{Load-and-Store-Architektur}
    \catchword{Pipelining (Einteilung in 5 Phasen: IF, RD, OP, MM, WT)}
  }

  \question{In welche Klassen ist der Instruktionssatz bei RISC Prozessoren aufgeteilt}{\page{52}}
  {
    \catchword{Instruktionen zum Speicherzugriff (LOAD, STORE)}
    \catchword{Verarbeitungsinstruktionen (ADD, SUB)}
    \catchword{Verzweigungsinstruktionen (JMP, BNE)}
    \catchword{Steuerungsinstruktionen (HALT, WAIT)}
  }

  \question{Was kann zu besonderen Problemen beim Pipelining führen}{\page{54}}
  {
    \catchword{Verzweigung innerhalb des Programms (synchron)}
    \catchword{Unterbrechung von außen (asynchron)}
    \result{Dann müssen solche Instruktionen, die bereits begonnen wurden, abgebrochen 
            werden; bei manchen geht das nicht, und es muss die Verarbeitungskette angehalten 
            werden (pipe stalling).}
  }

  \statement{Nennen Sie Eigenschaften von RISC in der PDV}{\page{54}}
  {
    \catchword{Da die Verarbeitunsoperationen auf Daten im kArbeitsspeicher immer zusätzlich
               eine LOAD- und eine STORE-Operation erfordern, werden die Maschinenprogramme
               deutlich länger.}
    \catchword{Erheblicher zusätzlicher Verwaltungsaufwand bei bedingten Sprüngen}
  }

  \question{Wie heißt das Kernstück des Rechenwerkes und wozu dient es}{\page{41}}
  {
    \normaltext{ALU: Durchführung von Operationen auf Operanden}
  }

  \question{Welche zwei Arten von Operationen gibt es bezogen auf die Operanden}{\page{42}}
  {
    \catchword{Binäre Operationen: Zwei Quelloperanden z.B. OR}
    \catchword{Unäre Operationen: Ein Quelloperand z.B. NOT}
  }

  \question{Welche drei Klassen von Operationen gibt es}{\page{42}}
  {
    \catchword{Boolsche}
    \catchword{Arithmetische}
    \catchword{Logische}
  }

  \statement{Nennen Sie die vier Flaggen, die erzeugt werden}{\page{43}}
  {
    \catchword{Zero Bit: Zeigt an, dass das Ergebnis null war}
    \catchword{Negativ Bit: Zeigt an, dass das Ergebnis negativ war}
    \catchword{Carry Bit: Übertrag, wenn mit positiven Zahlen gerechnet wird}
    \catchword{Overflow Bit: Zahlenüberlauf, wenn mit Zahlen im 2-er Komplement gerechnet wird}
  }

  \question{Was versteht man unter Maschineninstruktionen}{\page{44}}
  {
    \normaltext{Informationen, die das Steuerwerk braucht, um Instruktionen auf 
                Maschinenebene durchzuführen}
  }

  \question{Aus welchen Teilen besteht jede Instruktion}{\page{44}}
  {
    \catchword{Operationsteil}
    \catchword{Adressteil}
  }

  \question{Welche drei Gruppen von Instruktionen können unterschieden werden}{\page{44}}
  {
    \catchword{Steruerungsinstruktionen}
    \catchword{Verarbeitungsoperationen}
    \catchword{Verzweigungen}
  }

  \question{Was dient bei bedingten Sprüngen zur Überprüfung der Sprungbedingungen}{\page{45}}
  {
    \normaltext{Die entsprechenden Status Flags der ALU}
  }

  \question{Was wird bei der Speicheradressierung stets verwendet}{\page{46}}
  {
    \normaltext{Ein internes Register: Program Counter (PC), Indexregister (IX)
                oder der Stack Pointer (SP)}
  }

  \statement{Erklären Sie die Adressierungsmodi des Program Counters}{\page{47}}
  {
    \catchword{Direkt: Der Operand erfolgt unmittelbar auf die Instruktion, der PC zeigt drauf}
    \catchword{Absolut: Die absolute Adresse des Operanden folgt unmittelbar auf die Instruktion}
    \catchword{Relativ: Die relative Adresse des Operanden folgt unmittelbar auf die Instruktion,
               die absolute Adresse muss daraus durch Addition des PC erst berechnet werden}
    \catchword{Indirekt: Die absolute Adresse der absoluten Adresse des Operanden folgt unmittelbar
               auf die Instruktion}
  }

  \question{Was wird beim Unterprogrammsprung gemacht}{\page{47}}
  {
    \catchword{Aktueller PC auf dem Stack ablegen}
    \catchword{Stack Pointer erhöhen}
    \catchword{Unterprogramm ausführen}
    \catchword{Rücksprungadresse vom Stack holen}
    \catchword{Stack Pointer erniedrigen}
  }

  \question{Welche drei Gruppen von Leitungen benutzt ein Bus}{\page{55}}
  {
    \catchword{Datenleitungen}
    \catchword{Adressleitungen}
    \catchword{Steuerleitungen}
  }

  \question{Worin liegt die besondere Eignung des Busprinzips}{\page{55}}
  {
    \normaltext{Große Flexibilität beim Einrichten oder bei Änderungen}
  }

  \question{Was ist der größte Nachteil des Busses}{\page{55}}
  {
    \normaltext{Alle Daten werden über ihn transportiert, dabei kann ein Engpass entstehen}
  }

    \question{Welche Arten von Bussen gibt es}{\page{56}}
  {
    \catchword{Synchrone Busse (festes Zeitraster wird vorgegeben, richtet sich nach der 
               langsamsten Komponente)}
    \catchword{Asynchrone Busse (Quittierungsverfahren - Handshaking, nutzt die individuellen
               Geschwindigkeiten der Komponenten aus)}
  }

  \statement{Erklären Sie Multiplexverfahren}{\page{59}}
  {
    \normaltext{Für Daten und Steuerleitungen werden dieselben Leitungen im zeitlichen Wechesel
                benutzt. Dazu wird eine weitere Steuerleitung zur Siganlisierung des Wechsels benötigt.}
  }

  \question{Wozu wird die Schnittstelle zum Bus verwendet}{\page{83}}
  {
    \catchword{Gesamte Datenübertragung zum und vom Gerät}
    \catchword{Gesamte Steurung des Gerätes}
    \catchword{Steuerung des Geräteanschlusses selbst}
  }

  \question{Welches sind die wichtigsten Register eines Geräteanschlusses}{\page{85}}
  {
    \catchword{Data Buffer Register (DBR) ist ein Datenpuffer zur Zwischenspeicherung von Daten}
    \catchword{Control and Status Register (CSR) enthält einzelne Bits für die Steuerung des Gerätes}
  }

  \question{Welche Geräteschnittstellen kann man unterscheiden}{\page{88}}
  {
    \catchword{Parallelschnittstellen, auf denen die einzelnen Bits paralles, die Zeichen selber 
               aber immer nacheinander (seriell) übertragen werden. Da für jedes Bit eine Leitung benötigt 
               wird, sind die Anschlüsse und Kabel aufwendig, aber man erreicht eine hohe
               Übertragungsgeschwindigkeit.}
    \catchword{Seriellschnittstellen, auf welchen auch die einzelnen Bits nacheinander übertragen werden. 
               Dadurch können die Daten unabhängig von der Datenbreite über eine einzig Leitung
               übertragen werden. Allerdangs entsteht hier ein höherer Aufwand bei der Umsetzung, und 
               die Übertragungsgeschwindigkeit ist niedriger.}
  }

  \question{Woraus besteht ein Bus}{\page{55}}
  {
    \normaltext{Aus Leitungsbündeln, mit denen die Steckplätze für die Komponenten verbunden sind}
  }

  \question{Was sollte mit Teilen der Zentraleinheit geschehem}{\page{55}}
  {
    \catchword{Möglichst nahe an den technischen Prozess bringen}
    \catchword{Möglichst weit weg von Störeinflüssen}
    \result{Stehen manchmal im Widerspruch zueinander}
  }

  \question{Warum dürfen bestimmte Längen nicht überschritten werden}{\page{56}}
  {
    \normaltext{Wegen der Signallaufzeit (etwa 5 ns/m)\resultol{Kollisionen auf dem Bus}}
  }

  \question{Welche Störungen können in Kabeln auftreten}{\page{56}}
  {
    \catchword{Signalverfälschungen durch äußere Störungen}
    \catchword{Interne Reflexionen}
  }

  \question{Wie können diese Störungen verhindert werden}{\page{56}}
  {
    \catchword{Abgeschirmte Leitungen}
    \catchword{Verdrillte Leitungen mit definiertem Wellenwiderstand}
  }

  \statement{Nennen Sie die technischen Merkmale eines Busses}{\page{56}}
  {
    \catchword{Zahl der Leitungen}
    \catchword{Verwendung jeder Leitung als Daten-, Adress-, oder Steuerleitung}
    \catchword{Belegung der Steckplätze}
    \catchword{Art der Signale}
  }

  \statement{Nennen Sie die Aufgabe der Geräteanschlüsse}{\page{82}}
  {
    \normaltext{Ankopplung von externen Geräten verschiedenster Art}
  }

  \question{Wie erfolgt die Kommunikation mit der Peripherie}{\page{82}}
  {
    \catchword{Zustand des Gerätes muss abgefragt werden}
    \catchword{Oder von dem Gerät gemeldet werden}
  }

  \question{Was ist die eigentliche Aufgabe eines Geräteanschlusses}{\page{82}}
  {
    \catchword{Anpassung der Daten, der Adressen und Steuersignale}
    \catchword{Steuerung des Datenflusses}
  }

  \statement{Nennen Sie die fünf verschiedenen Geräteanschlüsse}{\page{83}}
  {
    \catchword{Ein-/ Ausgabe-Prozessoren}
    \catchword{Blockmultiplexer}
    \catchword{Controller}
    \catchword{Byte-Multiplexer}
    \catchword{Interfaces}
  }

  \statement{Nennen Sie die zwei Übertragungsverfahren}{\page{86}}
  {
    \catchword{Einzelzeichenübertragung: Bei zeichenorientierten Geräten}
    \catchword{Blockübertragung: Vorwiegend bei Massenspeicher}
  }

  \question{Welche zwei Arten von Datenübertragung zwischen Controller und Speicher gibt es}{\page{87}}
  {
    \catchword{Blockpufferung: Ankommende Daten werden im Puffer zwischengespeichert und dort abgeholt}
    \catchword{Speicherdirektzugriff (DMA): Meist ohne Puffer und ohne Eingriff der CPU eine
               direkte Kommunikation mit dem Arbeitsspeicher}
  }

  \statement{Nennen Sie Eigenschaften der synchronen serieller Übertragung}{\page{89}}
  {
    \catchword{Es wird eine lückenlose Folge von Zeichen übertragen.}
    \catchword{Wenn keine aktuellen Daten anliegen werden Füllzeichen übertragen.}
    \catchword{Hohe Datenrate, da Start und Stoppbits entfallen.}
    \catchword{Nachteilig ist das erkennen von Füllzeichen und die Synchronisierung der Taktfrequenz.}
  }

  \statement{Nennen Sie Eigenschaften der asynchronen serieller Übertragung}{\page{89}}
  {
    \catchword{Bits eines Zeichens werden kontinuierlich übertragen, jedoch können zwischen den Zeichen
    beliebig lange Ruhezeiten liegen.}
    \catchword{Anfang und Ende eines Zeichen werden durch Start- und Stoppbit gekennzeichnet.}
    \catchword{Kein erkennen von Füllzeichen und synchronisieren der Taktfrequenz nötig.}
    \catchword{Dafür müssen aber Start- und Stoppbits erkannt werden.}
  }

  \question{Welche Möglichkeiten gibt es Geräte an einen Conroller anzuschließen}{\page{90}}
  {
    \catchword{Busanschluß, auf dem wie in der Zentraleinheit Daten-, Adreß- und Steuerleitungen
    für alle angeschlossenen Geräte geführt werden.}
    \catchword{Sternanschluß, bei dem jedes Gerät über einen eigenen Anschluß am Controller verfügt.}
    \catchword{Daisy Chain (Verkettung), bei der zunächst ein Gerät an den Controller angeschlossen wird.
    An dieses Hauptgerät werden weitere identische Geräte in Form einer Kette angeschlossen.}
  }

  \question{Nennen Sie die Eigenschaften der Parallelschnittstelle}{\page{91}}
  {
    \catchword{Die Bits eines Datenworts werden parallel übertragen}
    \catchword{Nur minimale Anforderungen an die eigentlichen Schnittstellenumsetzer}
    \catchword{Übertragunsrate recht hoch und an keinen Takt gebunden}
    \catchword{Nur für geringe Entfernungen geeignet}
  }

  \question{Welche Pegel verwenden die Signale der Parallelschnittstelle}{\page{92}}
  {
    \normaltext{Alle Signalpegel entsprechen TTL}
  }

  \question{Welche Einheiten verbindet eine serielle Schnittstelle}{\pages{94}{95}}
  {
    \normaltext{Datenendeinrichtung und Datenübertragungseinrichtung}
  }

  \statement{Nennen Sie wichtigsten Arten von Leitungen}{\page{95}}
  {
    \catchword{Ground}
    \catchword{Datenleitungen}
    \catchword{Steuerleitungen}
    \catchword{Taktleitungen}
    \catchword{Sende- und Empfangsleitungen}
  }

  \question{Welche Codierung wird verwendet}{\page{96}}
  {
    \normaltext{NRZ (Non Return to Zero)}
  }

  \question{Wie groß ist die Baudrate in Bezug auf die Bitrate}{\page{98}}
  {
    \normaltext{Wegen der binären Codierung gleich groß}
  }

  \statement{Beschreiben Sie den Aufbau eines A-t-Diagramms}{\pages{106}{107}}
  {
    \catchword{Auf der x-Achse die Zeit t}
    \catchword{Auf der y-Achse die Adresse A der zum Zeitpunkt t verarbeiteten Instruktion,
               also den Inhalt des Programmzählers (PC)}
  }

  \question{Wie sieht eine sequentielle Abarbeitung in einem A-t-Diagramm aus}{\page{106}}
  {
    \normaltext{Wie eine Gerade, deren Steigung im Wesentlichen von der Rechengeschwindigkeit
                der CPU abhängt}
  }

  \question{Wie sieht ein Programmsprung in einem A-t-Diagramm aus}{\page{106}}
  {
    \normaltext{Programmsprünge erscheinen als senkrechte Linie im Diagramm}
  }

  \question{Wie kann nach Ablauffehlern gesucht werden}{\page{107}}
  {
    \normaltext{Durch Hardware-Monitore, die den Ablauf am Bus verfolgen, entweder durch
                Oszilloskope oder durch Logikanalysatoren}
  }

  \question{Warum hat der Schnittstellenumsetzer UART eine besondere Bedeutung}{\page{99}}
  {
    \catchword{Die serielle Schnitstelle ist praktisch an jedem Rechner vorhanden.}
  }

    \question{Welche Aufgabe hat der Empfänger (Seriell-Parallel-Wandler)}{\page{99}}
  {
    \catchword{Er setzt die im Datenregister DBR-T angelieferten Daten aus der bitparallelen 
    Darstellung in eine serielle um.}
  }

  \question{Welche Aufgabe hat der Sender (Parallel-Seriell-Wandler)}{\page{99}}
  {
    \catchword{Er setzt die ankommenden seriellen Daten in die parallele Darstellung um.}
  }

  \statement{Nennen Sie die unterschiedlichen Betriebsarten}{\page{108}}
  {
    \catchword{Einprogrammbetrieb: Der Rechner bearbeitet immer nur ein Programm}
    \catchword{Einbenutzerbetrieb: Nur eine Benutzer gleichzeitig aktiv
               \resultol{keine Zugriffssicherung notwendig}}
    \catchword{Mehrprogrammbetrieb: Mehrere Programme stehen ablaufbereit im Speicher}
  }

  \question{Worauf sollte bei der Programmentwicklung geachtet werden}{\page{108}}
  {
    \normaltext{Dass der Prozessbetrieb nicht behindert oder gestört wird}
  }

  \statement{Nennen Sie die zwei Verfahren, mit denen der Mehrprogrammbetrieb
             realisiert werden kann}{\page{110}}
  {
    \catchword{Time-sharing: CPU wird in Zeitscheiben zugewiesen}
    \catchword{Resource-sharing: Betriebsmittel werden nach Prioritäten oder Wartezeiten vergeben}
  }

  \question{Welche Teile besitzen sowohl Sender als auch Empfänger }{\pages{99}{100}}
  {
    \catchword{Busschnittstelle}
    \catchword{Geräteschnittstelle}
    \catchword{Pegelumsetzer}
  }

  \question{Was versteht man unter Polling}{\page{110}}
  {
    \normaltext{Bein Abrufbetrie (polling) werden Informationen über den Zustand der Geräte
                oder der Datenübertragung von einem Programm abgerufen. Dazu fragt die CPU in einer Schleife
                die CSRs der Geräteanschlüsse ab. Anhand dieser Informationen verzweigt das Programm in 
                entsprechende Routinen.}
  }

  \question{Nennen Sie Vorteile des Polling Betriebs}{\page{111}}
  {
    \catchword{Einfachheit der programmtechnischen Realisierung.}
    \catchword{Die Hardware kann relativ einfach ausgelgt werden.}
  }

  \question{Nennen Sie Nachteile des Polling Betriebs}{\page{112}}
  {
    \catchword{Die CPU ist die meist Zeit damit beschäftigt, alle CSRs abzufragen.}
    \catchword{Die Zeitauflösung (Genauigkeit) mit der man den Zeitpunkt für das Eintreten
               eines Ereignisses festlegen kann, wird zunächst durch die Zeit für einenn Schleifendurchlauf
               vorgegeben.}
    \catchword{Die Reihenfolge von verschiedenen Ereignissen kann verfälscht werden.}
    \catchword{Bei einer notwendigen Änderung z.B. Abschalten oder Hinzufügen eines neuen
               Gerätes wird eine Programmänderung notwendig, die aber nicht im laufenden Betrieb 
               durchgeführt werden kann.}
  }

  \question{Was ist die grundsätzliche Anforderung an einen Prozessrechner bei
            Realzeitverarbeitung}{\page{113}}
  {
    \normaltext{Die Verarbeitung der Prozessdaten muss schritthaltend erfolgen, d.h. so schnell,
    dass der Prozess nicht auf  die Ergebnisse warten muss. Man könnte auch sagen, die Anforderung
    durch den Prozess bestheht darin, dass ein Ereignis abgearbeitet sein muss, bevor ein neues eintritt.}
  }

  \question{Was ist die Forderung für einen sicheren Betrieb}{\page{115}}
  {
    \normaltext{Die Summe aus der maximal möglichen Reaktionszeit tr und der längsten Bearbeitungszeit tb
    soll kleiner sein, als der Minimalabstand von Prozessereignissen tp abzüglich der maximalen Stellzeiten ts.
    Bei Gleichheit liegt der denkbar ungünstige Fall vor, der eintreten kann (engl.: worst case).}
  }

  \question{Was kann bei Realzeitanwendungen zum verstopfen des Rechners fürhen}{\page{115}}
  {
    \normaltext{Häufige Ereignisse mit langer Bearbeitungszeit. Zur Vermeidung sollen solche Ereignisse gar 
    nicht zugelassen werden.}
  }


  \question{Was wird im Interrupt-Betrieb durch Ereignisse erzeugt}{\page{117}}
  {
    \normaltext{Es wird eine Unterbrechungsanforderung erzeugt, welcher stattgegeben wird, indem das gerade
    laufende Programm unterbrochen und ein Sprung in ein Hilfsprogramm durchgeführt wird, welches als Interrupt
    Service Routine (ISR) bezeichnet wird.}
  }

  \question{Welche Aufgaben übernimmt die ISR}{\page{117}}
  {
    \catchword{Übernahme von Daten}
    \catchword{Berechnungen}
    \catchword{Ausgabe von Steuerdaten}
    \catchword{nach Beendigung wieder in das unterbrochene Programm zurück springen}
  }

  \statement{Nennen Sie unterschiede zwischen Unterprogrammsprüngen und Unterbrechungen}{\pages{117}{118}}
  {
    \catchword{Bei einer Unterbrechung wird der Zeitpunkt des Sprungs durch ein Ereignis bestimmt, und 
    nicht durch das laufende Programm.}
    \catchword{ISRs stellen eigenständige Programme dar, die unabhängig voneinander erstellt und geladen
    werden können.}
    \catchword{Die Verarbeitungsziele der ISRs müssen nichts miteinander zu tun haben, und eine Kommunikation
    is zunächst nicht erforderlich.}
  }

  \question{Was sind Semaphore}{\page{118}}
  {
    \normaltext{Signalträger, vergleichbar mit den Flaggen im Prozessor-Status-Wort, mit denen ISRs 
                synchronisiert werden können, falls ihre Arbeiten voneinander abhängen.}
  }

  \statement{Nennen Sie Vorteile beim Interrupt-Betrieb}{\page{118}}
  {
    \catchword{Es können andere Programme im Mehrprogrammbetrieb bearbeitet werden, wenn keine Aufgaben für die 
    Prozesslenkung vorliegen.}
    \catchword{ISR können als eigenständige Programme erstellt, in den Arbeitsspeicher geladen, und bei Bedarf 
    entladen werden.}
    \catchword{Die Erfassung von zeitlichen Abläufen wird genauer, sowohl was den Zeitpunkt eines Ereignisses
    betrifft, als auch die zeitliche Abfolge.}
  }

  \statement{Nennen Sie Nachteile beim Interrupt-Betrieb}{\page{118}}
  {
    \catchword{Höherer Aufwand bei der Hardware und bei der Software.}
    \catchword{Die Koordinierung aller Programme ist aufwendiger.}
    \catchword{Für die Planung und Erstellung der Software müssen neue Methoden eingesetzt werden.}
    \catchword{Die Hardware muss geeignete Schaltungen und Verbindungen enthalten.}
  }

  \question{Welche Anforderungen werden an die Hardware für den Interrupt-Betrieb gestellt}{\page{120}}
  {
    \catchword{Geräteanschluss müssen in der Lage sein, aus einem vom Gerät ankommenden Ereignis eine 
    Unterbrechungsanforderung zu erzeugen.}
    \catchword{Der Bus muss spezielle Steuerleitungen für die Unterbrechungsanforderungen und für deren
    Bestätigung besitzen.}
    \catchword{Es muss ein geeignetes Protokoll definiert sein, wie die Leitungen zu verwenden sind.}
  }

  \question{Welche 3 wesentlichen Aufgaben hat das Unterbrechungswerk}{\page{122}}
  {
    \catchword{Die Abweisung oder Zulassung von Interrupt-Anforderungen und die Auswahl, wenn mehrere
    gleichzeitig auftreten.}
    \catchword{Die Durchführung des Interrupt-Zyklus auf dem Bus und das Lesen des Interrupt-Vektors.}
    \catchword{Die Aktivierung des Steuerwerks, welches die Interrupt-Sequenz durchführt.}
  }

  \question{Was versteht man unter Non-maskable Interrupts (NMI)}{\page{122}}
  {
    \normaltext{Unterbrechungen, die nicht abgeblockt werden dürfen. Diese werden dann auf einer 
    besonderen Leitung angefordert.}
  }

  \question{Was ist die Aufgabe der Interrupt-Logik im Geräteanschluss}{\page{122}}
  {
    \catchword{Im Falle einer Abweisung dafür sorgen, dass keine Unterbrechungsanforderung verloren geht.}
    \catchword{Die Unterbrechungsanforderungen nach ihrer Erledigung löschen.}
  }

   \question{Welche Methoden zur Auswahl von Interrupt-Anforderungen gibt es}{\page{128}}
  {
    \catchword{Maskierung}
    \catchword{Prioritäten}
  }

  \question{Wie funktioniert Maskierung}{\page{128}}
  {
    \catchword{Die Interrupt-Leitungen des Busses werden uncodiert jeweils einem Bit in einem 
    Interrupt-Register zugeführt, das eine Maskenfunktion hat.}
    \catchword{Wenn das entsprechende Maskenbit gesetzt ist, wird der Interrupt zugelassen.}
    \catchword{Da hier noch Konkurrenzen auftreten können, muss eine zusätzliche Prioritätenregelung 
    eingeführt werden.}
  }

   \question{Wie funktioniert Prioritätenregelung nach Rangordnung}{\pages{128}{129}}
  {
    \catchword{Die ankommenden Interrupt-Leitungen werden als Dualzahlen codiert und dadurch auf wenig 
    Bit komprimiert.}
    \catchword{Diese wird als mehrfaches logisches ODER überlagert, oder in ihrer Gesamtheit gleich 
    als Dualzahl interpretiert.}
    \catchword{Diese Zahl wird als Hardware-Priorität mit der aktuellen Software-Priorität numerisch 
    verglichen.}
    \catchword{Wenn die Hardware Priorität größer als die Software-Priorität ist, wird der Interrupt
    zugelassen. (Das hat zur Folge, dass alle Interrupts abgewiesen werden könne, wenn im PSW ein genügend
    großer Wert eingetragen wird)}
  }

  \statement{Erklären Sie detailiert den Ablauf eines Interrupts}{\pages{119}{120}}
  {
    \catchword{Eine Unterbrechungsanforderung wird von einem Geräteanschluss erzeugt
               und das aktuelle Programm wird unterbrochen}
    \catchword{Der aktuelle Stand des Programmzählers (PC) und der Inhalt der
               Prozessor-Status-Worts (PS) werden auf dem Stack abgelegt. Die
               Adresse dafür enthält der Stack-Pointer (SP)}
    \catchword{Die Unterbrechungsanforderung muss identifiziert werden, dazu dient
               ein Vektor. Mit diesem Vektor wird in der Interrupt-Vektor-Tabelle (IVT)
               die Startadresse (der neue PC) für die zugeordnete ISR gesucht}
    \catchword{Diese Zuordnung kann über eine eigenständige Routine erfolgen,
               die Interrupt-Dispatcher-Routine, welche die Geräte zur Identifizierung
               abfragt}
    \catchword{Diese Methode ist allerdings sehr langsam, deshalb wird in der Regel
               die Zuordnung automatisch durch vektorisierte Interrupts, die bei
               der Interrupt-Anforderungen den Vektor mitliefern}
    \catchword{Mit dem neuen PC aus der IVT erfolgt der Sprung in die ISR. Dabei muss
               ein neuer Prozesszustand (PS) definiert werden}
    \catchword{Nach der Abarbeitung der ISR erfolgt der Rücksprung über eine spezielle
               Rücksprunginstruktion (RTI, Return from Interrupt). Der alte PC und
               PS werden vom Stack geholt und zurückgeladen}
    \catchword{Das unterbrochene Programm wird fortgesetzt}
  }

  \statement{Skizzieren Sie eine Interrupt-Anforderung}{\page{119}}
  {
    \normaltext{Siehe Bild 2.51 auf Seite 119}
  }

  \question{Was muss gemacht werden, wenn bei der Interrupt-Bearbeitung
            Konflikte auftreten}{\page{122}}
  {
    \catchword{Vorrangregeln}
    \catchword{Vergabe von Prioritäten}
  }

  \question{Welche zwei Probleme können bei eintreffenden Ereignissen auftreten}{\page{122}}
  {
    \catchword{Von einer Ereignisquelle treffen kurzzeitig mehr Ereignisse ein,
               als verarbeitet werden können. }
    \catchword{Von verschiedenen Ereignisquellen treffen praktisch gleichzeitig Ereignisse ein.
               Dann muss eine Reihenfolge der Abarbeitung gewählt werden}
  }

  \question{Wann können Ereignisfolgen nur aufgefangen werden}{\page{122}}
  {
    \normaltext{Nur, wenn die zulässige Antwortzeit genügend groß ist}
  }

  \statement{Nennen Sie zwei Methoden Ereignisfolgen aufzunehmen}{\page{123}}
  {
    \catchword{Ereignisse werden zwischengepuffert}
    \catchword{Die ISR kann von neuen Ereignissen immer wieder unterbrochen werden}
  }

  \question{Welche Probleme ergeben sich, wenn die ISR unterbrochen wird}{\page{123}}
  {
    \normaltext{Sie muss besondere Eigenschaften haben, so dass sie jederzeit wieder
                benutzt werden kann. Sie muss unterbrechbar sein, bevor sie ihre
                Ergebnisse ausgegeben hat. Diese Eigenschaft wird als \important{reentrancy}
                bezeichnet, wörtlich als Wiedereintrittsfähigkeit}
  }

  \question{Was müssen Programme, die mit der Eigenschaft reentrancy
            bezeichnet werden, machen}{\page{124}}
  {
    \normaltext{Alle anfallenden Zwischenergebnisse müssen in jedem Durchlauf gut
                gesichert werden. Das kann dadurch erreicht werden, gleich am Beginn
                der Routine alle benötigten Speicherplätze von Variablen und alle
                CPU-Register auf dem Stack gesichert werden}
  }

  \question{Was muss bei vielen eintreffenden Ereignissen geschehen und zu welchen
            Problemen führt dies}{\page{124}}
  {
    \catchword{Ereignisse müssen in einer Warteschlange gesammelt werden, und können
               dann in der Reihenfolge bearbeitet werden, in der sie eingetroffen
               sind}
    \result{Die Reaktionszeit kann beliebig lang werden}
  }

  \question{Welches Vorgehen liefert bessere Ergebnisse}{\page{124}}
  {
    \normaltext{Besser Ergebnisse erhält man, wenn man auch die ISRs unterbricht,
                allerdings nicht wahllos, sondern es müssen Prioritäten gesetzt werden}
  }

  \question{Wie lautet eine einfache Regel zur Vergabe von Prioritäten}{\page{125}}
  {
    \catchword{Ereignisse mit hoher Folgefrequenz und kurzer Bearbeitungszeit, bekommen
               höhere Prioritäten}
    \catchword{Ereignisse, die selten auftreten, aber eine lange Bearbeitungszeit haben,
               bekommen niedrigere Prioritäten}
    \result{Diese Regel muss aber häufig durchbrochen werden}
  }

  \question{Was muss bei einer Vorrangregel geschehen}{\page{125}}
  {
    \normaltext{Unterbrechungsanforderungen müssen zugelassen oder abgewiesen werden}
  }

  \question{Was für den Impuls von eintreffenden Ereignissen gelten}{\page{125}}
  {
    \normaltext{Der Länge des Impuls sollte möglichst kurz sein gegenüber allen
                Prozess- und Verarbeitungszeiten}
  }

  \question{Worüber kann festgelegt werden, ob ein Interrupt zugelassen
            oder abgewiesen wird}{\page{125}}
  {
    \normaltext{Durch das Interrupt-Enable-Bit (IE) im Interface}
  }

  \statement{Skizzieren Sie die Interrupt Verkettung}{\page{126}}
  {
    \normaltext{Siehe Bild 2.56 auf Seite 126}
  }

  \statement{Erklären Sie das Prinzip der Daisy Chain bei Interrupts}{\page{126}}
  {
    \normaltext{Bei der \important{Daisy Chain} werden die Anschlüsse hardwaremäßig
                verkettet, so dass sich eine Rangfolge der Interrupt-Anforderungen
                ergibt. Hierbei hat der Anschluss, welcher der CPU am nächsten liegt
                die höchste Priorität.}
    \result{Die Rangfolge ist dabei statisch}
  }

  \question{Welche Komponente fällt bei dynamischen Prioritäten die Entscheidung,
            welcher Interrupt der wichtigste ist}{\page{127}}
  {
    \normaltext{Das Unterbrechungswerk}
  }


  \statement{Nennen Sie die Grundregeln für ISR bei Realzeitverhalten}{\pages{131}{132}}
  {
    \catchword{Die ISR muss die Prozessdaten so schnell wie möglich abholen (lesen) und abspeichern.}
    \catchword{Während dieser Zeit soll sie durch eine genügend hohe Prirität nicht unterbrechbar sein, 
    auf jeden Fall nicht durch ein gleichartiges Ereignis, das diese ISR benutzt.}
    \catchword{Die Ausgabewerte müssen so schnell wie möglich berechnet und ausgegeben werden.}
  }


  \question{Aus welchen zwei Werten wird die Interrupt-Anforderung IRQ gebildet}{\page{132}}
  {
    \normaltext{Aus der logischen UND-Verknüpfung von R- und IE-Bit}
  }

  \question{Wie kann verhindert werden, dass eine Interrupt-Behandlung unterbrochen wird}{\page{134}}
  {
    \normaltext{Indem am Anfang der Interrupt-Behandlung das IE-Bit auf 0 gesetzt wird.
                Das R-Bit wird dann zwar gesetzt, aber der Interrupt wird nicht zugelassen.
                Unmittelbar vor dem Ende der ISR wird der Interrupt wieder freigegeben.}
  }

  \statement{Nennen Sie vorbeugende Maßnahmen gegen Netzausfall}{\page{136}}
  {
    \catchword{Einsatz von Unterbrechungsfreier Stromversorgung (USV).}
    \catchword{Einsatz von Notstromaggregaten.}
    \catchword{Übergang aller Systeme in einen sicheren Zustand, bei dem die Folgeschäden minimal
    bleiben (fail-safe Technik).}
  }
  
  \statement{Nennen Sie die Aktionen beim Netzausfall}{\page{136}}
  {
    \catchword{Der Prozess muss in einen sicheren Zustand gebracht werden, wobei auf eine korrekte
    Reihenfolge geachtet werden muss.}
    \catchword{Der aktuelle Zustand des Prozesses sollte erfasst und gesichert werden.}
    \catchword{Der Zustand des Programms sollter gesichert werden, d.h. an welcher Stelle er mit
    welchen Werten unterbrochen wurde.}
  }
  
  \question{Was ist die besondere Eigenschaft bei Systemaufrufen von Interrupts und Traps}{\page{142}}
  {
    \normaltext{Sie springen Hilfsprogramme (Handler) an, ohne dass deren Startadress bekannt ist.
    Sie haben stattdessen einen Zeiger auf eine Tabelle der die Startadresse entnommen werden kann.}
  }

  \question{Was versteht man unter Direct Memory Access (DMA)}{\page{145}}
  {
    \normaltext{Um die CPU zu entlasten werden Daten direkt zwischen Controller und Arbeitsspeicher
    ausgetauscht.}
  }

  
  \question{Was für Anforderungen werden an die Hardware bei DMA gestellt}{\page{145}}
  {
  \statement{Nennen und erklären Sie kurz die auftretenden Fehler}{\pages{140}{142}}
  }
  {
    \catchword{Diese Methode erfordert besondere Intelligenz beim Controller.}
    \catchword{Die CPU muss darauf vorbereitet sein.}
    \catchword{Auf dem Bus muss ein geeignetes Protokoll definiert sein.}
    \catchword{Der Arbeitsspeicher muss nur in einigen Ausnahmefällen besonders ausgestattet sein.}
  }


  \statement{Nennen Sie die drei charakteristischen Abschnitte des DMA-Betriebs}{\page{145}}
  {
    \catchword{Der Controller benötigt eine Eingabe von der CPU}
    \catchword{Der Controller führt eine Verarbeitung durch}
    \catchword{Der Controller liefert eine Ausgabe}
  }





  \newpage
  \section{Periphere Geräte}


  \statement{Beschreibe kurz die generelle Aufgabe der Prozessperepherie}{\page{188}}
  {
    \catchword{Prozessgrößen in Rechnerdaten umwandeln und umgekehrt}
  }

  \question{In welcher Form können die Daten des Prozess vorliegen}{\page{188}}
  {
    \catchword{elektrischer und nicht elektrischer Form (Strom, Spannung, Temperatur)}
    \catchword{digital oder analog, sie ändern sich diskret oder kontinuierlich (Stückzahlen, MEssströme)}
    \catchword{dynamische Binärwerte (Unterbrechung einer Lichtschranke)}
  }

  \statement{Beschreiben Sie anhand eines Schaubildes den Weg der Informationen
             vom Prozess in den Rechner und umgekehrt}{\page{189}}
  {
    \normaltext{wichtiges Bild!!!}
  }

  \statement{Beschreiben Sie grundlegend die Aufgaben einer Messkette}{\page{190}}
  {
    \normaltext{An der Messstelle wird die Messgröße erfasst, die eventuell dann 
              transformiert werden muss. Das Messgerät setzt den Messwert in ein Signal
              um, das leicht weiterverarbeitet werden kann. Das Messgerät besteht
              aus zwei Komponenten:}
    \catchword{Messfühler: er setzt die Messgröße durch Ausnutzen naturwissenschafftlicher
             Gesetzmäßigkeiten in leichter erfassbare Größen um. Z.B.: Druck/Temparatur in einen Weg} 
    \catchword{Messwandler: einen Weiter Umwandlung mit dem Ergebnis eines meist elektrischen
               Signals, das gut übertragen werden kann.}
    \normaltext{Zuletzt wird das analoge elektische Signal mittels A/D-Wandler in ein digitales
                Singal umgewandelt und an den Rechner übertragen.}
  }

  \statement{Nennen Sie einen Grund für einen frühe A/D-Wandlung (vor der Übertragung zum Rechner)}{\page{191}} 
  {
    \catchword{digitale Daten können durch sichere Codes vor Übertragungsfehlern geschützt werden,
               das Ergebnis wird nicht weiter verfälscht.} 
  }

  \statement{Nennen Sie Faktoren, die in das Modell des gesamten Messprozesses eingehen müssen,
             insbesondere Fehlermöglichkeiten)}{\page{191}} 
  {
    \catchword{Übertragungseingenschafften aller Glieder in der Kette} 
    \catchword{Übersetzungsverhältnisse} 
    \catchword{Übertragungsfunktionen und -kennlinien} 
    \catchword{Signallaufzeiten} 
    \catchword{Kodierungen}
    \normaltext{Jede Umwandlung, und jeder Transport vor der A/D-Wandlung verfälscht das Ergebnis.}
  }

  \newpage
  \section{Prozessrechner-Software}


  \newpage
  \section{Sonstiges}


\end{enumerate}


\end{document}
