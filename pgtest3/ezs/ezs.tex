\documentclass[a4paper,12pt]{article}

\usepackage{amsmath}
\usepackage{amssymb}
\usepackage{amstext}
\usepackage{german}
\usepackage{ngerman}
\usepackage[utf8]{inputenc}
\usepackage{color}
\usepackage{latexsym}
\usepackage{array}
\usepackage{stmaryrd}
\usepackage{hyperref}
\usepackage{index}
\usepackage{graphicx}
\usepackage{a4wide}

\title{{\Huge Echtzeitsysteme}\\Fragenkatalog}
\author{{\Large Christian Fehler, Benjamin Mies, Michael Oeste}}
\date{\small\today}

\newindex{default}{idx}{ind}{Stichwortverzeichnis}

% Mit Antworten
\newcommand{\question}[3]{\pagebreak[3]\item {\textbf{#1?}}\ (S.\ #2)#3}
\newcommand{\statement}[3]{\pagebreak[3]\item {\textbf{#1!}}\ (S.\ #2)#3}
\newcommand{\questionnopage}[2]{\pagebreak[3]\item {\textbf{#1?}}#2}
\newcommand{\statementnopage}[2]{\pagebreak[3]\item {\textbf{#1!}}#2}

% Ohne Antworten
%\newcommand{\question}[3]{\pagebreak[3]\item {\textbf{#1?}}\ (S.\ #2)}
%\newcommand{\statement}[3]{\pagebreak[3]\item {\textbf{#1!}}\ (S.\ #2)}
%\newcommand{\questionnopage}[2]{\pagebreak[3]\item {\textbf{#1?}}}
%\newcommand{\statementnopage}[2]{\pagebreak[3]\item {\textbf{#1!}}}

\newcommand{\catchword}[1]{\\-\ #1}
\newcommand{\normaltext}[1]{\\#1}
\newcommand{\result}[1]{\\$\Rightarrow$\ #1}

\newcommand{\resultol}[1]{$\Rightarrow$\ #1}

\newcommand{\page}[1]{#1}
\newcommand{\pages}[2]{#1\ -\ #2}

\newcommand{\important}[1]{\textbf{#1}}

\begin{document}

\maketitle
\newpage
\tableofcontents

\newpage
\section{Einführung}

\begin{enumerate}

  \subsection{Automatisierung durch Prozessdatenverarbeitung}

  \subsubsection{Automatisierung von Prozessen}

  \question{In welchen Bereichen sind Automaten dem Menschen unterlegen}{\page{1}}
  {
    \catchword{Erkennen von Dingen}
    \catchword{Erkennen von Situationen}
    \catchword{Freiheit der Entscheidung}
  }

  \question{In welchen Bereichen sind Automaten dem Menschen überlegen}{\page{1}}
  {
    \catchword{Schnelligkeit}
    \catchword{Präzision}
  }

  \question{Wie nennt man den Vorgang zur Einführung von Automaten}{\page{1}}
  {
   \normaltext{Automatisierung}
  }

  \question{Wie nennt man den Umfang, in dem ein Vorgang oder Prozess automatisiert ist}{\page{1}}
  {
    \normaltext{Automatisierungsgrad}
  }

  \question{Wann ist dieser umso höher}{\page{1}}
  {
    \normaltext{Je einfacher ein Vorgang oder Prozess ist}
  }

  \question{Welche zwei Einheiten findet man bei automatisierten Prozessen}{\page{1}}
  {
    \catchword{Technischer Prozess}
    \catchword{Automatische Steuerung}
  }

  \question{Wie lautet die DIN Definition eines Automaten}{\page{2}}
  {
    \normaltext{Ein Automat ist ein künstliches System, das selbstständig ein Programm
                befolgt. Auf Grund des Programms trifft das System Entscheidungen, die
                auf der Verknüpfung von Eingaben mit dem jeweiligen Zustand des Systems
                beruhen, und Ausgaben zur Folge haben.}
  }

  \statement{Nennen Sie die zwei wesentlichen Merkmale eines Automaten}{\page{2}}
  {
    \catchword{Grundstruktur aus einer kausalen Folge von Eingabe, Verarbeitung und Ausgabe}
    \catchword{Die Verarbeitung beinhaltet Entscheidungen zwischen verschiedenen Möglichkeiten}
  }

  \subsubsection{Ziele der Automatisierung}

  \question{Welche Ziele hat die Automatisierung}{\page{2}}
  {
    \catchword{Vereinfachung des Prozesses}
    \catchword{Menschen für Arbeiten mit höherem Anspruch freisetzen}
    \catchword{Höhere Produktivität}
  }

  \statement{Nennen Sie die fünf typischen Bereiche,
             in denen der Mensch Entlastung erfahren kann}{\pages{2}{4}}
  {
    \catchword{Arbeiten, die regelmäßig und monoton wiederkehren}
    \catchword{Arbeiten, die hohe Anforderungen an die Konzentration stellen. Mit Wechsel von Zeiten
               hoher Aktivität und hoher Anforderung an die Aufmerksamkeit und Zeiten geringer Aktivität,
               die mit Langeweile ausgefüllt sind.}
    \catchword{Arbeiten, die eine hohe Arbeitsgeschwindigkeit verlangen}
    \catchword{Arbeiten, die eine große Datenmenge liefern}
    \catchword{Arbeiten, die in gesundheitsgefährdender Umgebung stattfinden}
  }

  \question{Welche Bereiche umfasst eine höhere Produktivität}{\page{4}}
  {
    \catchword{Größere Menge (Quantität)}
    \catchword{Bessere Qualität}
  }

  \statement{Nennen Sie die Grenzen der Automatisierung}{\page{4}}
  {
    \catchword{Ein automatisches System kann nicht auf alle vorkommenden Ereignisse reagieren}
    \catchword{Kleine Fehler können katastrophale Folgen haben}
  }

  \question{In welchen Bereichen ist die Erkenntnisfähigkeit des Menschen notwendig}{\page{4}}
  {
    \catchword{Spracherkennung}
    \catchword{Bildverarbeitung}
  }

  \subsubsection{Prozessdatenverarbeitung}

  \question{Was ist die Besonderheit bei der PDV}{\page{5}}
  {
    \normaltext{Die Lenkung eines technischen Prozesses}
  }

  \question{Welche besonderen Anforderungen werden an den Rechner gestellt}{\page{5}}
  {
    \catchword{Einhaltung der zeitlichen Abfolgen}
    \catchword{Einhaltung der vorgegebenen Bearbeitungszeiten}
    \result{Realzeitbedingungen einhalten}
  }

  \question{Welche Anforderungen stellt das Einhalten der zeitlichen Abfolgen}{\page{5}}
  {
    \catchword{Schritthaltende Verarbeitungsgeschwidigkeit}
    \catchword{Viele Entscheidungsmöglichkeiten in einem Programm}
    \catchword{Sicherheit}
  }

  \question{Welche zwei Einheiten kommunizieren in der PDV}{\page{5}}
  {
    \catchword{Rechner}
    \catchword{Prozess}
  }

  \subsection{Begriffe aus der Prozessdatenverarbeitung}

  \subsubsection{Prozess}

  \question{Wie ist der Begriff Prozess definiert}{\page{6}}
  {
    \normaltext{Der Vorgang zur Umformung, zum Transport oder zur Speicherung von Materie, 
                Energie oder Information. Ein technischer Prozess ist ein Prozess, dessen
                Zustandsgrößen mit technischen Mitteln erfasst und beeinflusst werden können.}
  }

  \question{Was bezeichnet man als Verarbeitungsart}{\page{6}}
  {
    \catchword{Umformung}
    \catchword{Transport}
    \catchword{Speicherung}
  }

  \question{Aus welchen zwei Teilen besteht ein Elementarprozess}{\page{6}}
  {
    \catchword{Verarbeitungsart}
    \catchword{Verarbeitungsgut}
  }

  \question{In welche drei Teile lässt sich die Verarbeitungsart unterteilen}{\page{6}}
  {
    \catchword{Transport}
    \catchword{Umformung}
    \catchword{Speicherung}
  }

  \question{In welche drei Teile lässt sich das Verarbeitungsgut unterteilen}{\page{6}}
  {
    \catchword{Materie}
    \catchword{Energie}
    \catchword{Information}
  }

  \question{Mit welchem Prozess ist ein Speicherprozess fast immer gekoppelt}{\page{6}}
  {
    \normaltext{Transportprozess}
  }

  \question{Mit welchem Prozess ist ein Umformunsprozess fast immer gekoppelt}{\page{6}}
  {
    \normaltext{Transport- oder Speicherprozess}
  }

  \question{In welche Arten kann man Prozesse nach ihrer Komplexität einteilen}{\pages{7}{8}}
  {
    \catchword{Elementarprozesse: Sind durch genau eine Verarbeitungsart und ein
               Verarbeitungsgut gekennzeichnet}
    \catchword{Einzelprozesse: Stellen die kleinste geschlossene Prozesseinheit dar
               Sind aus mehreren Elementarprozessen zusammengesetzt}
    \catchword{Maschinen: Basieren auf mehreren Elementar- oder Einzelprozessen}
    \catchword{Anlagen: Bestehen aus mehreren Maschinen}
    \catchword{Verbundprozesse: Bestehen aus mehreren Einzelprozessen, die zusammenarbeiten}
    \catchword{Betriebsprozesse: Umfassen verschiedene Verarbeitungsbereiche eines Betriebs}
  }

  \question{Wie kann man die Verarbeitungsstruktur unterteilen}{\page{8}}
  {
    \catchword{Kontinuierliche Verarbeitung (z.B. Transportsysteme für Flüssigkeiten und Gase)}
    \catchword{Diskontinuierliche oder diskrete Verarbeitung bei Stückprozessen (z.B. Automobilbau)}
    \catchword{Chargenprozesse: Verarbeitung auf Teilmengen eines sonst kontinuierlichen Gutes}
  }

  \question{Wie kann die Ablaufstruktur sein}{\page{9}}
  {
    \catchword{Deterministisch}
    \catchword{Stochastisch}
  }

  \question{Wie definiert man deterministische Prozesse}{\page{9}}
  {
    \catchword{Zeitliche Abfolge liegt fest}
    \catchword{Sind vorhersagbar}
  }

  \question{Wie definiert man stochastische Prozesse}{\page{9}}
  {
    \catchword{Keine vorhersagbare Abfolge}
    \catchword{Gehorcht den Gesetzen der Statistik}
  }

  \statement{Nennen und erklären Sie kurz die Verarbeitungsbereiche}{\page{9}}
  {
    \catchword{Fertigung: Verarbeitung von Gütern}
    \catchword{Verfahrenstechnik: Kontinuierliche oder chargenweise Verarbeitung
               von Gütern oder Energie}
    \catchword{Verteilungsprozesse: Transport und Speicherung von Materie, Energie und Information}
    \catchword{Verwaltung: Verarbeitung von Informationen}
  }

  \subsubsection{Prozessrechner}

  \question{Wie ist ein Rechner laut DIN definiert}{\page{10}}
  {
    \normaltext{Ein Rechensystem, das aus einer Zentraleinheit (mit Prozessor, Zentralspeicher
                und Ein-/Ausgabewerken) und der notwendigen Peripherie (Ein-/Ausgabegeräte,
                Massenspeicher und Prozessperipherie) besteht}
  }

  \question{Wie lautet die DIN Definition eines Prozessrechners}{\page{10}}
  {
    \normaltext{Als Prozessrechner wird ein Rechner bezeichnet, der mittels Prozessperipherie direkt
                an einen Prozess gekoppelt ist.}
  }

  \statement{Geben Sie die Prozessgrößen an}{\page{11}}
  {
    \catchword{Prozess-Kennwerte}
    \catchword{Prozessparameter}
    \catchword{Prozesszustand}
    \catchword{Eingangsgrößen}
    \catchword{Ausgangsgrößen}
  }

  \subsubsection{Prozessdaten und Prozessgrößen}

  \question{Wie unterscheiden sich Prozessdaten und Prozessgrößen}{\page{11}}
  {
    \catchword{Prozessdaten werden zwischen Prozess und Prozessrechner ausgetauscht, und sind
               dem Rechner zugeordnet}
    \catchword{Prozessgrößen beschreiben einen technischen Prozess unabhängig von seiner Lenkung
               durch einen Rechner}
  }

  \question{Was sind Prozessgrößen}{\page{11}}
  {
    \normaltext{Technische, physikalische oder chemische Größen, die mit technischen Mitteln
                erfasst und beeinflusst werden können, und durch Zahlenwerte (Daten) beschrieben
                werden können}
  }

  \question{Welche Verarbeitungsarten gibt es, und welche Größen sind charakteristisch}{\page{12}}
  {
    \catchword{Speicherprozess \resultol{die Menge}}
    \catchword{Transport \resultol{der Durchsatz}}
    \catchword{Umformung \resultol{Art und Umfang der Änderung}}
  }

  \question{Was sind Prozesskennwert}{\page{12}}
  {
    \normaltext{Prozesskennwerte oder Anlagedaten sind technische Daten, die durch den Aufbau eines
                Prozesses, seine Struktur, Konstruktion und Anwendung festgelegt sind. Also Festwerte,
                zu denen vor allem auch Grenzwerte gehören.}
  }

  \question{Was sind Prozesszustandswerte}{\page{12}}
  {
    \normaltext{Prozesszustandwerte sind die im Betrieb auftretenden veränderlichen Größen, die den
                aktuellen Zustand des Prozesses beschreiben, und mit technischen Mitteln erfasst
                und beeinflusst werden können.}
  }

  \question{Was sind Prozessparameter}{\page{12}}
  {
    \normaltext{Prozessparameter stellen eine Zwischenstufe dar, also solche Größen, die in verschiedenen
                Prozessablaufen verschiedene Werte annehmen können.}
  }

  \question{Wonach kann man Betriebsgrößen unterscheiden}{\page{12}}
  {
    \catchword{Flussrichtung}
    \catchword{Eingangsgrößen}
    \catchword{Ausgangsgrößen}
  }

  \statement{Beschreiben Sie die Eingangsgrößen eines Prozesses}{\page{12}}
  {
    \catchword{Fließen in den Prozess hinein}
    \catchword{Wirken auf den Prozess}
    \catchword{Unabhängig vom Prozesszustand}
    \catchword{Kann kontinuierlich, stückweise oder in Chargen erfolgen}
  }

  \question{Welche Arten von Eingangsgrößen gibt es}{\page{12}}
  {
    \catchword{Stellgrößen: Wirken mehr oder minder direkt auf die Prozesszustandsgrößen}
    \catchword{Störgrößen: Treten meist als unerwünschte und oft schwer kontrollierbare Störungen auf}
  }

  \statement{Beschreiben Sie die Ausgangsgrößen eines Prozesses}{\page{13}}
  {
    \catchword{Fließen aus dem Prozess hinaus}
    \catchword{Können noch auf den Prozess zurückwirken}
    \catchword{Stets abhängig vom aktuellen Zustand des Prozesses}
    \catchword{Abhängig von den Stellgrößen}
  }

  \question{Welche Anforderungen werden an die Messgrößen gestellt}{\page{13}}
  {
    \catchword{Aktuelle Zustandsgrößen sollen unabhängig von eineinander sein}
    \catchword{Sollen keine Rückwirkungen auf den Prozess haben}
  }

  \question{Aus welchen Komponenten bestehen Prozessdaten}{\page{14}}
  {
    \normaltext{Eingabe- und Ausgabedaten des Prozessrechners}
  }

  \question{In welcher Form können Eingabedaten vorkommen}{\page{14}}
  {
    \catchword{Digital}
    \catchword{Analog}
    \catchword{Alarm durch nur ein Bit}
  }

  \subsubsection{Prozesskopplung}

  \question{Welche Arten der Prozesskopplung gibt es}{\pages{14}{16}}
  {
    \catchword{Indirekt gekoppelter Betrieb (engl.: off-line) wobei die Daten zwischen Rechner und Prozess
               mittels Datenträger übermittelt werden.}
    \catchword{Direkt gekoppelter Betrieb (engl.: on-line) mit direktem Datenaustausch
               zwischen Rechner und Prozess}
    \catchword{Offener Prozessbetrieb (engl.: open loop control) wobei der Fluss der Daten
               einseitig ist und noch teilweise die Mitarbeit des Menschen erfordert.}
    \catchword{Geschlossener Prozessbetrieb (engl.: closed loop control) ist ein vollständiger
               Regelkreis, bei dem Istwerte vom Prozess aufgenommen, mit den Sollwerten im
               Rechner vergleichen und daraus wieder Steuerdaten für den Prozess gewonnen werden.}
  }

  \question{Wie kann man den Prozessbetrieb weiter unterteilen}{\page{15}}
  {
    \catchword{Messwerterfassung: Hier übernimmt der Rechner nur Daten vom Prozess,
               um sie weiter zu verarbeiten}
    \catchword{Prozesssteuerung: Hier werden ausschließlich Daten vom Rechner zum Prozess übertragen}
  }

  \question{Warum ergeben sich beim geschlossenen Prozessbetrieb höhere Anforderungen
            als im offenen Prozessbetrieb}{\page{16}}
  {
    \normaltext{Da hier 2 gegenläufige Datenströme verarbeitet werden müssen}
    \catchword{Eingabedaten: Sollwerte vom Menschen, und Istwerte vom Prozess}
    \catchword{Verarbeitung: Bestimmung des Prozesszustands, Vergleich von Ist- und Sollwerten,
               Berechnung der notwendigen Aktionen zum Abgleich im Prozess und Umwandlung in Prozessdaten}
    \catchword{Ausgabedaten: Steuerdaten an den Prozess und Zustandsdaten an den Menschen}
  }

  \subsection{Prozessmodelle}

  \question{Welche Komponenten müssen in einem Prozessmodell vorhanden sein}{\page{17}}
  {
    \catchword{Eingangsdaten}
    \catchword{Ausgangsdaten}
    \catchword{Zustandsdaten}
    \result{Zusammenhänge zwischen diesen Größen}
  }

  \question{Welche zwei Arten von Modellen gibt es und wie können diese Überprüft werden}{\page{17}}
  {
    \catchword{Gegenständliche Modell}
    \catchword{Mathematisches Modell}
    \result{Simulation mit einem Computer}
  }

  \subsubsection{Prozessbeschreibung}

  \question{Was muss bei der Prozessbeschreibung gemacht werden}{\page{17}}
  {
    \catchword{Strukturierung des Prozesses in Teilprozesse}
    \catchword{Verfeinerung bis zu Elementarprozessen}
    \catchword{Beschreibung ihrer Zusammenhänge}
  }

  \question{Wie werden statische Strukturen dargestellt}{\page{17}}
  {
    \normaltext{Meistens durch ein, an den jeweiligen Einsatzbereich angepasstes Blockschaltbild}
  }

  \question{Wie werden stationäre Abläufe dargestellt}{\page{17}}
  {
    \normaltext{Im Blockdiagramm als Wirkungs- oder Transportwege}
  }

  \question{Welchen Unterschied gibt es bei den Ablaufstrukturen gegenüber
            den bekannten Programmablaufplänen}{\page{17}}
  {
    \normaltext{Parallelverarbeitung}
  }

  \statement{Welche Arten von Bearbeitung gibt es}{\page{18}}
  {
    \catchword{Sequentielle Bearbeitung: Immer nur eine Station aktiv, das Bearbeitungsstück wird
               von einer zur nächsten Station weitergegeben}
    \catchword{Serielle Bearbeitung: Alle Stationen aktiv, die Bearbeitungsstücke werden von einer
               zur nächsten Station weitergegeben}
    \catchword{Parallele Bearbeitung: Entweder identische Vorgänge zur Steigerung des Durchsatzes,
               oder Vorgänge die zu einem Endprodukt gebraucht werden}
  }

  \subsubsection{Prozesserkennung}

  \question{Welche Arten der Prozesserkennung gibt es}{\page{19}}
  {
    \catchword{Empirische Prozesserkennung \resultol der technische Prozess ist bereits realisiert, und soll 
               nachträglich automatisiert werden}
    \catchword{Analytische Modellierung \resultol der Ablauf wird durch bekannte physikalische oder chemische
               Zusammenhänge (Gleichungen) dargestellt. }
  }

  \question{In welchen Ebenen kann die Erstellung eines Modells erfolgen}{\page{19}}
  {
    \catchword{Als statisches Modell, in dem die Prozessgrößen, Eingangs-, Ausgangsgrößen und Zustandsgrößen
               definiert und beschrieben werden.}
    \catchword{Als stationäres Modell, in dem die Zusammenhänge zwischen den Prozessgrößen für den stationären 
               Zustand beschrieben werden, also für den Fall, dass alle Größen im wesentlichen stabil sind.}
    \catchword{Als dynamische Modell, in dem die zeitlichen Veränderungen der Prozessgrößen durch
               innere oder äußere Wirkungen beschrieben werden, insbesondere auch Einschalt- 
               und Einschwingvorgänge.}
  }

  \question{Wodurch entsteht die Unvollkommenheit eines Modells, und wie kann diese
            verringert werden}{\page{20}}
  {
    \catchword{Sie entsteht dadurch, dass die Störgrößen nicht genügend genau bekannt sind.}
    \catchword{Dies kann durch geeignete Analysen und Simulationen verringert werden}
  }

  \question{Was bezeichnet man als lernenden Prozessmodell}{\page{20}}
  {
    \catchword{Modelle, die nicht nur passiv die Parameter auf Grund von Prozessdaten bestimmen,
               sondern die optimalen Werte aktiv selber sucht.}
  }

  \subsubsection{Verteilte Verarbeitung}

  \statement{Erklären Sie die Verteilte Verarbeitung}{\page{20}}
  {
    \catchword{Unterteilung von komplexen Prozessen in Teilbereiche}
    \catchword{Für Teilbereiche einen eigenen hochspezialisierten Prozessrechner einsetzen,
               der einen höheren Wirkungsgrad haben sollte}
  }

  \question{Was muss geschehen, um eine gemeinsame Prozesslenkung zu erhalten und
            was ergibt sich daraus}{\page{20}}
  {
    \normaltext{Koppelung der Einzelrechner, die durch einen Leitrechner geführt werden}
    \result{Eine hierarchische Baumstruktur entsteht}
  }

  \question{Welche Komponente sitzt noch über dem Leitrechner}{\page{21}}
  {
    \normaltext{Betriebsrechner}
  }

  \question{Welche Vorteile ergeben sich aus der Verteilung der Aufgaben}{\page{21}}
  {
    \catchword{Leistungssteigerung}
    \catchword{Höhere Verfügbarkeit \resultol{Leitrechner kann einspringen}}
  }

  \subsection{Ein einfaches Beispiel, das Parkhaus}

  \subsubsection{Problembeschreibung}

  \subsubsection{Lösungsansätze}

  \question{Welche Lösungsansätze gibt es um ein Beispiel zu lösen}{\page{24}}
  {
    \catchword{Stationäre Lösung: Dauerhafte Erfassung der Zustandgrößen}
    \catchword{Dynamische Lösung: Indirekte Erfassung über Änderungen und einen Anfangszustand}
  }

  \question{Welches Problem ergibt sich bei dynamischen Lösungen, bezogen auf die Zeit}{\page{26}}
  {
    \catchword{Die Erfassung der Eingabedaten erfolgt nicht gleichzeitig}
    \result{Programmzeiten müssen klein im Vergleich zu den Prozesszeiten sein}
  }

  \question{Welche Arten von Kommunikation gibt es zwischen Rechner und Peripherie}{\pages{26}{27}}
  {
    \catchword{Abfragebetrieb (Polling)}
    \catchword{Interrupt-Betrieb}
  }

  \subsubsection{Praktische Probleme}

  \question{Welche Probleme ergeben sich bei eintreffenden Ereignissen}{\page{29}}
  {
    \catchword{Kurz hintereinander: Ereignisse können verloren gehen}
    \catchword{Selten: Rechner ist mit unproduktiven Warten beschäftigt}
  }

  \subsection{Praktische Anforderungen an die PDV}

  \question{Was ist die Grundforderung an die PDV}{\page{30}}
  {
    \normaltext{Der Rechner muss dem Prozessgeschehen, insbesondere seinem wirklichen
                Zeitverhalten angepasst sein. Dies wird als Realzeitbetrieb bezeichnet.}
  }

  \newpage
  \section{Prozessrechner Hardware}

  \subsection{Rechnerkonfiguration}

  \question{Aus welchen zwei Einheiten besteht ein Prozessrechner}{\page{33}}
  {
    \catchword{Zentraleinheit}
    \catchword{Peripherie}
  }

  \question{In welche Gruppen können die peripheren Geräte eingeteilt werden}{\page{34}}
  {
    \catchword{Ein-/Ausgabegeräte}
    \catchword{Externspeicher oder Massenspeicher}
    \catchword{Prozessperipherie}
    \catchword{Kommunikationsperipherie}
  }

  \subsection{Die Zentraleinheit}

  \question{Aus welchen drei Komponenten besteht die Zentraleinheit}{\page{35}}
  {
    \catchword{CPU}
    \catchword{Hauptspeicher}
    \catchword{Anschlüsse für periphere Geräte}
  }

  \question{Aus welchen zwei Komponenten besteht die CPU}{\page{36}}
  {
    \catchword{Steuerwerk (Leitwerk)}
    \catchword{Rechenwerk}
  }

  \statement{Nennen Sie die drei wesentlichen Merkmale einer CPU}{\page{36}}
  {
    \catchword{Datenstruktur}
    \catchword{Instruktionssatz}
    \catchword{Geschwindigkeit}
  }

  \question{Um welches Prinzip handelt es sich, wenn Daten und Instruktionen gleichwertig
            im Speicher stehen}{\page{36}}
  {
    \normaltext{Man spricht vom \important{von-Neuman}-Prinzip}
  }

  \question{Worum handelt es sich bei einem Bus}{\page{37}}
  {
    \normaltext{Um einen gemeinsamen Datenpfad}
  }

  \question{Welche Bus-Arten gibt es}{\page{38}}
  {
    \catchword{Synchrone Busse: Feste Zeitvorgabe durch einen Takt}
    \catchword{Asynchrone Busse: Arbeiten mit einem Quittierungsverfahren}
  }

  \subsection{Die CPU}

  \subsubsection{Der Aufbau der CPU}

  \question{Welche Werke vereint die CPU}{\page{39}}
  {
    \catchword{Steuerwerk (Leitwerk)}
    \catchword{Rechenwerk}
    \catchword{Evtl. Unterbrechungswerk}
    \catchword{Evtl. Rechenwerk für Gleitkommazahlen}
  }

  \question{Welche Arten von Leitungen gibt es in der CPU}{\page{39}}
  {
    \catchword{Adressleitung \resultol{nur zum Senden von Signalen (unidirektional)}}
    \catchword{Datenleitung \resultol{zum Senden und Empfangen (bidirektional)}}
    \catchword{Steuerleitung \resultol{sowohl uni- als auch bidirektional ausgelegt}}
  }

  \question{Welche Hilfregister besitzt das Steuerwerk}{\page{39}}
  {
    \catchword{Instruktionsregister}
    \catchword{Program Counter}
    \catchword{Stack Pointer}
    \catchword{Indexregister}
  }

  \statement{Erklären Sie den Aufbau der CPU (schematisch)} {\page{40}}
  {
    \normaltext{Siehe Bild 2.3 Seite 40}
  }

  \question{Wozu dient das Unterbrechungswerk}{\page{40}}
  {
    \catchword{Zur Bearbeitung von Interrupts, die hier über ein Bit im
               Prozessor-Status-Wort zugelassen oder verhindert werden können}
  }

  \subsubsection{Das Rechenwerk}

  \question{Wie heißt das Kernstück des Rechenwerks und wozu dient es}{\page{41}}
  {
    \normaltext{ALU: Durchführung von Operationen auf Operanden}
  }

  \question{Welche zwei Arten von Operationen gibt es bezogen auf die Operanden}{\page{42}}
  {
    \catchword{Binäre Operationen: Zwei Quelloperanden z.B. OR}
    \catchword{Unäre Operationen: Ein Quelloperand z.B. NOT}
  }

  \question{Welche drei Klassen von Operationen gibt es}{\page{42}}
  {
    \catchword{Boolsche}
    \catchword{Arithmetische}
    \catchword{Logische}
  }

  \statement{Nennen Sie die vier Flaggen, die erzeugt werden}{\page{43}}
  {
    \catchword{Zero Bit: Zeigt an, dass das Ergebnis Null war}
    \catchword{Negativ Bit: Zeigt an, dass das Ergebnis negativ war}
    \catchword{Carry Bit: Übertrag, wenn mit positiven Zahlen gerechnet wird}
    \catchword{Overflow Bit: Zahlenüberlauf, wenn mit Zahlen im 2-er Komplement gerechnet wird}
  }

  \subsubsection{Maschineninstruktionen}

  \question{Was versteht man unter Maschineninstruktionen}{\page{44}}
  {
    \normaltext{Informationen, die das Steuerwerk braucht, um Instruktionen auf
                Maschinenebene durchzuführen}
  }

  \question{Aus welchen Teilen besteht jede Instruktion}{\page{44}}
  {
    \catchword{Operationsteil}
    \catchword{Adressteil}
  }

  \question{Welche drei Gruppen von Instruktionen können unterschieden werden}{\pages{44}{45}}
  {
    \catchword{Steruerungsinstruktionen}
    \catchword{Verarbeitungsoperationen}
    \catchword{Verzweigungen}
  }

  \question{Was dient bei bedingten Sprüngen zur Überprüfung der Sprungbedingungen}{\page{45}}
  {
    \normaltext{Die entsprechenden Status Flags der ALU}
  }

  \question{Was wird bei der Speicheradressierung stets verwendet}{\page{46}}
  {
    \normaltext{Ein internes Register: Program Counter (PC), Indexregister (IX)
                oder der Stack Pointer (SP)}
  }

  \statement{Erklären Sie die Adressierungsmodi des Program Counters}{\pages{46}{47}}
  {
    \catchword{Direkt: Der Operand erfolgt unmittelbar auf die Instruktion, der PC zeigt drauf}
    \catchword{Absolut: Die absolute Adresse des Operanden folgt unmittelbar auf die Instruktion}
    \catchword{Relativ: Die relative Adresse des Operanden folgt unmittelbar auf die Instruktion,
               die absolute Adresse muss daraus durch Addition des PC erst berechnet werden}
    \catchword{Indirekt: Die absolute Adresse der absoluten Adresse des Operanden folgt unmittelbar
               auf die Instruktion}
  }

  \question{Was wird beim Unterprogrammsprung gemacht}{\page{47}}
  {
    \catchword{Aktueller PC auf dem Stack ablegen}
    \catchword{Stack Pointer erhöhen}
    \catchword{Unterprogramm ausführen}
    \catchword{Rücksprungadresse vom Stack holen}
    \catchword{Stack Pointer erniedrigen}
  }

  \subsubsection{Das Mikroprogramm}

  \question{Welche wesentlichen drei Aufgaben hat das Mikroprogramm}{\page{48}}
  {
    \catchword{Einlesen von Instruktionen und Daten aus dem Hauptspeicher, und das Abspeichern 
               von Ergebnissen}
    \catchword{Interpretieren von Instruktionen, wobei es in eigene Programmteile (Module) verzweigt}
    \catchword{Steuern der CPU d.h. des Datenverkehrs innerhalb der CPU, und ins besondere 
               des Rechenwerks (ALU)}
  }

  \question{Was muss mit neuen Instruktionen geschehen}{\page{49}}
  {
    \catchword{Instruktion lesen}
    \catchword{Instruktion interpretieren}
  }

  \question{Welche zwei Arten von Verzweigungen gibt es}{\page{50}}
  {
    \catchword{Unbedingter Sprung}
    \catchword{Bedingter Sprung}
  }

  \question{Wie bezeichnet man Unterprogramme des Mikroprogramms}{\page{51}}
  {
    \catchword{Nanoprogramme}
  }

  \subsubsection{RISC}

  \question{Wofür steht die Abkürzung RISC}{\page{52}}
  {
    \normaltext{Reduced Instruction Set Computing}
  }

  \question{Wofür steht die Abkürzung CISC}{\page{52}}
  {
    \normaltext{Complex Instruction Set Computing}
  }

  \question{Worin unterscheiden sich CISC und RISC}{\pages{52}{53}}
  {
    \catchword{Load-and-Store-Architektur}
    \catchword{Pipelining (Einteilung in 5 Phasen: IF, RD, OP, MM, WT)}
  }

  \question{In welche Klassen ist der Instruktionssatz bei RISC Prozessoren aufgeteilt}{\page{52}}
  {
    \catchword{Instruktionen zum Speicherzugriff (LOAD, STORE)}
    \catchword{Verarbeitungsinstruktionen (ADD, SUB)}
    \catchword{Verzweigungsinstruktionen (JMP, BNE)}
    \catchword{Steuerungsinstruktionen (HALT, WAIT)}
  }

  \question{Was kann zu besonderen Problemen beim Pipelining führen}{\pages{53}{54}}
  {
    \catchword{Verzweigung innerhalb des Programms (synchron)}
    \catchword{Unterbrechung von außen (asynchron)}
    \result{Dann müssen solche Instruktionen, die bereits begonnen wurden, abgebrochen
            werden; bei manchen geht das nicht, und es muss die Verarbeitungskette angehalten
            werden (pipe stalling)}
  }

  \statement{Nennen Sie Eigenschaften von RISC in der PDV}{\page{54}}
  {
    \catchword{Da die Verarbeitunsoperationen auf Daten im Arbeitsspeicher immer zusätzlich
               eine LOAD- und eine STORE-Operation erfordern, werden die Maschinenprogramme
               deutlich länger.}
    \catchword{Erheblicher zusätzlicher Verwaltungsaufwand bei bedingten Sprüngen}
  }

  \subsection{Der BUS}

  \subsubsection{Aufbau des Busses}

  \question{Welche drei Gruppen von Leitungen benutzt ein Bus}{\page{55}}
  {
    \catchword{Datenleitungen}
    \catchword{Adressleitungen}
    \catchword{Steuerleitungen}
  }

  \question{Worin liegt die besondere Eignung des Busprinzips}{\page{55}}
  {
    \normaltext{Große Flexibilität beim Einrichten oder bei Änderungen}
  }

  \question{Was ist der größte Nachteil des Busses}{\page{55}}
  {
    \normaltext{Alle Daten werden über ihn transportiert, dabei kann ein Engpass entstehen}
  }

  \question{Woraus besteht ein Bus}{\page{55}}
  {
    \normaltext{Aus Leitungsbündeln, mit denen die Steckplätze für die Komponenten verbunden sind}
  }

  \subsubsection{Technische Ausführung}

  \question{Was sollte mit Teilen der Zentraleinheit geschehen}{\page{55}}
  {
    \catchword{Möglichst nahe an den technischen Prozess bringen}
    \catchword{Möglichst weit weg von Störeinflüssen}
    \result{Stehen manchmal im Widerspruch zueinander}
  }

  \question{Warum dürfen bestimmte Längen nicht überschritten werden}{\page{56}}
  {
    \normaltext{Wegen der Signallaufzeit (etwa 5 ns/m)\resultol{Kollisionen auf dem Bus}}
  }

  \question{Welche Störungen können in Kabeln auftreten}{\page{56}}
  {
    \catchword{Signalverfälschungen durch äußere Störungen}
    \catchword{Interne Reflexionen}
  }

  \question{Wie können diese Störungen verhindert werden}{\page{56}}
  {
    \catchword{Abgeschirmte Leitungen}
    \catchword{Verdrillte Leitungen mit definiertem Wellenwiderstand}
  }

  \statement{Nennen Sie die technischen Merkmale eines Busses}{\page{56}}
  {
    \catchword{Zahl der Leitungen}
    \catchword{Verwendung jeder Leitung als Daten-, Adress-, oder Steuerleitung}
    \catchword{Belegung der Steckplätze}
    \catchword{Art der Signale}
  }

  \subsubsection{Protokolle}

  \question{Welche Arten von Bussen gibt es}{\page{56}}
  {
    \catchword{Synchrone Busse (festes Zeitraster wird vorgegeben, richtet sich nach der 
               langsamsten Komponente)}
    \catchword{Asynchrone Busse (Quittierungsverfahren - Handshaking, nutzt die individuellen
               Geschwindigkeiten der Komponenten aus)}
  }

  \subsubsection{Multiplexverfahren}

  \statement{Erklären Sie Multiplexverfahren}{\page{59}}
  {
    \normaltext{Für Daten und Steuerleitungen werden dieselben Leitungen im zeitlichen Wechsel
                benutzt. Dazu wird eine weitere Steuerleitung zur Signalisierung des Wechsels benötigt.}
  }

  \subsubsection{Konkrete Bussysteme}

  \subsection{Arbeitsspeicher}

  \subsubsection{Speicherorganisation}

  \subsubsection{Speichersteuerung}

  \subsubsection{Speicherelemente}

  \subsubsection{Erweiterung des Adressbereichs}

  \subsubsection{Cache Memory}

  \subsection{Geräteanschlüsse}

  \statement{Nennen Sie die Aufgabe der Geräteanschlüsse}{\page{82}}
  {
    \normaltext{Ankopplung von externen Geräten verschiedenster Art}
  }

  \question{Wie erfolgt die Kommunikation mit der Peripherie}{\page{82}}
  {
    \catchword{Zustand des Gerätes muss abgefragt werden}
    \catchword{Oder von dem Gerät gemeldet werden}
  }

  \question{Was ist die eigentliche Aufgabe eines Geräteanschlusses}{\page{82}}
  {
    \catchword{Anpassung der Daten, der Adressen und Steuersignale}
    \catchword{Steuerung des Datenflusses}
  }

  \statement{Nennen Sie die fünf verschiedenen Geräteanschlüsse}{\page{83}}
  {
    \catchword{Ein-/ Ausgabe-Prozessoren}
    \catchword{Blockmultiplexer}
    \catchword{Controller}
    \catchword{Byte-Multiplexer}
    \catchword{Interfaces}
  }

  \subsubsection{Die Schnittstelle zum Bus}

  \question{Wozu wird die Schnittstelle zum Bus verwendet}{\page{83}}
  {
    \catchword{Gesamte Datenübertragung zum und vom Gerät}
    \catchword{Gesamte Steuerung des Gerätes}
    \catchword{Steuerung des Geräteanschlusses selbst}
  }

  \question{Welches sind die wichtigsten Register eines Geräteanschlusses}{\page{85}}
  {
    \catchword{Data Buffer Register (DBR) ist ein Datenpuffer zur Zwischenspeicherung von Daten}
    \catchword{Control and Status Register (CSR) enthält einzelne Bits für die Steuerung des Gerätes}
  }

  \subsubsection{Übertragungsverfahren}

  \statement{Nennen Sie die zwei Übertragungsverfahren}{\page{86}}
  {
    \catchword{Einzelzeichenübertragung: Bei zeichenorientierten Geräten}
    \catchword{Blockübertragung: Vorwiegend bei Massenspeicher}
  }

  \question{Welche zwei Arten von Datenübertragung zwischen Controller und Speicher gibt es}{\page{87}}
  {
    \catchword{Blockpufferung: Ankommende Daten werden im Puffer zwischengespeichert und dort abgeholt}
    \catchword{Speicherdirektzugriff (DMA): Meist ohne Puffer und ohne Eingriff der CPU eine
               direkte Kommunikation mit dem Arbeitsspeicher}
  }

  \subsubsection{Geräteschnittstellen}

  \question{Welche Geräteschnittstellen kann man unterscheiden}{\page{88}}
  {
    \catchword{Parallelschnittstellen, auf denen die einzelnen Bits parallel, die Zeichen selber 
               aber immer nacheinander (seriell) übertragen werden. Da für jedes Bit eine Leitung benötigt 
               wird, sind die Anschlüsse und Kabel aufwendig, aber man erreicht eine hohe
               Übertragungsgeschwindigkeit.}
    \catchword{Seriellschnittstellen, auf welchen auch die einzelnen Bits nacheinander übertragen werden. 
               Dadurch können die Daten unabhängig von der Datenbreite über eine einzig Leitung
               übertragen werden. Allerdings entsteht hier ein höherer Aufwand bei der Umsetzung, und 
               die Übertragungsgeschwindigkeit ist niedriger.}
  }

  \statement{Nennen Sie Eigenschaften der synchronen serieller Übertragung}{\page{89}}
  {
    \catchword{Es wird eine lückenlose Folge von Zeichen übertragen.}
    \catchword{Wenn keine aktuellen Daten anliegen werden Füllzeichen übertragen.}
    \catchword{Hohe Datenrate, da Start und Stoppbits entfallen.}
    \catchword{Nachteilig ist das erkennen von Füllzeichen und die Synchronisierung der Taktfrequenz.}
  }

  \statement{Nennen Sie Eigenschaften der asynchronen serieller Übertragung}{\page{89}}
  {
    \catchword{Bits eines Zeichens werden kontinuierlich übertragen, jedoch können zwischen den Zeichen
               beliebig lange Ruhezeiten liegen.}
    \catchword{Anfang und Ende eines Zeichen werden durch Start- und Stoppbit gekennzeichnet.}
    \catchword{Kein erkennen von Füllzeichen und synchronisieren der Taktfrequenz nötig.}
    \catchword{Dafür müssen aber Start- und Stoppbits erkannt werden.}
  }

  \question{Welche Möglichkeiten gibt es Geräte an einen Conroller anzuschließen}{\page{90}}
  {
    \catchword{Busanschluss, auf dem wie in der Zentraleinheit Daten-, Adress- und Steuerleitungen
               für alle angeschlossenen Geräte geführt werden.}
    \catchword{Sternanschluß, bei dem jedes Gerät über einen eigenen Anschluss am Controller verfügt.}
    \catchword{Daisy Chain (Verkettung), bei der zunächst ein Gerät an den Controller angeschlossen wird.
               An dieses Hauptgerät werden weitere identische Geräte in Form einer Kette angeschlossen.}
  }

  \subsubsection{Die Parallelschnittstelle}

  \question{Nennen Sie die Eigenschaften der Parallelschnittstelle}{\page{91}}
  {
    \catchword{Die Bits eines Datenworts werden parallel übertragen}
    \catchword{Nur minimale Anforderungen an die eigentlichen Schnittstellenumsetzer}
    \catchword{Übertragungsrate recht hoch und an keinen Takt gebunden}
    \catchword{Nur für geringe Entfernungen geeignet}
  }

  \question{Welche Pegel verwenden die Signale der Parallelschnittstelle}{\page{92}}
  {
    \normaltext{Alle Signalpegel entsprechen TTL}
  }

  \subsubsection{Die asynchrone serielle Schnittstelle (V.24, RS-232-C)}

  \question{Welche Einheiten verbindet eine serielle Schnittstelle}{\pages{94}{95}}
  {
    \normaltext{Datenendeinrichtung und Datenübertragungseinrichtung}
  }

  \statement{Nennen Sie wichtigsten Arten von Leitungen}{\page{95}}
  {
    \catchword{Ground}
    \catchword{Datenleitungen}
    \catchword{Steuerleitungen}
    \catchword{Taktleitungen}
    \catchword{Sende- und Empfangsleitungen}
  }

  \question{Welche Codierung wird verwendet}{\page{96}}
  {
    \normaltext{NRZ (Non Return to Zero)}
  }

  \question{Wie groß ist die Baudrate in Bezug auf die Bitrate}{\page{98}}
  {
    \normaltext{Wegen der binären Codierung gleich groß}
  }

  \subsubsection{Der Schnittstellenumsetzer UART}

  \question{Warum hat der Schnittstellenumsetzer UART eine besondere Bedeutung}{\page{99}}
  {
    \catchword{Die serielle Schnittstelle ist praktisch an jedem Rechner vorhanden.}
  }

  \question{Welche Aufgabe hat der Sender (Parallel-Seriell-Wandler)}{\page{99}}
  {
    \catchword{Er setzt die im Datenregister DBR-T angelieferten Daten aus der bitparallelen 
               Darstellung in eine serielle um.}
  }

  \question{Welche Aufgabe hat der Empfänger (Seriell-Parallel-Wandler)}{\page{99}}
  {
    \catchword{Er setzt die ankommenden seriellen Daten in die parallele Darstellung um.}
  }

  \question{Welche Teile besitzen sowohl Sender als auch Empfänger }{\pages{99}{100}}
  {
    \catchword{Busschnittstelle}
    \catchword{Geräteschnittstelle}
    \catchword{Pegelumsetzer}
  }

  \subsection{Betriebsarten}

  \subsubsection{Ablaufdarstellung}

  \question{Wie sieht eine sequentielle Abarbeitung in einem A-t-Diagramm aus}{\page{106}}
  {
    \normaltext{Wie eine Gerade, deren Steigung im Wesentlichen von der Rechengeschwindigkeit
                der CPU abhängt}
  }

  \question{Wie sieht ein Programmsprung in einem A-t-Diagramm aus}{\page{106}}
  {
    \normaltext{Programmsprünge erscheinen als senkrechte Linie im Diagramm}
  }

  \statement{Beschreiben Sie den Aufbau eines A-t-Diagramms}{\pages{106}{107}}
  {
    \catchword{Auf der x-Achse die Zeit t}
    \catchword{Auf der y-Achse die Adresse A der zum Zeitpunkt t verarbeiteten Instruktion,
               also den Inhalt des Programmzählers (PC)}
  }

  \question{Wie kann nach Ablauffehlern gesucht werden}{\page{107}}
  {
    \normaltext{Durch Hardware-Monitore, die den Ablauf am Bus verfolgen, entweder durch
                Oszilloskope oder durch Logikanalysatoren}
  }

  \subsubsection{Betriebsarten}

  \statement{Nennen Sie die unterschiedlichen Betriebsarten}{\page{108}}
  {
    \catchword{Einprogrammbetrieb: Der Rechner bearbeitet immer nur ein Programm}
    \catchword{Einbenutzerbetrieb: Nur eine Benutzer gleichzeitig aktiv
               \resultol{keine Zugriffssicherung notwendig}}
    \catchword{Mehrprogrammbetrieb: Mehrere Programme stehen ablaufbereit im Speicher}
  }

  \question{Worauf sollte bei der Programmentwicklung geachtet werden}{\page{108}}
  {
    \normaltext{Dass der Prozessbetrieb nicht behindert oder gestört wird}
  }

  \statement{Nennen Sie die zwei Verfahren, mit denen der Mehrprogrammbetrieb
             realisiert werden kann}{\page{110}}
  {
    \catchword{Time-sharing: CPU wird in Zeitscheiben zugewiesen}
    \catchword{Resource-sharing: Betriebsmittel werden nach Prioritäten oder Wartezeiten vergeben}
  }

  \subsubsection{Der Abrufbetrieb (Polling)}

  \question{Was versteht man unter Polling}{\page{110}}
  {
    \normaltext{Bein Abrufbetrieb (Polling) werden Informationen über den Zustand der Geräte
                oder der Datenübertragung von einem Programm abgerufen. Dazu fragt die CPU in einer Schleife
                die CSRs der Geräteanschlüsse ab. Anhand dieser Informationen verzweigt das Programm in 
                entsprechende Routinen.}
  }

  \question{Nennen Sie Vorteile des Polling Betriebs}{\pages{110}{111}}
  {
    \catchword{Einfachheit der programmtechnischen Realisierung}
    \catchword{Die Hardware kann relativ einfach ausgelegt werden}
  }

  \question{Nennen Sie Nachteile des Polling Betriebs}{\pages{112}{113}}
  {
    \catchword{Die CPU ist die meist Zeit damit beschäftigt, alle CSRs abzufragen.}
    \catchword{Die Zeitauflösung (Genauigkeit) mit der man den Zeitpunkt für das Eintreten
               eines Ereignisses festlegen kann, wird zunächst durch die Zeit für einen Schleifendurchlauf
               vorgegeben.}
    \catchword{Die Reihenfolge von verschiedenen Ereignissen kann verfälscht werden.}
    \catchword{Bei einer notwendigen Änderung z.B. Abschalten oder Hinzufügen eines neuen
               Gerätes wird eine Programmänderung notwendig, die aber nicht im laufenden Betrieb 
               durchgeführt werden kann.}
  }

  \subsubsection{Realzeitverhalten}

  \question{Was ist die grundsätzliche Anforderung an einen Prozessrechner bei
            Realzeitverarbeitung}{\page{113}}
  {
    \normaltext{Die Verarbeitung der Prozessdaten muss schritthaltend erfolgen, d.h. so schnell,
                dass der Prozess nicht auf  die Ergebnisse warten muss. Man könnte auch sagen,
                die Anforderung durch den Prozess besteht darin, dass ein Ereignis abgearbeitet sein muss,
                bevor ein neues eintritt.}
  }

  \question{Was ist die Forderung für einen sicheren Betrieb}{\page{115}}
  {
    \normaltext{Die Summe aus der maximal möglichen Reaktionszeit tr und der längsten Bearbeitungszeit tb
                soll kleiner sein, als der Minimalabstand von Prozessereignissen tp abzüglich der maximalen 
                Stellzeiten ts. Bei Gleichheit liegt der denkbar ungünstige Fall vor, der eintreten kann
                (engl.: worst case).}
  }

  \question{Was kann bei Realzeitanwendungen zum Verstopfen des Rechners fuhren}{\page{115}}
  {
    \normaltext{Häufige Ereignisse mit langer Bearbeitungszeit. Zur Vermeidung sollen solche Ereignisse gar 
                nicht zugelassen werden.}
  }

  \subsubsection{Der Anforderungsbetrieb, Interrupt-Betrieb}

  \question{Was wird im Interrupt-Betrieb durch Ereignisse erzeugt}{\page{117}}
  {
    \normaltext{Es wird eine Unterbrechungsanforderung erzeugt, welcher stattgegeben wird, indem das gerade
                laufende Programm unterbrochen und ein Sprung in ein Hilfsprogramm durchgeführt wird,
                welches als Interrupt Service Routine (ISR) bezeichnet wird.}
  }

  \question{Welche Aufgaben übernimmt die ISR}{\page{117}}
  {
    \catchword{Übernahme von Daten}
    \catchword{Berechnungen}
    \catchword{Ausgabe von Steuerdaten}
    \catchword{Nach Beendigung wieder in das unterbrochene Programm zurück springen}
  }

  \statement{Nennen Sie Unterschiede zwischen Unterprogrammsprüngen und Unterbrechungen}{\pages{117}{118}}
  {
    \catchword{Bei einer Unterbrechung wird der Zeitpunkt des Sprungs durch ein Ereignis bestimmt, und 
               nicht durch das laufende Programm.}
    \catchword{ISRs stellen eigenständige Programme dar, die unabhängig voneinander erstellt und geladen
               werden können.}
    \catchword{Die Verarbeitungsziele der ISRs müssen nichts miteinander zu tun haben, und eine Kommunikation
               ist zunächst nicht erforderlich.}
  }

  \question{Was sind Semaphore}{\page{118}}
  {
    \normaltext{Signalträger, vergleichbar mit den Flaggen im Prozessor-Status-Wort, mit denen ISRs 
                synchronisiert werden können, falls ihre Arbeiten voneinander abhängen.}
  }

  \statement{Nennen Sie Vorteile beim Interrupt-Betrieb}{\page{118}}
  {
    \catchword{Es können andere Programme im Mehrprogrammbetrieb bearbeitet werden, wenn keine Aufgaben für die 
               Prozesslenkung vorliegen.}
    \catchword{ISR können als eigenständige Programme erstellt, in den Arbeitsspeicher geladen, und bei Bedarf 
               entladen werden.}
    \catchword{Die Erfassung von zeitlichen Abläufen wird genauer, sowohl was den Zeitpunkt eines Ereignisses
               betrifft, als auch die zeitliche Abfolge.}
  }

  \statement{Nennen Sie Nachteile beim Interrupt-Betrieb}{\page{118}}
  {
    \catchword{Höherer Aufwand bei der Hardware und bei der Software.}
    \catchword{Die Koordinierung aller Programme ist aufwendiger.}
    \catchword{Für die Planung und Erstellung der Software müssen neue Methoden eingesetzt werden.}
    \catchword{Die Hardware muss geeignete Schaltungen und Verbindungen enthalten.}
  }

  \statement{Skizzieren Sie eine Interrupt-Anforderung}{\page{119}}
  {
    \normaltext{Siehe Bild 2.51 auf Seite 119}
  }

  \statement{Erklären Sie detailliert den Ablauf eines Interrupts}{\pages{119}{120}}
  {
    \catchword{Eine Unterbrechungsanforderung wird von einem Geräteanschluss erzeugt
               und das aktuelle Programm wird unterbrochen}
    \catchword{Der aktuelle Stand des Programmzählers (PC) und der Inhalt der
               Prozessor-Status-Worts (PS) werden auf dem Stack abgelegt. Die
               Adresse dafür enthält der Stack-Pointer (SP)}
    \catchword{Die Unterbrechungsanforderung muss identifiziert werden, dazu dient
               ein Vektor. Mit diesem Vektor wird in der Interrupt-Vektor-Tabelle (IVT)
               die Startadresse (der neue PC) für die zugeordnete ISR gesucht}
    \catchword{Diese Zuordnung kann über eine eigenständige Routine erfolgen,
               die Interrupt-Dispatcher-Routine, welche die Geräte zur Identifizierung
               abfragt}
    \catchword{Diese Methode ist allerdings sehr langsam, deshalb wird in der Regel
               die Zuordnung automatisch durch vektorisierte Interrupts, die bei
               der Interrupt-Anforderungen den Vektor mitliefern}
    \catchword{Mit dem neuen PC aus der IVT erfolgt der Sprung in die ISR. Dabei muss
               ein neuer Prozesszustand (PS) definiert werden}
    \catchword{Nach der Abarbeitung der ISR erfolgt der Rücksprung über eine spezielle
               Rücksprunginstruktion (RTI, Return from Interrupt). Der alte PC und
               PS werden vom Stack geholt und zurückgeladen}
    \catchword{Das unterbrochene Programm wird fortgesetzt}
  }

  \question{Welche Anforderungen werden an die Hardware für den Interrupt-Betrieb gestellt}{\page{120}}
  {
    \catchword{Geräteanschluss müssen in der Lage sein, aus einem vom Gerät ankommenden Ereignis eine 
               Unterbrechungsanforderung zu erzeugen.}
    \catchword{Der Bus muss spezielle Steuerleitungen für die Unterbrechungsanforderungen und für deren
               Bestätigung besitzen.}
    \catchword{Es muss ein geeignetes Protokoll definiert sein, wie die Leitungen zu verwenden sind.}
  }

  \question{Welche 3 wesentlichen Aufgaben hat das Unterbrechungswerk}{\page{122}}
  {
    \catchword{Die Abweisung oder Zulassung von Interrupt-Anforderungen und die Auswahl, wenn mehrere
               gleichzeitig auftreten.}
    \catchword{Die Durchführung des Interrupt-Zyklus auf dem Bus und das Lesen des Interrupt-Vektors.}
    \catchword{Die Aktivierung des Steuerwerks, welches die Interrupt-Sequenz durchführt.}
  }

  \question{Was versteht man unter Non-Maskable Interrupts (NMI)}{\page{122}}
  {
    \normaltext{Unterbrechungen, die nicht ab geblockt werden dürfen. Diese werden dann auf einer 
                besonderen Leitung angefordert}
  }

  \question{Was ist die Aufgabe der Interrupt-Logik im Geräteanschluss}{\page{122}}
  {
    \catchword{Im Falle einer Abweisung dafür sorgen, dass keine Unterbrechungsanforderung verloren geht.}
    \catchword{Die Unterbrechungsanforderungen nach ihrer Erledigung löschen.}
  }

  \question{Was muss gemacht werden, wenn bei der Interrupt-Bearbeitung
            Konflikte auftreten}{\page{122}}
  {
    \catchword{Vorrangregeln}
    \catchword{Vergabe von Prioritäten}
  }

  \question{Welche zwei Probleme können bei eintreffenden Ereignissen auftreten}{\page{122}}
  {
    \catchword{Von einer Ereignisquelle treffen kurzzeitig mehr Ereignisse ein,
               als verarbeitet werden können. }
    \catchword{Von verschiedenen Ereignisquellen treffen praktisch gleichzeitig Ereignisse ein.
               Dann muss eine Reihenfolge der Abarbeitung gewählt werden}
  }

  \question{Wann können Ereignisfolgen nur aufgefangen werden}{\page{122}}
  {
    \normaltext{Nur, wenn die zulässige Antwortzeit genügend groß ist}
  }

  \statement{Nennen Sie zwei Methoden Ereignisfolgen aufzunehmen}{\page{123}}
  {
    \catchword{Ereignisse werden zwischengepuffert}
    \catchword{Die ISR kann von neuen Ereignissen immer wieder unterbrochen werden}
  }

  \question{Welche Probleme ergeben sich, wenn die ISR unterbrochen wird}{\page{123}}
  {
    \normaltext{Sie muss besondere Eigenschaften haben, so dass sie jederzeit wieder
                benutzt werden kann. Sie muss unterbrechbar sein, bevor sie ihre
                Ergebnisse ausgegeben hat. Diese Eigenschaft wird als \important{reentrancy}
                bezeichnet, wörtlich als Wiedereintrittsfähigkeit}
  }

  \question{Was müssen Programme, die mit der Eigenschaft reentrancy
            bezeichnet werden, machen}{\page{124}}
  {
    \normaltext{Alle anfallenden Zwischenergebnisse müssen in jedem Durchlauf gut
                gesichert werden. Das kann dadurch erreicht werden, gleich am Beginn
                der Routine alle benötigten Speicherplätze von Variablen und alle
                CPU-Register auf dem Stack gesichert werden}
  }

  \question{Was muss bei vielen eintreffenden Ereignissen geschehen und zu welchen
            Problemen führt dies}{\page{124}}
  {
    \catchword{Ereignisse müssen in einer Warteschlange gesammelt werden, und können
               dann in der Reihenfolge bearbeitet werden, in der sie eingetroffen
               sind}
    \result{Die Reaktionszeit kann beliebig lang werden}
  }

  \question{Welches Vorgehen liefert bessere Ergebnisse}{\page{124}}
  {
    \normaltext{Besser Ergebnisse erhält man, wenn man auch die ISRs unterbricht,
                allerdings nicht wahllos, sondern es müssen Prioritäten gesetzt werden}
  }

  \question{Wie lautet eine einfache Regel zur Vergabe von Prioritäten}{\page{125}}
  {
    \catchword{Ereignisse mit hoher Folgefrequenz und kurzer Bearbeitungszeit, bekommen
               höhere Prioritäten}
    \catchword{Ereignisse, die selten auftreten, aber eine lange Bearbeitungszeit haben,
               bekommen niedrigere Prioritäten}
    \result{Diese Regel muss aber häufig durchbrochen werden}
  }

  \question{Was muss bei einer Vorrangregel geschehen}{\page{125}}
  {
    \normaltext{Unterbrechungsanforderungen müssen zugelassen oder abgewiesen werden}
  }

  \question{Was für den Impuls von eintreffenden Ereignissen gelten}{\page{125}}
  {
    \normaltext{Der Länge des Impuls sollte möglichst kurz sein gegenüber allen
                Prozess- und Verarbeitungszeiten}
  }

  \question{Worüber kann festgelegt werden, ob ein Interrupt zugelassen
            oder abgewiesen wird}{\page{125}}
  {
    \normaltext{Durch das Interrupt-Enable-Bit (IE) im Interface}
  }

  \statement{Skizzieren Sie die Interrupt Verkettung}{\page{126}}
  {
    \normaltext{Siehe Bild 2.56 auf Seite 126}
  }

  \statement{Erklären Sie das Prinzip der Daisy Chain bei Interrupts}{\page{126}}
  {
    \normaltext{Bei der \important{Daisy Chain} werden die Anschlüsse hardwaremäßig
                verkettet, so dass sich eine Rangfolge der Interrupt-Anforderungen
                ergibt. Hierbei hat der Anschluss, welcher der CPU am nächsten liegt
                die höchste Priorität.}
    \result{Die Rangfolge ist dabei statisch}
  }

  \question{Welche Komponente fällt bei dynamischen Prioritäten die Entscheidung,
            welcher Interrupt der wichtigste ist}{\page{127}}
  {
    \normaltext{Das Unterbrechungswerk}
  }

  \question{Welche Methoden zur Auswahl von Interrupt-Anforderungen gibt es}{\page{128}}
  {
    \catchword{Maskierung}
    \catchword{Prioritäten}
  }

  \question{Wie funktioniert Maskierung}{\page{128}}
  {
    \catchword{Die Interrupt-Leitungen des Busses werden uncodiert jeweils einem Bit in einem 
               Interrupt-Register zugeführt, das eine Maskenfunktion hat.}
    \catchword{Wenn das entsprechende Maskenbit gesetzt ist, wird der Interrupt zugelassen.}
    \catchword{Da hier noch Konkurrenzen auftreten können, muss eine zusätzliche Prioritätenregelung 
               eingeführt werden.}
  }

   \question{Wie funktioniert Prioritätenregelung nach Rangordnung}{\pages{128}{129}}
  {
    \catchword{Die ankommenden Interrupt-Leitungen werden als Dualzahlen codiert und dadurch auf wenig 
               Bit komprimiert.}
    \catchword{Diese wird als mehrfaches logisches ODER überlagert, oder in ihrer Gesamtheit gleich 
               als Dualzahl interpretiert.}
    \catchword{Diese Zahl wird als Hardware-Priorität mit der aktuellen Software-Priorität numerisch 
               verglichen.}
    \catchword{Wenn die Hardware Priorität größer als die Software-Priorität ist, wird der Interrupt
               zugelassen. (Das hat zur Folge, dass alle Interrupts abgewiesen werden könne, wenn
               im PSW ein genügend großer Wert eingetragen wird)}
  }

  \statement{Nennen Sie die Grundregeln für ISR bei Realzeitverhalten}{\pages{131}{132}}
  {
    \catchword{Die ISR muss die Prozessdaten so schnell wie möglich abholen (lesen) und abspeichern.}
    \catchword{Während dieser Zeit soll sie durch eine genügend hohe Priorität nicht unterbrechbar sein, 
               auf jeden Fall nicht durch ein gleichartiges Ereignis, das diese ISR benutzt.}
    \catchword{Die Ausgabewerte müssen so schnell wie möglich berechnet und ausgegeben werden.}
  }

  \question{Aus welchen zwei Werten wird die Interrupt-Anforderung IRQ gebildet}{\page{132}}
  {
    \normaltext{Aus der logischen UND-Verknüpfung von R- und IE-Bit}
  }

  \question{Wie kann verhindert werden, dass eine Interrupt-Behandlung unterbrochen wird}{\page{134}}
  {
    \normaltext{Indem am Anfang der Interrupt-Behandlung das IE-Bit auf 0 gesetzt wird.
                Das R-Bit wird dann zwar gesetzt, aber der Interrupt wird nicht zugelassen.
                Unmittelbar vor dem Ende der ISR wird der Interrupt wieder freigegeben.}
  }

  \subsubsection{Interne Ereignisse (Traps)}

  \statement{Nennen Sie vorbeugende Maßnahmen gegen Netzausfall}{\page{136}}
  {
    \catchword{Einsatz von Unterbrechungsfreier Stromversorgung (USV).}
    \catchword{Einsatz von Notstromaggregaten.}
    \catchword{Übergang aller Systeme in einen sicheren Zustand, bei dem die Folgeschäden minimal
               bleiben (fail-safe Technik).}
  }

  \statement{Nennen Sie die Aktionen beim Netzausfall}{\page{136}}
  {
    \catchword{Der Prozess muss in einen sicheren Zustand gebracht werden, wobei auf eine korrekte
               Reihenfolge geachtet werden muss.}
    \catchword{Der aktuelle Zustand des Prozesses sollte erfasst und gesichert werden.}
    \catchword{Der Zustand des Programms sollte gesichert werden, d.h. an welcher Stelle er mit
               welchen Werten unterbrochen wurde.}
  }

  \statement{Nennen und erklären Sie kurz die auftretenden Fehler}{\pages{140}{142}}
  {
    \catchword{Diese Methode erfordert besondere Intelligenz beim Controller.}
    \catchword{Die CPU muss darauf vorbereitet sein.}
    \catchword{Auf dem Bus muss ein geeignetes Protokoll definiert sein.}
    \catchword{Der Arbeitsspeicher muss nur in einigen Ausnahmefällen besonders ausgestattet sein.}
  }

  \question{Was ist die besondere Eigenschaft bei Systemaufrufen von Interrupts und Traps}{\page{142}}
  {
    \normaltext{Sie springen Hilfsprogramme (Handler) an, ohne dass deren Startadresse bekannt ist.
                Sie haben stattdessen einen Zeiger auf eine Tabelle der die Startadresse
                entnommen werden kann.}
  }

  \subsubsection{Der Speicherdirektzugriff (DMA)}

  \question{Was versteht man unter Direct Memory Access (DMA)}{\page{145}}
  {
    \normaltext{Um die CPU zu entlasten werden Daten direkt zwischen Controller und Arbeitsspeicher
                ausgetauscht.}
  }

  \question{Was für Anforderungen werden an die Hardware bei DMA gestellt}{\page{145}}
  {
    \catchword{Eine besondere Fähigkeit (Intelligenz) des Controllers.}
    \catchword{Die CPU muss darauf vorbereitet sein.}
    \catchword{Auf dem Bus muss ein entsprechendes Protokoll definiert sein.}
    \catchword{Der Arbeitsspeicher muss nur in Ausnahmefällen besonders ausgestattet sein.}
  }

  \statement{Nennen Sie die drei charakteristischen Abschnitte des DMA-Betriebs}{\page{145}}
  {
    \catchword{Der Controller benötigt eine Eingabe von der CPU}
    \catchword{Der Controller führt eine Verarbeitung durch}
    \catchword{Der Controller liefert eine Ausgabe}
  }

  \question{Wie wird die Datenübertragung eingeleitet}{\page{146}}
  {
    \normaltext{Wie bei einem Interrupt durch eine Unterbrechungsanforderung, allerdings
                auf einer besonderen Steuerleitung. Damit wird angegeben, dass die CPU
                für die Übertragung nicht benötigt wird. Diese Anforderung sollte eine
                sehr hohe Priorität haben.}
  }

  \statement{Nennen Sie die Möglichkeiten der DMA-Übertragung}{\page{147}}
  {
    \catchword{Cycle-stealing \resultol{Der Controller erhält den Bus bei Bedarf nach jedem abgeschlossenen
               Buszyklus der CPU}}
    \catchword{Prozessorhalt \resultol{Wenn der Controller den Bus zugeteilt erhält, belegt er ihn, bis der
               gesamte DMA-Block übertragen ist}}
    \catchword{Burst-mode \resultol{Der Controller überträgt jeweils eine festgelegte Anzahl von Daten
               hintereinander über den Bus}}
    \catchword{Denkbar ist auch eine Betriebsart, bei der ein Burst so lange dauert, bis das FiFo abgearbeitet
               ist, oder die CPU den Bus zurückverlangt.}
  }

  \question{Was haben alle Verfahren gemeinsam}{\page{147}}
  {
    \normaltext{Die DMA-Anforderungen (NPR) haben praktisch immer die höchste Priorität.}
  }

  \question{Wieviele NPR- und NPG- Leitungen gibt es auf einem Bus}{\page{148}}
  {
    \normaltext{Jeweils nur eine. Falls mehrere DMA-Controller eingesetzt werden sollen, verwendet 
                man zu ihrer Prioritätensteuerung eine Verriegelungstechnik mit einer Daisy Chain.}
  }

  \question{Welchen Vorteil bietet DMA im Realzeitbetrieb}{\page{149}}
  {
    \normaltext{Prozessdaten können in großer Menge und mit hoher Geschwindigkeit eingelesen werden}
  }

  \question{Welchen Nachteil bietet DMA im Realzeitbetrieb}{\page{149}}
  {
    \normaltext{Laufende Programme, auch Interrupt Service Routinen, können durch
                DMA verlangsamt werden}
  }

  \question{Was kann mit der Laufzeit und der Antwortzeit einer ISR durch DMA passieren}{\page{150}}
  {
    \normaltext{Können sich erhöhen, wenn der Bus durch DMA belegt ist}
  }

  \question{Wann kann die Belastung durch DMA gegen 0 gehen}{\page{150}}
  {
    \normaltext{Wenn die DMA-Buszyklen immer zu Zeiten auftreten, in denen die CPU den Bus
                sowieso nicht benötigt.}
  }

  \question{Wie kann die Datenübertragung verbessert werden}{\page{150}}
  {
    \normaltext{Durch Zwischenspeicherung der Daten im Controller}
  }

  \newpage
  \section{Periphere Geräte}

  \subsection{Standardperipherie}

  \subsubsection{Das Terminal}

  \subsubsection{Graphikeingabe}

  \subsubsection{Drucker}

  \subsubsection{Plotter}

  \subsubsection{Lochstreifen und Lochkarten}

  \subsubsection{Spracheingabe und Sprachausgabe}

  \subsubsection{Magnetbandgeräte}

  \subsubsection{Magnetplatten}

  \subsection{Prozessperipherie}

  \statement{Beschreiben Sie kurz die generelle Aufgabe der Prozessperipherie}{\page{188}}
  {
    \catchword{Prozessgrößen in Rechnerdaten umwandeln und umgekehrt}
  }

  \question{In welcher Form können die Daten des Prozess vorliegen}{\page{188}}
  {
    \catchword{elektrischer und nicht elektrischer Form (Strom, Spannung, Temperatur)}
    \catchword{digital oder analog, sie ändern sich diskret oder kontinuierlich (Stückzahlen, Messströme)}
    \catchword{dynamische Binärwerte (Unterbrechung einer Lichtschranke)}
  }

  \question{Worin lassen sich Messwerte unterteilen}{\page{188}}
  {
    \catchword{primäre Messwerte}
    \catchword{sekundäre Messwerte}
  }

  \statement{Beschreiben Sie anhand eines Schaubildes den Weg der Informationen
             vom Prozess in den Rechner und umgekehrt}{\page{189}}
  {
    \normaltext{Sehr wichtig! Siehe Bild 3.14 Seite 189}
  }

  \subsubsection{Messwerterfassung}

  \statement{Beschreiben Sie grundlegend die Aufgaben einer Messkette}{\page{190}}
  {
    \normaltext{An der Messstelle wird die Messgröße erfasst, die eventuell dann 
               transformiert werden muss. Das Messgerät setzt den Messwert in ein Signal
               um, das leicht weiterverarbeitet werden kann. Das Messgerät besteht
               aus zwei Komponenten:}
    \catchword{Messfühler: er setzt die Messgröße durch Ausnutzen naturwissenschaftlicher
               Gesetzmäßigkeiten in leichter erfassbare Größen um. Z.B.: Druck/Temperatur in einen Weg} 
    \catchword{Messwandler: einen Weiter Umwandlung mit dem Ergebnis eines meist elektrischen
               Signals, das gut übertragen werden kann.}
    \normaltext{Zuletzt wird das analoge elektrische Signal mittels A/D-Wandler in ein digitales
                Signal umgewandelt und an den Rechner übertragen.}
  }

  \statement{Nennen Sie einen Grund für einen frühe A/D-Wandlung
             (vor der Übertragung zum Rechner)}{\page{191}}
  {
    \catchword{digitale Daten können durch sichere Codes vor Übertragungsfehlern geschützt werden,
               das Ergebnis wird nicht weiter verfälscht.} 
  }

  \statement{Nennen Sie Faktoren, die in das Modell des gesamten Messprozesses eingehen müssen,
             insbesondere Fehlermöglichkeiten)}{\page{191}} 
  {
    \catchword{Übertragungseingenschafften aller Glieder in der Kette}
    \catchword{Übersetzungsverhältnisse}
    \catchword{Übertragungsfunktionen und -kennlinien}
    \catchword{Signallaufzeiten}
    \catchword{Kodierungen}
    \normaltext{Jede Umwandlung, und jeder Transport vor der A/D-Wandlung verfälscht das Ergebnis.}
  }

  \question{Was ist der Vorteil von digitalen Signalen im Gegensatz zu analogen}{\page{191}}
  {
    \catchword{Geeignete Codierung mit Prüfbits}
    \catchword{Geeignete Codierung mit automatischer Fehlerkorrektur}
  }

  \question{Was versteht man unter sekundären Messwerten}{\page{192}}
  {
    \catchword{gehen nach einer oder mehreren Umwandlungen aus den primären Messwerten hervor}
    \catchword{können eher fehlerbehaftet sein}
    \catchword{lassen nicht eindeutig auf den Prozesszustand zurückschließen}
  }

  \question{Worin lassen sich Messgrößen nach technischer Natur unterscheiden}{\page{192}}
  {
    \catchword{elektrische}
    \catchword{nicht elektrische}
  }

  \question{Worin lassen sich Messgrößen nach zeitlichem Verhalten unterscheiden}{\page{192}}
  {
    \catchword{statische Signale \resultol{deren Werte zeitlich konstant sind}}
    \catchword{stationäre Signale \resultol{deren Kenngrößen (Signalparameter) konstant sind}}
    \catchword{quasistationäre Signale \resultol{deren Werte sich nur langsam ändern}}
    \catchword{dynamische oder transiente Signale \resultol{deren Werte sich schnell ändern}}
    \catchword{Ereignisse \resultol{sprunghafte Änderung eines Signals, wobei nicht die Signalwerte 
               selber von Bedeutung sind, sondern nur die Tatsache der Änderung, ihre Richtung
               und ihr Zeitpunkt}}
  }

  \question{Worin lassen sich Messgrößen nach der Signalform unterscheiden}{\page{192}}
  {
    \catchword{analoge Signale \resultol{deren Werte sich kontinuierlich ändern}}
    \catchword{diskontinuierliche Signale \resultol{deren Werte sich sprunghaft zu bestimmten
               Zeitpunkten ändern}}
    \catchword{digitale Signale \resultol{können nur ganz diskrete Werte annehmen}}
    \catchword{quantisierte Signale \resultol{deren diskrete Werte äquidistant sind}}
    \catchword{binäre Signale \resultol{Extremfall von digitalen Signalen mit nur 2 versch. Werten}}
    \catchword{Impulsfolgen aus kurzen Binärsignalen \resultol{deren Abstände oder Anzahl pro 
               Zeiteinheit die Information darstellen}}
    \catchword{Abtastimpulse \resultol{die zeitlich äquidistant sind oder beliebig verteilt sind,
               und weitere Informationen in ihren Impulshöhen bieten}}
  }

  \question{Was versteht man unter primären Messwerten}{\page{192}}
  {
    \catchword{Sie werden direkt am Prozess genommen und sind noch unverfälscht}
    \catchword{Werte, die einen Prozesszustand direkt beschreiben und ihm eindeutig zugeordnet
               werden können}
  }

  \question{Was versteht man unter Messgliedern}{\page{193}}
  {
    \normaltext{Unter Messgliedern versteht man im einfachsten Fall die Sensoren oder Messfühler,
                die eine Messgröße in eine andere umwandeln. Meist sind sie in ein Messgerät
                eingebaut, das außer dem Sensor noch einen Messwandler besitzt.}
  }

  \question{Was versteht man unter einer Messeinrichtung}{\page{193}}
  {
    \normaltext{Als Messeinrichtung bezeichnet man Messgeräte, Messglieder und -fühler,
                aber auch größere Anlagen, in denen mehrere Messeräte vereinigt sind
                und eventuell auch schon eine Messwertverarbeitung stattfindet.}
  }

  \statement{Nennen und erklären Sie die vier Kategorien der Messglieder}{\page{194}}
  {
    \catchword{Messwertumsetzer oder Wandler: Umwandlung von nichtelektrischen Größen in andere,
               leichter erfassbare, nichtelektrische Größen}
    \catchword{Messfühler oder Sensoren: Umwandlung von nichtelektrischen in elektrische Größen}
    \catchword{Analog-Digital-Umsetzer: Umwandlung von analogen Signalen in digitale Werte}
    \catchword{Trigger oder Schwellwertgeber: Umsetzung von analogen elektrischen oder auch
               nichtelektrischen Werten in einen Binärwert}
  }

  \question{Worin unterscheiden sich passive und aktive Messglieder}{\page{195}}
  {
    \normaltext{Sie unterscheiden sich danach, ob sie für ihren Betrieb eine Hilfsenergie
                benötigen oder nicht.}
  }

  \question{Was muss bei der Auswahl der Messglieder beachtet werden}{\page{195}}
  {
    \catchword{Art der Messgröße}
    \catchword{Messgenauigkeit}
    \catchword{Messgeschwindigkeit}
    \catchword{Störfestigkeit}
  }

  \question{Welche Messglieder gibt es und wie funktionieren sie}{\pages{195}{199}}
  {
    \catchword{Schleifdrahtpotentiometer \resultol{erlaubt die direkte Umwandlung eines Weges x in
               eine statische analoge elektrische Spannung.}}
    \catchword{Codierstab \resultol{bewirkt die Diskretisierung einer Strecke durch Einteilung in Abschnitte 
               gleicher Länge. Dann muss die Position relativ zum Codierstab erfasst und erkannt werden.}}
    \catchword{Strichgitter und Lochscheibe \resultol{Die aktuelle Position kann nur aus der Bewegung und der 
               Vorgeschichte durch Mitzählen bestimmt werden.}}
    \catchword{Radarmessung \resultol{Laufzeitmessung mit bekannter Laufgeschwindigkeit.}}
    \catchword{Kapazitive und induktive Messverfahren \resultol{Kondensator wird deformiert und dadurch seine
               Kapazität geändert, oder eine Spuleninduktivität durch Einschieben eines Ferritkernes.}}
    \catchword{Dehnungsmessstreifen \resultol{Widerstandswert änder sich bei Dehnung.}}
    \catchword{Optische Interferometrie \resultol{Entfernungen werden über Laufzeitunterschiede
               zu Referenzstrecken durch optische Überlagerung gemessen.}}
    \catchword{Piezoelektrische Sensoren \resultol{wandeln geringste Deformationen in Spannungen um.}}
  }

  \question{Was muss mit analogen elektrischen Eingangssignalen häufig geschehen}{\page{200}}
  {
    \normaltext{Eine Umsetzung in digitale, binär codierte Werte durch Analog-Digital-Umsetzer
                oder Wandler}
  }

  \question{Wenn der Analog-Digital-Umsetzer als Prozess betrachtet wird, hat er welche Ein-
            und Ausgänge}{\page{200}}
  {
    \catchword{Eingang: Zeitlich veränderliche Spannung $U(t)$}
    \catchword{Eingang: Feste Referenzspannung $U_R$}
    \catchword{Ausgang: Digitalwert $D(t)$}
  }

  \question{Was bestimmt das Auflösungsvermögen}{\page{200}}
  {
    \normaltext{Die Anzahl der ausgegebenen Digitalstellen, welche aber nicht gleichbedeutend
                mit der Genauigkeit ist}
  }

  \question{Welche Codierung können die Ausgabewerte haben}{\page{200}}
  {
    \catchword{Dualzahl}
    \catchword{Binäre Offset}
    \catchword{2-er Komplement}
    \catchword{1-er Komplement}
  }

  \question{Was muss bei sich schnell veränderlichen Eingangswerten getan werden}{\page{201}}
  {
    \normaltext{Es muss ein Speicher für den Momentanwert der Eingangsspannung eingesetzt werden,
                der als Abtast- und Halteglied bezeichnet wird. Dabei wird die Eingangsspannung
                zum Beispiel in einem Kondensator gespeichert. Die Öffnungszeit des Gates wird
                als Aufnahmezeit bezeichnet.}
  }

  \question{Was wird als Haltezeit bezeichnet}{\page{201}}
  {
    \normaltext{Die Zeit, während der das Signal für den ADU einen konstanten Wert haben muss,
                wird als Abtastzeit oder Aperturzeit bezeichnet}
  }

  \question{Was besagt das Shannon Theorem}{\pages{201}{202}}
  {
    \normaltext{Eine Abtastrate $f_s$ erlaubt eine informationsgetreue Umwandlung nur solcher Signale, die
                nur Frequenzanteile bis zu einer Obergrenze $f_m \textless f_s/2$ enthalten.}
  }

  \question{Wodurch wird die Frequenz begrenzt}{\page{202}}
  {
    \normaltext{Da während einer endlichen Zeit $ T = N/f_s $  nur N Abtastungen durchgeführt
                werden können, wird dabei die Frequenzauflösung auf einen endlichen Bereich
                $ \Delta f = 2f_m/N = f_s/N = 1/T $ begrenzt. Das führt zu einer oberen
                Frequenzgrenze $ f_{max} = f_m - \Delta f $.  }
  }

  \question{Was tritt bei Überschreitung der Zeitauflösung auf}{\page{202}}
  {
    \normaltext{Es tritt der sogenannte Alias-Effekt auf, der das Vorhandensein weit niedriger
                Frequenzen vortäuscht.}
  }

  \question{Was kann man gegen diesen Effekt tun}{\page{202}}
  {
    \normaltext{Vorschalten von Filtern in Form von Tiefpässen}
  }

  \question{Welche Abtastfrequenzen werden bei welchem Betrieb erreicht}{\page{202}}
  {
    \catchword{Abrufbetrieb (Polling) \resultol{100 Hz}}
    \catchword{Anforderungsbetrieb (mit Interrupt) \resultol{1000Hz}}
    \catchword{Speicherdirektzugriff (DMA) \resultol 100 kHz (das ist die Grenze üblicher ADUs)}
    \normaltext{Für höhere Abtastfrequenzen setzt man am besten einen eigenen Datenpuffer z.B. als
                FIFO davor.}
  }

  \question{Wozu wird ein Messstellenumschalter (Multiplexer) verwendet}{\page{203}}
  {
    \normaltext{Der Ausgang wird im zeitlichen Wechsel an einen der Ausgänge durchgeschaltet.}
  }

  \question{Welche Möglichkeiten gibt es einen Multiplexer zu verwenden}{\page{203}}
  {
    \catchword{wahlfrei (RAM) über ein Adresswort}
    \catchword{sequentiell (SAM)}
  }

  \question{Welche Vor- und Nachteile hat SAM}{\page{203}}
  {
    \catchword{Vorteil: oft schneller, da zyklisch weitergeschaltet wird}
    \catchword{Nachteil: Die Abtastrate geht fr jeden der Eingangskanäle in dem Maße zurück, wie die Zahl der 
               Kanäle zunimmt.}
  }

  \question{Welchen Nachteil haben Multiplexer}{\page{203}}
  {
    \normaltext{Da im Multiplexverfahren die Messsignale nacheinander umgewandelt werden, können
                unter Umständen zeitliche Beziehungen verloren gehen; z.B. wenn man die Wert von
                verschiedenen Messstellen für exakt den selben Zeitpunkt wissen möchte.}
  }

  \statement{Nennen und erklären Sie die sechs bei einem ADU auftretenden
             Umsetzungsfehler}{\pages{204}{205}}
  {
    \catchword{Quantisierungsfehler: Bedingt durch die endliche Auflösung des ADU, also durch
               die endliche Anzahl von Ausgangsbits}
    \catchword{Verstärkunsfehler: Bezieht sich auf den nominellen Umsetzungsfaktor vom analogen
               Eingangssignal $U$ zum digitalen Ausgabewert $D$. Kann durch Eichung immer wieder
               neu bestimmt werden}
    \catchword{Nullpunktsverschiebung: Systematische Verschiebung aller Ausgangswerte um einen
               festen Betrag}
    \catchword{Linearitätsfehler: Beschreibt die Abweichung des lokalen vom mittleren Umsetzungfaktor
               und zeigt sich in einer gekrümmten Kennlinie}
    \catchword{Monotoniefehler: Lokale extreme Nichtlinearitäten der Kennlinie, wo die Monotonie
               gestört ist}
    \catchword{Gesamtfehler (oder umgekehrt die Genauigkeit): Summe aller Einzelfehler}
  }

  \question{Worin unterscheiden sich die verschiedenen Analog-Digital-Umsetzer}{\page{205}}
  {
    \catchword{Realisierungsaufwand, also den entstehenden Kosten}
    \catchword{Auflösung}
    \catchword{Genauigkeit}
    \catchword{Geschwindigkeit}
  }

  \question{Wie kann eine Polaritätserkennung realisiert werden}{\page{206}}
  {
    \catchword{Erkennung der Polarität mit einem Komparator}
    \catchword{Dieser Komparator liefert das MSB der Daten}
    \catchword{Der Komparator dient auch zur Auswahl des Signals bei positiver Spannung
               oder des invertierten Signals bei negativer Spannung}
    \catchword{Die negative Spannung wird über einen Inverter zur Verfügung gestellt}
    \catchword{Durch dieses Vorgehen bekommt der ADU immer positive Werte und kann die
               restlichen Bits ohne das MSB bestimmen}
  }

  \question{Welche verschiedenen Analog-Digital-Umsetzer gibt es}{\pages{206}{212}}
  {
    \catchword{Simultanwandler oder Parallelwandler}
    \catchword{Sägezahnumsetzer}
    \catchword{Zählumsetzer}
    \catchword{Folgeumsetzer}
    \catchword{Integrationswandler}
    \catchword{Spannungs-Frequenz-Umsetzer}
    \catchword{Stufenumsetzer oder Wägecodierer}
  }

  \question{Wie groß sollten Abweichungen maximal sein}{\page{206}}
  {
    \normaltext{Kleiner als 1/2 LSB, damit das Ergebniss nicht verfälscht wird}
  }

  \statement{Erklären Sie detailiert den Simultanwandler (Parallelwandler)}{\page{206}}
  {
    \normaltext{Das anliegende Eingangssignal wird gleichzeitig über eine Reihe von Komparatoren
                (Nullwertsvergleicher) mit einer Spannungsteilerkette verglichen, die von der
                Bezugsspannung $U_R$ gespeist wird. Dabei liefern alle Komparatoren eine 0,
                für die der Signalwert kleiner ist als der Vergleichswert, die übrigen eine 1.
                An einer bestimmten Position erfolgt dabei der Umschlag von 0 nach 1. Dieser
                \important{1-aus-m Code} wird durch ein geeignetes Codiernetz in eine Dualzahl
                umgesetzt. Dieses Verfahren ist sehr aufwendig, da man für $n$ Bits insgesamt
                $2^n - 1$ Komparatoren benötigt. Der Umsetzer ist sehr schnell, da der
                Momentanwert der Eingangsspannung unmittelbar angezeigt wird. Die Eingangsspannung
                muss nur so lange konstant bleiben, dass alle Komparatoren durchschalten können,
                was als Aperturzeit bezeichnet wird. Die Genauigkeit der Wandlung wird fast
                ausschließlich von der Güte der Komparatoren bestimmt}
  }

  \statement{Skizzieren Sie das Blockschaltbild des Simultanwandlers (Parallelwandlers)} {\page{206}}
  {
    \normaltext{Siehe Bild 3.27, Seite 206}
  }

  \statement{Erklären Sie detailiert den Sägezahnumsetzer}{\page{207}}
  {
    \normaltext{Die Signalspannung wird durch einen einzigen Komperator mit der zeitlinear 
                ansteigenden Spannung eines Sägezahngenerators verglichen. Beim Start des Sägezahns wird
                ein Zähler zurückgesetzt, und zählt dann die Taktimpulse bis der Komperator feststelle,
                dass die ansteigende Spannung gleich der Messspannung ist. Damit ist die Anzahl der 
                gezählten Impulse ein Maß für die Eingangsspannung. (vgl. Bild 3.28)
                Die Wandlungsrate ist hier durch die Frequenz des Sägezahns festgelegt. Die Auflösung kann
                fast beliebig groß gemacht werden, denn sie wird durch die Taktfrequenz für den Zähler und
                die Periodendauer des Sägezahns bestimmt. Auflösungen bis zu 20 bit sind hier möglich. Die
                Genauigkeit hängt aber im wesentlichen von der Linearität des Sägezahns ab, im geringeren
                Umfang auch von der Frequenzstabilität des Taktgenerators. Ein besonderer Nachteil ist die
                große Totzeit, die entsteht, wenn der Sägezahn nach Beendigung der Zählung bis zum Ende
                durchlaufen muss. Auch kann er meißt nicht von außen gestartet werden, sonder bestimmt selber
                Beginn und Ende der Umsetzung, kann das aber durch ein BUSY-Signal mitteilen.}
  }

  \statement{Skizzieren Sie das Blockschaltbild des Sägezahnumsetzer}{\page{207}}
  {
    \normaltext{Siehe Bild 3.28, Seite 207}
  }

  \statement{Erklären Sie detailiert den Zählumsetzer}{\page{208}}
  {
    \normaltext{Der Zählumsetzer vergleicht die Eingangsspannung mit einer Sägezahnspannung
                aus einem Digital-Analog-Wandler (DAU), der den Digitalwert aus einem Zähler
                in eine Analogspannung umsetzt. Dieser Zähler kann von außen durch einen
                Startimpuls zurückgesetzt werden und gestartet werden und zählt die Impulse
                eines Taktgenerators solange, bis der Komparator die zählung stoppt. Danach
                kann der Wert des Zählers ausgewertet werden. Das Auflösungsvermögen wird hier
                durch die Taktfrequenz und die Anstiegszeiten des Sägezahns bestimmt, welche
                wiederum durch den Umsetzungsfaktor des DAU bestimmt werden. Die Wandlungsrate
                ist dann nicht mehr festgelegt, sondern vom Eingangssignal abhängig. Die
                Genauigkeit ist im wesentlichen durch den DAU bestimmt.}
  }

  \statement{Skizzieren Sie das Blockschaltbild des Zählumsetzers} {\page{208}}
  {
    \normaltext{Siehe Bild 3.29, Seite 208}
  }

  \statement{Erklären Sie den Folgeumsetzer}{\page{208}}
  {
    \normaltext{Der Folgeumsetzer ist eine Variante des Zählumsetzers, bei der der Zähler als
                Vorwärts-/Rückwärtszähler ausgelegt ist und durch den Komperator richtungsgesteuert wird.
                Wenn der Signalwert sich nicht zu schnell ändert, liegt hier immer ein Momentanwert wor.
                Falls die Abweichung von Signal- und Vergleichsspannung betragsmäßig einen bestimmten Wert
                überschreitet, liefert der Umsetzer ein BUSY-Signal, umd dies zu melden.}
  }

  \statement{Skizzieren Sie das Blockschaltbild des Folgeumsetzers} {\page{208}}
  {
    \normaltext{Siehe Bild 3.30, Seite 208}
  }

  \question{Was wird als Wandlungsrate bezeichnet}{\page{208}}
  {
    \normaltext{Die Frequenz mit der ein Umsetzer umwandeln dann, also der Kehrwert der
                Umsetzungszeit, die durch die Laufzeiten durch die Bauteile bestimmt ist}
  }

  \statement{Erklären Sie detailiert den Integrationswandler nach dem Zwei-Rampen-Verfahren}{\pages{208}{210}}
  {
    \normaltext{Beim Integrationswandler nach dem Zwei-Rampe-Verfahren (engl.: dual slope) wird
                das Messsignal über einen festen Zeitraum $t_1 = N*T$ integriert, welcher durch
                einen zentralen Takt bestimmt wird. Dabei wird ein Kondensator C auf eine Spannung
                $U_C$ aufgeladen, die exakt dem Mittelwert der Messspannung $U_x$ während der Zeit $t_1$
                entspricht. Anschließend wird der Kondensator durch die Bezugsspannungsquelle $U_R$
                entladen, und die dazu benötigte Zeit $t_2 = x*T$ durch mitzählen der Taktimpulse gemessen.
                Man erhält daraus den Messwert $U_x = U_R * x/N$ wobei sich alle übrigen Gerätegrößen
                (R, C, T) herausheben, wenn sie während der Messzeit konstant sind. Dadurch wird dieses
                Verfahren sehr genau. Auch die Auflösung ist sehr gut, wenn man mit einer genügend hohen
                Taktfrequenz arbeitet, typischerweise 1MHz, die durch den DAU begrenzt wird. Die Besonderheit
                bei diesem Verfahren liegt darin, dass durch die Integration ein Mittelwert über einen größeren
                Zeitraum gebildet wird, und deabei Schankungen und Störungen des Messsignals herausgemittelt
                werden. Da kleinen Signalen oft eine Störung durch das Versorgungsnetz mit 50 Hz überlagert ist,
                kann man mit Messzeiten, die ein Vielfaches der Periodendauer von 20 msec sind, diese Störungen
                weitgehend unterdrücken. Für Momentanwertmessungen muss hier immer ein Sample-an-Hold-Glied
                eingesetzt werden.}
  }

  \statement{Skizzieren Sie das Blockschaltbild des Integrationswandlers}{\page{209}}
  {
    \normaltext{Siehe Bild 3.31, Seite 209}
  }

  \statement{Erklären Sie detailiert den Spannungs-Frequenz-Umsetzer}{\page{210}}
  {
    \normaltext{Die anliegende Spannung wird in eine Frequenz umgesetzt. Das Vefahren ist
                nicht besonders linear. Während einer vorgegebenen Zeitspanne $t$ wird
                ein Kondensator $C$ durch eine bezugsspannungsquelle aufgeladen, der sich
                dann über einen Widerstand $R$ solange entlädt, bis er den Wer des Messsignals
                ereicht hat. Der Komparator startet dann den Aufladevorgang von neuem, so dass
                am Ausgang eine Folge von Impulsen konstanter Länge mit veränderlichen Abständen
                entsteht. Dieser Abstand ist in etwa umgekehrt proportional zur anliegenden
                Messspannung $U$ und damit die Impulsfolgefrequenz $f$ proportional zu $U$.
                Die Impulsfrequenz kann dann mit einem Frequenzzähler über kürzere oder
                längere Intervalle gemessen und dabei gemittelt werden. Damit wird auch die
                Auflösung festgelegt, die in weiten Grenzen variiert werden kann. Die Genauigkeit
                wird hier durch den Linearitätsfehler der Anordnung beschränkt.}
  }

  \statement{Skizzieren Sie das Blockschaltbild des Spannungs-Frequenz-Umsetzer} {\page{210}}
  {
    \normaltext{Siehe Bild 3.32 Seite 210}
  }

  \statement{Erklären Sie detailiert den Stufenumsetzer (Wägecodierer)}{\pages{211}{212}}
  {
    \normaltext{Der Stufenumsetzer arbeitet nach dem Prinzip der sukzessiven Aproximation. Ein DAU
                wird für die Erzeugung einer Vergleichsspannung eingesetzt, aber nicht einfach linear
                durchfahren. Stattdessen wird bitweise sukkzessiv abgeglichen. Dabei wird zuerst das
                MSB bestimmt, indem es zunächst versuchsweise gesetzt wird. Die vom DAU erzeugte
                Spannung wird dann mit einem Komparator verglichen. Wenn die vom DAU erzeugte Spannung
                größer als die Eingangsspannung ist, wird das Bit wieder gelöscht, ansonsten bleibt
                es erhalten. Da hier der Signalwert innerhalb weniger Schritte erreicht wird, nämlich
                gerade so vieler, wie die Auflösung beträgt. Während der Zeit des Umwandels muss das
                Eingangssignal konstant bleiben.}
  }

  \statement{Skizzieren Sie das Blockschaltbild des Stufenumsetzers (Wägecodierers)} {\pages{211}{212}}
  {
    \normaltext{Siehe Bild 3.33, Seite 211}
  }

  \subsubsection{Prozesssteuerung}

  \question{Was muss bei der Prozesssteuerung durch den Prozessrechner getan werden}{\page{213}}
  {
    \normaltext{Durch den Prozessrechner müssen Steuergrößen erzeugt werden, mit denen der Zustand
                eines Prozesses beeinflusst werden kann}
  }

  \question{Was versteht man unter dem offenen Prozessbetrieb}{\page{213}}
  {
    \normaltext{Die notwendigen Steuerdaten müssen aus dem Prozessmodell bestimmt werden}
  }

  \question{Was versteht man unter dem geschlossenen Prozessbetrieb}{\page{213}}
  {
    \normaltext{Zur Bestimmung der Steuerdaten werden auch die aktuellen Messwerte einbezogen,
                und man spricht von einem Regelkreis oder einer Regelung}
  }

  \question{Die Einwirkung der Stellgröße erfolgt an welchem Ort}{\page{214}}
  {
    \normaltext{Stellort}
  }

  \question{Wie groß muss der Wirkungsbereich der Stellgröße mindestens sein}{\page{214}}
  {
    \normaltext{So groß, dass er eine Zustandsgröße des Prozesses erreicht und
                beeinflussen kann}
  }

  \question{Wie unterscheidet man Stellgrößen nach ihrem Stellort}{\page{214}}
  {
    \catchword{Zufluss von Verarbeitungsgut}
    \catchword{Abfluss von Verarbeitungsgut}
  }

  \question{Wie unterscheidet man Stellgrößen nach der Prozessnähe}{\page{214}}
  {
    \catchword{Primäre Stellgrößen: Wirken direkt auf die Prozessgrößen}
    \catchword{Sekundäre Stellgrößen: Wirken auf den Stellantrieb, der seinerseits
               ein primäres Stellglied betätigt}
  }

  \question{Wie unterscheidet man Stellgrößen nach der technischen Natur}{\page{214}}
  {
    \catchword{Elektrische Stellgrößen: Meist nur als sekundäre Größen}
    \catchword{Sekundäre Stellgrößen: Eher die Regel}
  }

  \question{Nach welchen fünf Gesichstpunkten kann man Stellgrößen einteilen}{\pages{214}{215}}
  {
    \catchword{Stellort}
    \catchword{Prozessnähe}
    \catchword{Technische Natur}
    \catchword{Stelleigenschaft und Signalform}
    \catchword{Zeitlichen Verhalten}
  }

  \question{Wie unterscheidet man Stellgrößen nach den Stelleigenschaften
            und den Signalformen}{\pages{214}{215}}
  {
    \catchword{Stetige Stellgrößen: Analoge Werte, die sich kontinuierlich ändern können}
    \catchword{Diskrete Stellgrößen: Digitale Werte}
    \catchword{Binäre Werte: Spezialfälle für diskrete Größen}
    \catchword{Ternäre Werte: 3-wertige Größen}
  }

  \question{Wie unterscheidet man Stellgrößen nach dem zeitlichen Verhalten}{\page{215}}
  {
    \catchword{Statische: Größen, die zeitlich konstant sind}
    \catchword{Stationäre: Größen mit konstanten Kenngrößen wie z.B. Frequenz}
    \catchword{Quasistationäre: Größen, die sich langsam ändern}
    \catchword{Synamische: Größen, die sich schnell ändern}
  }

  \statement{Erklären Sie die Kette bei den Stellgliedern}{\page{215}}
  {
    \normaltext{Die Kette reicht vom Prozessrechner mit seinen digitalen Ausgabewerten
                bis zum Prozess reicht, wo die primären Stellgrößen, meist mit einem hohen
                Energiebedarf verbunden sind.}
  }

  \question{Sind praktisch alle Stellglieder aktiv oder passiv und was heißt das}{\page{215}}
  {
    \normaltext{Fast alle sind passiv, dass heißt sie brauchen eine Hilfsenergie}
  }

  \question{Nach ihrer Komplexität unterscheidet man drei Arten von Stellgliedern. Welche}{\page{215}}
  {
    \catchword{Stellglieder}
    \catchword{Stellgeräte: Bestehen aus einem Stellglied mit Stellantrieb}
    \catchword{Stelleinrichtungen: Bestehen aus mehreren Stellgeräten}
  }

  \question{Nach ihrem Verhalten charakterisiert man unterschiedliche Stellgliedern}{\pages{216}{217}}
  {
    \catchword{Statisch: Durch eine Kennlinie gekennzeichnet}
    \catchword{Stetig: Monotone und stegie Kennlinie}
    \catchword{Schaltend: Ausgangsgröße kann nur diskrete Werte annehmen}
    \catchword{Dynamisch: Gekennzeichnet durch eine (meist komplexe) Übertragungsfunktion}
  }

  \question{Welche zwei wichtigen Änderungen einer Eingangsgöße gibt es}{\page{217}}
  {
    \catchword{Impulsantwort}
    \catchword{Sprungantwort}
  }

  \question{Welche Impulsantworten unterscheidet man}{\page{217}}
  {
    \catchword{Geräte mit Speicherverhalten: Verharren in der angenommenen Stellung
               auch wenn keine Steuereinwrikung mehr vorliegt}
    \catchword{Geräte mit Haftverhalten: Zeigen ein Speicherverhalten, das von einer
               Hilfsenergie unabhängig ist}
    \catchword{Geräte ohne Speicherverhalten: Kehren bei Ausfall der Steuerung oder der
               Hilfsenergie in eine Grundstellung oder Ruhelage zurück}
  }

  \question{Welche Sprungantworten unterscheidet man}{\page{218}}
  {
    \catchword{Proportionalverhalten: Die Ausgangsgröße ist proportional zur Eingangsgröße}
    \catchword{Differenzierverhalten: Die Ausgangsgröße reagiert nur kurz auf die Änderung
               der Eingangsgröße}
    \catchword{Integrierverhalten: Die Ausgangsgröße wächst an, solange das Eingangssignal
               anliegt}
    \catchword{Kriechverhalten: Das Ausgangssignal nimmt nur langsam seinen Endwert an}
    \catchword{Überschwingen: Das Ausgangssignal schießt über seinen Endwert hinaus und
               pendelt sich dann auf diesen ein}
    \catchword{Verzögerungsverhalten: Das Ausgangssignal reagiert erst nach einer Verzögerung}
  }

  \question{Was ist die generelle Funktionsweise eines D/A-Wandlers (ein Satz)}{\page{219}}
  {
    \normaltext{Aus einem Digitalwert und einer Referenzspannung wird ein analoge Ausgangsspannung gebildet}
  }

  \statement{Nennen Sie die charakteristischen Leistungsmerkmale der DAUs)}{\page{219}}
  {
    \catchword{Auflösung (8-12 Bit)}
    \catchword{Genauigkeit}
    \catchword{Geschwindigkeit}
  }

  \statement{Nennen Sie Fehlerquellen der DAUs)}{\page{219}}
  {
    \catchword{Nullpunktverschiebung}
    \catchword{Monotoniefehler}
    \catchword{Verstärkungsfehler}
    \catchword{Linearitätsfehler}
  }

  \statement{Nennen Sie drei verschiedene geläufige Verfahren der DAU}{\pages{219}{}}
  {
    \catchword{Stromsummenwandler}
    \catchword{R-2R-Wiederstandsnetzwerke}
    \catchword{Linear-Umsetzer}
  }

  \statement{Beschreiben Sie die prinzipielle Funktionsweise des Stromsummenwandlers}{\page{219}}
  {
    \catchword{Referenzspannung wird mittels verschiedener Widerstände in sich um den Faktor 2
               unterscheidende Ströme aufgeteilt}
    \catchword{Die zu den Digitalsbits gehörenden Ströme werden aufsummiert und ergeben den Ausgangsstrom}
  }

  \statement{Nennen Sie die Leistungsdaten des Stromsummenwandlers}{\page{219}}
  {
    \catchword{12 Bit Auflösung}
    \catchword{Genauigkeit: 0,025\%}
    \catchword{Wandlungszeit: 2$\mu$sec}
  }

  \statement{Nennen Sie den Vorzteil des Stromsummenwandlers und beschreiben Sie
             genau sein Problem}{\page{220}}
  {
    \catchword{geringer Aufwand: n Wiederstände (bei n Bits)}
    \catchword{Monotonie: In der Mitte des Wandlungsbrereich befinden sich die zwei Dualwerte 1000 000
               und 0111 1111, die sich nur um ein Bit unterscheiden. Im ersten Fall wird der Strom nur
               durch einen Wiederstand mit 1kOhm fließen (Strom ist dann 5mA), im zweiten Fall durch alle
               anderen parallel fließen: Die Summe wäre 254/265*5mV = 4,98mA. Es reicht aus, wenn der erste
               Widerstand um mehr als 4 Ohm von seinen 1kOhm abweicht und der Strom würde kleiner als 4,8 mA,
               und damit läge ein Monotoniefehler vor.}
  }

  \statement{Skizzieren Sie den prinzipiellen Aufbau des R-2R Widerstandnetzwerkes}{\page{221}}
  {
    \normaltext{Siehe Bild 3.39, Seite 221}
  }

  \statement{Skizzieren Sie den prinzipiellen Aufbau des Linear-Umsetzers}{\page{222}}
  {
    \normaltext{Siehe Bild 3.39, Seite 221}
  }

  \statement{Nennen Sie die entscheidenden Vor- und Nachteile des Linear-Umsetzers}{\page{222}}
  {
    \catchword{Nachteil: riesiger Aufwand: für n Bits braucht man $2^n$ Wiederstände und $(2^n)-1$ Umschalter}
    \catchword{Sehr genau, sowohl linear als auch monoton}
  }

  \subsubsection{Prozessanschluss}

  \question{Was wird durch die Art einer Übertragungsleitung bestimmt}{\page{229}}
  {
    \catchword{Übertragungseigenschaften einer Leitung}
    \catchword{Störempfindlichkeit einer Leitung}
  }

  \question{Durch welche einzelnen Elemente kann eine Übertragungsleitung ersatzweise
            beschrieben werden}{\page{229}}
  {
    \catchword{einen ohmschen Widerstand R, der eine frequenzunabhängige Signaldämpfung bewirkt und
               mit der Leitungslänge proportional anwächst.}
    \catchword{Induktivitäten L und Kapazitäten C, die zusammen mir R einenn Tiefpass bilden, der höhere
               Frequenzen abschneidet und die Bandbreite der Leitung bestimmt, welche proportional
               zur Länge abnimmt.}
    \catchword{Kopplungskapazitäten $C_k$ und Kopplungsinduktivitäten $L_k$, über welche die äußeren
               Störungen eingekoppelt werden. Zu diesen Störungen zählt auch Übersprechen von benachbarten
               Leitungen. Die Störempfindlichkeit wächst mit der Leitungslänge.}
  }

  \question{Was wird noch durch die Kapazitäten und Induktivitäten bestimmt}{\page{230}}
  {
    \catchword{Der komplexe Wellenwiderstand Z der Leitung, der als ohmscher Abschlusswiderstand
               am Kabelende benötigt wird, damit keine Reflexionen auftreten.}
    \catchword{Die Signalgeschwindigkeit c auf der Leitung, deren Kehrwert als Laufzeitkonstante
               bezeichnet wird.}
  }

  \question{Welche Aufbauarten von Leitungstypen kann man unterscheiden}{\pages{230}{231}}
  {
    \catchword{Einzeladern \resultol{kein definierter Wellenwiderstand, geringe Bandbreitenlänge
               (ca 1MHz*m), große Störanfälligkeit}}
    \catchword{Flachbandkabel \resultol{parallel verlaufende Einzeladern, relativ gut definierter
               Wellenwiderstand, relativ große Bandbreitenlänge (ca 100MHz*m)}}
    \catchword{Verdrillte Adernkabel \resultol{relativ guter Wellenwiderstand, relativ große 
               Bandbreitenlänge (über 1MHz*m), Störeinwirkung hebt sich in den beiden Leitern weitgehend
               gegenseitig auf (2 Ströme in entgegengesetzter Richtung)}}
    \catchword{Koaxialkabel \resultol{besteht aus Innenleiter und geschlossener, konzentrischer
               Aussenleiter, höchste Störfestigkeit, wohldefinierter Wellenwiderstand, Bandbreitenlänge
               hängt stark vom Isolationsmaterial ab (ca 100MHz*km)}}
    \catchword{Lichtwellenleiter \resultol{ummantelte Glasfaser, über die moduliertes Licht geleitet
               wird, praktisch absolut stör- und abhörsicher, hohe Bandbreitenlänge (ca 1GHz*km)}}
  }

  \question{Welche andere Art der Übertragung gibt es noch}{\page{231}}
  {
    \normaltext{Leitungsungebundene Übertragung durch Ultraschall, Infrarotlicht oder per Funk.}
  }

  \question{Welche Arten von Störungen können auftreten}{\pages{231}{232}}
  {
    \catchword{Wechselspannungen, die von außen induktiv oder kapazitiv eingekoppelt werden.}
    \catchword{Gleichspannungen, die über eine galvanische Kopplung entstehen. z.B. wenn Sender und
               Empfänger auf versch. Potentialen liegen.}
    \catchword{Gleichtaktstörungen, durch gleichphasige Einkopplungen auf den Leitern, oder durch
               Potentialdifferenz zwischen Sender und Empfänger.}
    \catchword{Gegentaktstörungen, die durch Gleichtaktstörungen entstehen, die nicht gleich groß
               sind, oder sich nicht kompensieren lassen.}
  }

  \question{Welche Störquellen gibt es}{\page{232}}
  {
    \catchword{Nachbarleitungen, auf denen Signale mit hochfrequenten Anteilen oder mit hoher
               Leistung transportiert werden.}
    \catchword{Erdschleifen, bei denen versch. Signale oder Versorungsströme über einen gemeinsamen
               Erdleiter laufen}
    \catchword{Prozesseinrichtungen, die mit hohen Listungen aus dem Netz betrieben werden. Hier
               werden die Störungen meistens induktiv eingekoppelt und führen zu Gegentaktstörungen mit
               Netzfrequenz.}
    \catchword{Hochfrequenz-Generatoren und -Anlagen, zur Wärmebehandlung. Hier wirkt meißt die
               kapazitive Ankopplung.}
    \catchword{Elektrische Funken, die beim Öffnen von Schaltern entstehen. Unter Umständen führen
               sie zu unberechnbaren Stoßwellen.}
  }

  \question{Welche Schutzeinrichtungen gibt es (mit Ersatzschaltbildern)}{\page{233}}
  {
    \catchword{Koppelkondensatoren (kapazitive Ankopplung)}
    \catchword{Koppelübertrager (induktive Ankopplung)}
    \catchword{Optokoppler}
    \catchword{Relais}
    \catchword{Begrenzer mit Dioden}
    \catchword{Ladungsübertrager (Flying Capacitors)}
    \catchword{Zerhackerschaltungen}
    \catchword{Tiefpässe}
    \catchword{Zenerdioden}
  }

  \question{Wie können Geräte im einfachsten Fall an die Zentraleinheit angeschlossen
            werden}{\page{236}}
  {
    \normaltext{Sternförmig, so dass die Geräte keine Verbindung untereinander haben, aber
                an die Zentraleinheit angeschlossen sind}
  }

  \question{Welche Komponenten eines Prozessrechners müssen an die Anforderungen
            des technischen Prozesses angepasst werden}{\page{240}}
  {
    \catchword{CPU}
    \catchword{Speichergröße}
    \catchword{Standardperipherie}
    \catchword{Prozessperipherie}
  }

  \subsection{Prozessrechnereinsatz}

  \subsubsection{Planung der Konfiguration}

  \question{Welcher Ablauf ist für die Planung der Konfiguration typisch}{\page{240}}
  {
    \catchword{Voruntersuchung: Kosten-Nutzen-Analyse}
    \catchword{Istaufnahme: Der zu lenkende Prozess wird untersucht}
    \result{Schema: Messaufgaben, Steuerungsaufgaben, Bedienungsaufgaben und Sicherheitsfragen}
    \catchword{Sollkonzept: Pfichtenheft wird erstellt}
    \catchword{Beschaffung: Durch eine gezielte Ausschreibung}
  }

  \subsubsection{Betriebssicherheit}

  \statement{Nennen Sie die zwei Problemgruppen bezüglich der Betriebssicherheit}{\page{241}}
  {
    \catchword{Sicherheit bei der Verarbeitung: Zuverlässigkeit und Störungsfreiheit im Betrieb}
    \catchword{Sicherheit für die Umgebung: Gefährdung von Gesundheit und Leben}
  }

  \statement{Geben Sie die verschiedenen möglichen Störungen an}{\page{241}}
  {
    \catchword{Netzausfall}
    \catchword{Verfälschung der Daten}
    \catchword{Temporäre Hardware-Fehler}
    \catchword{Permanente Hardware-Fehler}
    \catchword{Permanente Software-Fehler}
  }

  \question{Was versteht man unter der Failsafe-Technik}{\page{241}}
  {
    \normaltext{Eine Strategie zur Entschärfung von Gefahren, die erreichen soll,
                dass im Fehlerfall alle Systeme oder Einzelgruppen in einen sicheren
                Zustand übergehen. Das wird auch als \important{inhärente} Sicherheit
                bezeichnet. Was dabei ein sicherer Zustand ist, hängt vom Prozess ab.}
  }

  \question{Was versteht man unter dem Watchdog-Verfahren}{\page{242}}
  {
    \normaltext{In vorgegebenen zeitlichen Abständen müssen bestimmte Aktionen erfolgen.
                Bleiben diese aus, wird dies als Fehler registriert, der dann behandelt
                werden muss.}
  }

  \question{Wie ist die Zuverlässigkeit definiert}{\page{242}}
  {
    \normaltext{Die Zuverlässigkeit ist die Eigenschaft, dass sich ein System innerhalb
                vorgegebener Grenzen bewegt und seine Aufgaben somit erfüllt.}
  }

  \question{Was wird danach als Fehler bezeichnet}{\page{242}}
  {
    \normaltext{Eine unzulässige Abweichung vom Sollwert}
  }

  \question{Welche zwei Arten von Ausfällen gibt es}{\page{242}}
  {
    \catchword{Driftausfall: Wenn sich der Ausfall vorher durch Fehler bemerkbar macht}
    \catchword{Sprungausfall: Wenn der Ausfall unvorhersagbar eintritt}
  }

  \statement{Erklären sie die Kurze der Ausfallrate}{\pages{244}{245}}
  {
    \catchword{Am Anfang durch Frühausfälle relativ hoch}
    \catchword{Im mittleren Bereich nur statistische Ausfälle}
    \catchword{Am Ende setzen sich Verschleißausfälle durch}
  }

  \question{Wie verhält sich die Reparaturrate bezüglich der Ausfallrate}{\page{244}}
  {
    \normaltext{Sie zeigt ein entgegengesetztes Verhalten}
  }

  \question{Was kann die Lebensdauer eines Systems verringern}{\page{245}}
  {
    \catchword{Überlastung}
    \catchword{Unterlastung unter Umständen auch}
  }

  \question{Durch welche zwei Schaltungen kann die Gesamtverfügbarkeit beschrieben
            werden}{\page{245}}
  {
    \catchword{Serienschaltung: Das System fällt aus, wenn ein Gerät ausfällt}
    \catchword{Parallelschaltung: Das System bleibt verfügbar, solange mindestens ein
               Gerät funktioniert}
  }

  \question{Wie kann die Zuverlässigkeit eines Systems erhöht werden}{\page{248}}
  {
    \catchword{Einheiten mit genügend Reserve}
    \catchword{Ausreichende Kühlung}
    \catchword{Vorbeugende Wartung}
  }

  \statement{Erklären Sie die drei Arten der Reserve}{\page{249}}
  {
    \catchword{Kalte Reserve: Reservegerät ist nicht in Betrieb}
    \catchword{Warme Reserve: Reservegerät ist betriebsbereit angeschlossen und
               in Wartestellung}
    \catchword{Heiße Reserve: Reservegerät ist bereit parallel in Betrieb}
  }

  \statement{Erklären Sie das Prinzip der Mehrheitsentscheidung}{\page{250}}
  {
    \normaltext{Aufgaben werden auf unterschidlichen Rechnern mit unterschiedlicher Hard-
                und Software berechnet. Nach der Berechnung werden die Ergebnisse verglichen
                und es wird eine Mehrheitsentscheidnung getroffen.}
  }

  \subsubsection{Installation und Betrieb}

  \question{Welche Strörungen können aus dem Netz kommen}{\page{251}}
  {
    \catchword{Kurzzeitige, impulsförmige Störungen: Können durch Tiefpassschaltungen
               abgeblockt werden}
    \catchword{Netzausfälle: Können mit Unterbrechungsfreien Stromversorgungen (USV) und
               Notstromaggregaten überbrückt werden}
  }

  \question{Welche Sicherheitsvorkehrungen sollten im laufenden Betrieb ergriffen werden}{\page{252}}
  {
    \catchword{Backup der Daten}
    \catchword{Vorbeugende Wartung}
    \catchword{Repararturmaßnahmen protokollieren}
    \catchword{Schutzmaßnahmen gegen Fehlbedienung oder gar Sabotage}
    \catchword{Keine Entwicklungswerkzeuge}
  }

  \newpage
  \section{Prozessrechner-Software}

  \subsection{Strukturierung von Prozess-Software}

  \subsubsection{Klassische Strukturen der EDV}

  \subsubsection{Entscheidungstabellen}

  \subsubsection{Zustandsdiagramme und -tafeln}

  \subsubsection{Instanzennetze}

  \subsubsection{Petrinetze}

  \question{Wozu eignen sich Petrinetze insbesondere im Gegensatz zu 
            klassischen Strunkturierungs-Methoden von Prozessen}{\page{269}}
  {
    \normaltext{Neben statischen Strunkturen lassen sich auch dynamische
                Vorgänge darstellen, sogar parallele Arbeitsschritte}
  }

  \question{Was wird in einem Petrinetz duch Stellen, was durch Transitionen
            dargestellt}{\page{270}}
  {
    \catchword{Stellen: Zustand}
    \catchword{Transition: Aktion}
  }

  \question{Zeichnen Sie eine Konjunktion und eine Disjunktion}{\page{270}}
  {
    \catchword{Konjunktion: mehrere Eingangstellen auf einen Transition}
    \catchword{Disjunktion: mehrere Peie auf einen Stelle}
  }

  \statement{Zeichnen Sie einen Signalträger (Semaphor) in einem Petrinetz,
             der zwei Tasks gegenseitig aus einem Kritischen Abschnitt ausschließt}{\page{272}}
  {
    \normaltext{Bild 4.13, Seite 272 oben}
  }

  \statement{Zeichnen Sie ein Petrinetz, das zwei Tasks symmetrisch 
             synchronisiert}{\page{273}}
  {
    \normaltext{Bild 4.14, Seite 273 oben, ganz links}
  }

  \statement{Zeichnen Sie ein Petrinetz, das zwei Tasks asymmetrisch 
             synchronisiert}{\page{273}}
  {
    \normaltext{Bild 4.14, Seite 273 oben, Mitte}
  }

  \statement{Zeichnen Sie ein Petrinetz, dass den Ablauf einer
             Interrupt-Service-Routine erläutert}{\page{274}}
  {
    \normaltext{Bild 4.15, Seite 274 oben}
  }

  \subsubsection{Programmstrukturen}

  \subsection{Prozessrechner-Betriebssysteme}

  \subsubsection{Prozessrechnerbetrieb}

  \subsubsection{Betriebsysteme}

  \subsubsection{Das Taskmanagement}

  \statement{Beschreiben Sie den Zustand ruhend}{\page{290}}
  {
    \normaltext{Task ist vorhanden aber zur Zeit nicht zur Bearbeitung vorgesehen.
                Sie lagern auf der Platte, es gibt ein paar Zustandsinformationen
                und Informationen über ihren Betriebsmittelbedarf. In diesen Zusand
                kommen nsie nach ihrer Beendigung aus dem Zustand Rechnend oder nach
                ihrer Erstellung. Aus diesem Zustand kommt der Task in bereit, wenn
                er angefordert wird.}
  }

  \statement{Zeichnen Sie ein Diagramm in dem alle möglichen Zustände eines
             Tasks und ihren Übergängen enthalten ist}{\page{291}}
  {
    \normaltext{Bild 4.23, Seite 291 oben}
  }

  \statement{Beschreiben Sie den Zustand bereit}{\page{291}}
  {
    \normaltext{Task besitzt alle benötigten Betriebsmittel außer die CPU. Taks befinden
                sich hier normalerweise in einer Warteschlange, oft auch mit Prioritäten.
                Die Prozessverwaltung holt aus der Warteschlange die Tasks ab, die dann
                rechnend sind. Neben den aufgerufenen ruhenden Tasks kommen auch Tasks auch
                rechnen nach einem Interrupt oder aus dem Zustand wartend, wenn ihnen alle
                benötigten Betribesmittel zur Verfügung gestellt wurden.}
  }

  \statement{Beschreiben Sie den Zustand rechnend}{\page{291}}
  {
    \normaltext{Task besitzt die CPU. Es können also nur so viele Tasks wie CPUs rechnend sein}
  }

  \statement{Beschreiben Sie den Zustand wartend kurz}{\page{291}}
  {
    \normaltext{Tasks, die auf die Zuteilung von Betriebsmitteln warten. Sie können nach
                den Betriebsmittel sortiert sein und dort Warteschlangen bilden, auch hier
                kann es Prioritäten geben.}
  }

  \subsubsection{Die Datenverwaltung}

  \subsubsection{Betriebsmittelverwaltung}

  \subsubsection{Gerätetreiber}

  \subsubsection{Prozessbetrieb}

  \subsubsection{Systemverwaltung}

  \subsection{Realzeitprogrammiersprachen}

  \subsubsection{Anforderungen}

  \subsubsection{Assemblerprogrammierung}

  \subsubsection{Ablauforientierte Hochsprachen}

  \subsubsection{Systemprogrammiersprachen}

  \subsubsection{Prozessprogrammiersprachen}

  \subsubsection{Spezialsprachen}

  \newpage
  \section{Sonstiges}

  \subsection{Beispielklausur}
  
  \questionnopage{Welche Aufgaben lassen sich schlecht automatisieren}
  {
    \catchword{unvorhersehbare Ereignisse}
    \catchword{komplexe (z.B. Erkennung (visuell, akustisch))}
  }

  \questionnopage{Wie legt man den Automatisierungsgrad fest}
  {
    \normaltext{Abwägen: Kosten - Nutzen}
  }

  \questionnopage{Was bedeutet Durchsatz und in welchen Einheiten wird er gemessen}
  {
    \normaltext{Arbeitsmenge / Zeit, gemessen z.B. in Stück / Tag, Tonne / Stunde, Liter / Minute}
  }

  \questionnopage{Wie ist Auslasung definidert und welchen maximalen Wert kann sie erreichen,
                  welcher Wert wäre normal}
  {
    \catchword{Durchsatz / maximalen Durchsatz}
    \catchword{Maximal: 1}
    \catchword{Typisch: $ <1 \approx 0,92$}
  }

  \questionnopage{Was bedeutet Just In Time}
  {
    \normaltext{Es gibt keine Lagerhaltung, sie wird durch den Transport ersetzt. Manch ein
                LKW fährt auf deutschen Autobahnen als Lagerstelle rum.}
  }

  \questionnopage{Welche Vor- und Nachteile hat Just in Time}
  {
    \normaltext{Man spart Lagerkosten und man ist abhängig vom Verkehrsaufkommen}
  }

  \questionnopage{Welche Vor- und Nachteile hat Just in Time}
  {
    \normaltext{Man spart Lagerkosten aber man ist abhängig vom Verkehrsaufkommen}
  }
  
  \questionnopage{Welche Architekturen haben Sie kennengelernt (mit Skizze)}
  {
    \catchword{von Neumann}
    \catchword{Harvard}
    \catchword{Modifizierte Harvard}
  }
  
  \statementnopage{Skizzieren Sie die Erfassung und Ausgabe von Messgrößen in einem Rechner}
  {
  \normaltext{siehe Beispielklausur Aufgabe 5}
  }
  
  \statementnopage{Nennen Sie 3 Verfahren zur A/D-Wandlung}
  {
    \catchword{\important{Flash/Parallelwandler:} Widerstandskette teilt die Refernzspannung. 
    Mit einem Komperator wird die Eingangsspannung verglichen. Daraus ergibt sich eine 1 zu n
    Kodierung, die umkodiert werden muss.}
    \catchword{\important{Sägezahnwandler:} Eingangsspannung wird mit einem Sägezahnimpuls 
    verglichen. Der Takt gemessen (Zeit) bis der Komperator umschlägt.}
    \catchword{\important{Zählwandler:} Im Prinzip wie Sägezahnwandler, aber die 
    Vergleichsspannung wird über Zähler mit D/A-Wandler erzeugt; dadurch abgekürzter Zyklus möglich.}
  }
  
  \statementnopage{Skizzieren Sie detailiert die sukzessive Approximation (Skizze)}
  {
  \normaltext{siehe Beispielklausur Aufgabe 6b}
  }

  \questionnopage{Welche Kenngröße beschreibt die Leistungsähigkeit für Datenübertragung}
  {
    \normaltext{Das Bandbreiten - Längen - Produkt}
  }

  \questionnopage{Wie lauten typische Werte für zwei Leitungstypen}
  {
    \catchword{Einzelader: 1MHz * m}
    \catchword{Glasphaser: 1GHz * m}
  }
  
  \statementnopage{Skizzieren Sie das Ersatzschaltbild für Störungseinkopplung}
  {
  \normaltext{siehe Beispielklausur Aufgabe 7b}
  }

  \questionnopage{Was versteht man unter Leitungsanpassung}
  {
    \normaltext{Abschluss der Leitung mit ihrem Wellenwiderstand \resultol{keine Reflexion}}
  }
  
  \questionnopage{Welche Auswirkungen haben geometrische Ungleichmäßigkeiten}
  {
    \normaltext{Reflexionen an den Übergängen\result{überall kleine Reflektionen}
    \result{Leitungsrauschen}}
  }
  
  \statementnopage{Stellen Sie den prinzipiellen Ablauf eines Interrupts dar}
  {
    \catchword{externes Ereignis}
    \catchword{Erkennung des Gerätes}
    \catchword{Rettung des Kontextes}
    \catchword{Bearbeitung}
    \catchword{Widerherstellen des Kontextes}
    \catchword{vorheriges Programm läuft weiter}
  }
  
  \questionnopage{Wie kann einem Interrupt hardwaremäßig eine Priorität zugewiesen werden 
  (2 Methoden)}
  {
    \catchword{Daisychain}
    \catchword{Interruptkontroller}
  }
  
  \questionnopage{Wo im Rechner kann ein Interrupt maskiert werden}
  {
    \catchword{Gerät}
    \catchword{Interruptkontroller}
    \catchword{CPU}
  }
  
  \questionnopage{Wozu dient die ISR-Tabelle}
  {
    \normaltext{Zuweisung von Geräteidentifikatoren zu ISR-Adressen}
  }
  
  \questionnopage{Was bedeutet Maskierung einer Adresse}
  {
    \normaltext{Abschaltung von Interrupts durch spezielle Register}
  }
  
  \questionnopage{Wozu wird Maskierung verwendet}
  {
    \catchword{Priorisierung}
    \catchword{Vermeidung von Wiedereintritten bei nich reentrancy fähigem Code}
    \catchword{}
  }
  
  \statementnopage{Skizzieren Sie Folgesynchronisation als Petri-Netz}
  {
  \normaltext{siehe Beispielklausur Aufgabe 9}
  }
  
  \statementnopage{Skizzieren Sie Sperrsynchronisation als Petri-Netz}
  {
  \normaltext{siehe Beispielklausur Aufgabe 9}
  }
  
  \statementnopage{Petri-Netz-Aufgabe: A,B steigern. \catchword{A hat bis jetzt mehr geboten}
  \catchword{B wird gefragt} \catchword{Wenn B mehr bietet, wird A gefragt} \catchword{Wenn
  nicht bekommt A Gegenstand}}
  {
  \normaltext{siehe Beispielklausur Aufgabe 10}
  }
  
  \statementnopage{Beschreiben Sie detailiert den Verwaltungsablauf beim Zeitscheibenverfahren}
  {
  \catchword{Zeitzähler erzeugt Interrupt}
  \catchword{Interrupt startet Verwaltung}
  \catchword{Verwaltung hat eine geordnete Liste aller Tasks im Zustand \important{bereit}}
  \catchword{Auswahlverfahren zur Bestimmung der nächsten zu bedienenden Task}
  }

  \subsection{Vorlesung}

\statementnopage{Definieren Sie den Befriff Automatisierungsgrad}  
{
  /normaltext{Der Automatisierungsgrad gibt Auskunft über das Verhältnis 
  von manuellen zu automatisierten Fertigungsschritten. 
  Der gewählte Grad der Automatisierung ist in den meisten Fällen abhängig 
  von der zu produzierenden Stückzahl (Kosten-Nutzen), dem Bereich (Rohbau, 
  Lackiererei, Montage...), der Komplexität der Tätigkeit, den Lohnkosten 
  oder dem notwendigen Investitionsvolumen und dessen Finanzierbarkeit.}

}

  \questionnopage{Was bedeutet Echtzeit im Sinne der PDV}
  {
    \catchword{Schritthalten mit dem Prozess}
    \catchword{Direkte Kopplung mit dem Prozess}
    \catchword{Der Prozess gibt die Zeit vor}
  }
  
  
    \questionnopage{Welche Arbeitsverteilungen gibt es}
  {
    \catchword{rein sequentiell: Beispiel: Manufaktur}
    \catchword{FLießbandprinzip: Vorteil: Parallele Bearbeitung,
    leicht erlernbare einzelne Arbeitsschritte Problem: feste Taktzeit, alle 
    Bearbeitungsschritte müssen gleich viel Zeit in Anspruch nehmen}
    \catchword{entkoppeltes Fließbandprinzip: Vorteil: Durch Pufferung wird der
    fester Takt außer Kraft gesetzt}
    \catchword{Parallelverarbeitung mit Ausgleich unterschiedlich komplexer Arbeiten}
    \catchword{Pufferung durch Transport (Just in Time)}
  }
  
  \questionnopage{Warum ist in der PDV unter umständen einen demokratische 
  Entscheidungsfindung hinderlich, wann wird sie dennoch angewandt}
  {
    \catchword{zu teuer, aufwändig, langsam}
    \catchword{wenn man viel Zeit hat und die Entscheidung unsicher - Demokratie}
    \catchword{wenn einfache entscheidung, die schnell sein muss - Hierarchie}
  }
  
    \questionnopage{Welche Leistungseinheiten eines Rechners sind wichtig}
  {
    \catchword{Wortbreite in Bit}
    \catchword{Gleitkomma, Vektor-Rechenwerk}
    \catchword{CISC oder RISC}
    \catchword{Takt}
    
  }
  
    \questionnopage{Welche Effekte auf einer Leitung spielen bei hohen, 
    sehr hoehen und sehr niedrigen Frequenzen einen Rolle}
  {
    \catchword{sehr hohe: C und L spielen einen Rolle}
    \catchword{hohe Frequenzen: C spielt einen Rolle, L weniger, 
    das Signal wird gedämpft}
    \catchword{sehr niedrige: L und C spielen keine Rolle, Gleichstrom}
    
  }
  
  \questionnopage{Was gibt das Bandbreitenprodukt einer Leitung an}
  {
    \catchword{wie viele Informationen enthält eine Leitung bestimmter Länge}
    
  }
  
    \questionnopage{Was beeinflusst den komplexen Wellenwiederstand, und wozu ist 
    seine Kenntnis wichtig}
  {
    \catchword{H/m, F/m, Ohm/m, Leitung abschließen, Signalanpassung}
    
  }
  
    \questionnopage{Wie groß ist die Signalausbreitungsgeschwindigkeit auf einem 
    Kabel etwa}
  {
    \catchword{C/1,5, C $approx$ 300 000 000 m/sec}
    
  }
  
  
  
  
  
  
  

  
  \questionnopage{Welche Standard-Architekturen gibt es}
  {
    \catchword{Von-Neumann}
    \catchword{Havard}
    \catchword{Extended Havard}
  }

  \questionnopage{Was sind die typischen Eigenschaften eines Prozessrechners}
  {
    \catchword{Die Prozessperipherie ist direkt an den Prozess gekoppelt}
    \catchword{Echtzeitorganisation (Unterbrechungen)}
    \catchword{Freie Programmierbarkeit}
    \catchword{Einzelbitverarbeitung}
  }

  \questionnopage{Welche Größen sind schwierig zu messen}
  {
    \catchword{Rauhigkeit}
    \catchword{Textur einer Oberfläche, da es keine einheitliche Messvorschrift vorhanden ist}
  }

  \statementnopage{Erklären Sie die Handhabung der Komplexität durch Hierarchisierung}
  {
    \catchword{Vernetzung von Rechnern}
    \catchword{Kostensenkung}
    \catchword{Verfügbarkeit}
    \catchword{Spezialisierung erlaubt Kostensenkung}
    \catchword{Passenden Rechner vor Ort}
  }

  \questionnopage{Was sind die wichtigsten Kenngrößen von Prozessen}
  {
    \catchword{Durchsatz: In einer Zeitspanne produzierte Menge}
    \catchword{Auslastung: Momentane Produktion / mögliche Produktion}
  }

  \questionnopage{Warum gibt es in der Praxis keine 100\% Auslastung}
  {
    \catchword{Es müsste immer am Limit gearbeitet werden}
    \catchword{Standardabweichung}
  }

  \questionnopage{Was versteht man unter der Taktzeit}
  {
    \normaltext{Kehrwert des Durchsatzes}
  }

  \questionnopage{Was wird bei der Produktion mit Charges gemacht}
  {
    \catchword{Die entsprechende Chargesgröße wird produziert}
    \catchword{Danach wird umgerüstet, um etwas anderes zu produzieren}
    \result{Die Umrüstzeit muss zur Produktionzeit addiert werden}
  }

  \questionnopage{Wozu braucht man Meilensteinpläne}
  {
    \normaltext{Zur Überwachung eines größeren Projektes}
  }

  \statementnopage{Zeichnen Sie das Ersatzschaltbild einer Leitung}
  {
    \catchword{Die Leitung hat eine Induktivität $L$}
    \catchword{Die Leitung hat einen Widerstand $R$}
    \catchword{Die Leitung hat eine Kapazität $C$ gegenüber einer anderen Leitung oder Masse}
  }

  \questionnopage{Aus welchen Teilen besteht der komplexe Widerstand}
  {
    \catchword{Induktivitätsbelag $H/m$}
    \catchword{Kapazitätsbelag $C/m$}
    \catchword{Widerstandsbelag $\Omega/m$}
  }

  \questionnopage{Welche Störungstypen können bei Leitungen auftreten}
  {
    \catchword{Wechselspannungsstörungen (Gegentaktstörungen)}
    \catchword{Gleichspannungsstörungen (Gleichtaktstörungen)}
    \catchword{Galvanische Kopplung (Ausgleichsströme)}
  }

  \statementnopage{Nennen Sie Quellen der Störungen}
  {
    \catchword{Übersprechen}
    \catchword{Erdschleifen}
    \catchword{Hohe Leistung und Lastwechsel am Netz (Induktives 50Hz Brummen)}
    \catchword{Hochfrequenzgeräte hoher Leitungen}
  }

  \questionnopage{Welche Einkoppelungen treten bei niedrigen bzw. hohen Frequenzen auf}
  {
    \catchword{Niedrige Frequenzen: Induktivitäten}
    \catchword{Hohe Frequenzen: Kapazitäten}
  }

  \questionnopage{Wie kann bei elektrischen Störungen Abhilfe geschaffen werden}
  {
    \catchword{Abschirmung (weniger Kapazität)}
    \catchword{Abstand halten (weniger Induktivität)}
  }

  \questionnopage{Wie lassen sich Gegentaktstörungen verringern}
  {
    \normaltext{Durch hochsymmetrische Leitungen}
  }
  
  
  
  
  
  
  %%Wandler zusatzfragen...
  \subsection{Wandler-Zusatz}
  
\questionnopage{Was ist ein Komperator, wie wird er ralisiert, und was wird von 
ihm gefordert}
  {
    \catchword{Ein Komperator vergleicht zwei Eingangsspannung 
    und zeigt an einem Ausgang an, welche Spannung größer ist}
    \catchword{es handelt sich um einen analogen Verstrker mit einer möglichst
    starken Verstärkung}
  }
  
  \questionnopage{Welche Fehler treten beim Summen-Delta-Wandler üblicher Weise auf}
  {
      \catchword{Linearität wird verlassen, Paralleleffekt, je steiler, desto schlimmer}
  }
  
   \questionnopage{Was bedeutet Zeitkritisch}
  {
   \normaltext{Streng: jedes Ereignis muss vor jedem weiteren, das eintreffen kann
   abgearbeitet sein. Wenn es nicht zeitkristisch ist, dann ist eine Warteschlange 
   möglich.}
   \normaltext{Im Mittel muss die Bedingung aber immer noch gelten, sonst läuft
   die Warteschlange über oder einzelne Ereignisse verhungern.}
  
  }
  
  
 
 
  
    

  \statementnopage{Gehen Sie auf die Echtzeituhr ein}
  {
    \catchword{Absolute Zeit (MEZ): Festlegung durch das Physikalisch Technische Bundesamt}
    \catchword{Quarzuhren}
    \catchword{Batteriegepuffert}
    \catchword{Startzeitregister \resultol{Softwarerealisierung möglich}}
  }

  \questionnopage{Was sind die Unterschiede des kooperativen zum nicht kooperativen Multitasking}
  {
    \catchword{Kooperatives Multitasking: Verantwortung der Abgabe der CPU liegt beim Programm}
    \catchword{Nicht kooperatives Multitasking: Verantwortung der Abgabe der CPU liegt beim
               Betriebssystem}
  }

  \questionnopage{Auf welche Arten können Prozesse kommunizieren}
  {
    \catchword{Nachrichten: sicher, aber langsam}
    \catchword{Shared Memory: unsicher, aber schnell}
  }

  \statementnopage{Gehen Sie auf die Grundregeln für die Interruptprioritäten ein}
  {
    \catchword{Prozessdaten schnell abholen / speichern}
    \catchword{Abholung nicht unterbrechbar}
    \catchword{Ausgabewerte schnell berechnen / ausgeben}
    \catchword{Zeitkritische Interrupts müssen eine hohe Priorität haben}
  }

  \questionnopage{Was beschreibt bei Speichern die Zugriffszeit und was die Zykluszeit}
  {
    \catchword{Zugriffszeit: Zeit die ein Speicher braucht, um die angeforderten Daten zu liefern}
    \catchword{Zykluszeit: Zeit nach der ein Speicher wieder angefragt werden kann}
    \result {Zykluszeit ist immer größer gleich der Zugriffszeit}
  }

  \subsection{Übung}
  
  \questionnopage{Es soll eine Aufzugssteuerung als Petri-Netz modelliert werden.
  Dabei sollen das Tuer-Oeffen und -Schliessen, und das Anfahren und
  Anhalten beruecksichtigt werden. Wieviele parallele Aktionen
  finden statt}
  {
  \normaltext{siehe Übungen}
  }
  
  \questionnopage{Modellieren Sie das Erzeuger/Verbraucher-Problem: Der Erzeuger
  erzeugt ein Produkt und legt es auf eine Ablage, sofern diese
  frei ist. Danach kann ein weiteres Produkt erzeugt werden. Der
  Verbraucher nimmt - falls vorhanden - Produkte aus der Ablage.
  Aus welchen nebenlaeufigen Strukturen setzt sich das zu
  modellierende System zusammen}
  {
  \normaltext{siehe Übungen}
  }
  
  \statementnopage{Eine Ampelkreuzung soll modelliert werden, wobei Abbiegespuren,
  etc. nicht beruecksichtigt werden muessen. Da diagonal
  entgegengesetzte Ampeln die gleiche Farbe anzeigen, muessen nur
  zwei Ampeln modelliert werden. Weiterhin sollen nur die Farben
  Rot und Gruen beruecksichtigt werden. Die Ampeln besitzen
  Induktionsschleifen}
  {
  \normaltext{siehe Übungen}
  }
  
  \statementnopage{Ein Produktionssystem bestehe aus drei Maschinen M1, M2, M3. Die
  beiden Maschinenbediener B1, B2 erledigen eintreffende Auftraege
  nach folgendem Vorgehen: Jeder Auftrag wird zuerst von M1
  bearbeitet, danach wahlweise von M2 oder M3. Der Bediener B1
  bedient M1, M2. Der Bediener B2 bedient die Maschinen M1, M3}
  {
  \normaltext{siehe Übungen}
  }
  
  \statementnopage{Es soll ein Parkplatz vor einem Geschaeft als Petri-Netz
  modelliert werden. Der Parkplatz besteht aus drei Stellplaetzen.
  Hin und wieder kommen Autos, um dort zu parken. Es besteht jedoch
  auch die Moeglichkeit, dass die Autos wieder wegfahren, ohne dort
  zu parken, beispielsweise, weil der Parkplatz voll ist. Als
  Anfangszustand soll nur der mittlere Stellplatz von einem Auto
  belegt sein}
  {
  \normaltext{siehe Übungen}
  }
  
  \statementnopage{Sobald ein Flugzeug gelandet ist, wird die Gangway herangefahren
  und die Passagiere können aussteigen. Erst wenn alle Passagiere
  ausgestiegen sind, kann der Tankwagen heranfahren und das Flugzeug
  betanken. Die Passagiere werden per Bus in die Abfertigungshalle
  gebracht. Nach einem Sicherheitscheck ist das Flugzeug zum
  nächsten Start bereit. Modellieren Sie ein Stellen/Transitionen-
  Petrinetz für diesen Prozess}
  {
  \normaltext{siehe Übungen}
  }
  
  \statementnopage{Eine Hose, ein Paar Socken, ein Paar Schuhe und ein Pullover liegen
  bereit und werden in einer sinnvollen Reihenfolge angezogen.
  Modellieren Sie die verschiedenen Moeglichkeiten als Petri-Netz}
  {
  \normaltext{siehe Übungen}
  }
  
  \statementnopage{Modellieren Sie die Zubereitung von Ruehrei mit Speck als Petri-Netz.
  Zutaten: 300g durchwachsenen Speck, etwas Bratfett, 10 Eier, Salz}
  {
  \normaltext{siehe Übungen}
  }
  
  \statementnopage{Modellieren sie die Tankvorgaenge an einer Tankstelle mit zwei
  Zapfsaeulen (mit jeweils einem Standplatz) und einer Kasse als
  Petri-Netz}
  {
  \normaltext{siehe Übungen}
  }
  
  \statementnopage{Modellieren sie die Arztpraxis mit Anmeldung, 2 Ärzten und 3 Stühlen zum warten. 
  \catchword{sagen zu welchem Arzt man will}
  \catchword{setzen sich auf die Stühle}
  \catchword{gehen zum Arzt wenn frei}}
  {
  \normaltext{siehe Übungen}
  }

\end{enumerate}


\end{document}
