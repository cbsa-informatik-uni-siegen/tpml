\documentclass[a4paper,12pt]{article}

\usepackage{amsmath}
\usepackage{amssymb}
\usepackage{amstext}
\usepackage{german}
\usepackage{ngerman}
\usepackage[utf8]{inputenc}
\usepackage{color}
\usepackage{latexsym}
\usepackage{array}
\usepackage{stmaryrd}
\usepackage{hyperref}
\usepackage{index}
\usepackage{graphicx}

\title{{\Huge Echtzeitsysteme}\\Fragenkatalog}
\author{{\Large Christian Fehler, Benjamin Mies, Michael Oeste}}
\date{\small\today}

\newindex{default}{idx}{ind}{Stichwortverzeichnis}

\newcommand{\question}[3]{\pagebreak[3]\item {\textbf{#1?}}\ (S.\ #2)#3}
\newcommand{\statement}[3]{\pagebreak[3]\item {\textbf{#1!}}\ (S.\ #2)#3}
\newcommand{\catchword}[1]{\\-\ #1}
\newcommand{\normaltext}[1]{\\#1}
\newcommand{\result}[1]{\\ $\Rightarrow$\ #1}
\newcommand{\page}[1]{#1}
\newcommand{\pages}[2]{#1\ -\ #2}

\begin{document}

\maketitle
\newpage
\tableofcontents

\newpage
\section{Einführung}

\begin{enumerate}

  \question{In welchen Bereichen sind Automaten dem Menschen unterlegen}{\page{1}}
  {
    \catchword{Erkennen von Dingen}
    \catchword{Erkennen von Situationen}
    \catchword{Freiheit der Entscheidung}
  }

  \question{In welchen Bereichen sind Automaten dem Menschen überlegen}{\page{1}}
  {
    \catchword{Schnelligkeit}
    \catchword{Präzision}
  }

  \question{Wie nennt man den Vorgang zur Einführung von Automaten}{\page{1}}
  {
   \normaltext{Automatisierung}
  }

  \question{Wie nennt man den Umfang, in dem ein Vorgang oder Prozess automatisiert ist}{\page{1}}
  {
    \normaltext{Automatisierungsgrad}
  }

  \question{Wann ist dieser umso höher}{\page{1}}
  {
    \normaltext{Je einfacher ein Vorgang oder Prozess ist}
  }

  \question{Welche zwei Einheiten findet man bei automatisierten Prozessen}{\page{1}}
  {
    \catchword{Technischer Prozess}
    \catchword{Automatische Steuerung}
  }

  \question{Wie lautet die DIN Definition eines Automaten}{\page{2}}
  {
    \normaltext{Ein Automat ist ein künstliches System, das selbstständig ein Programm
                befolgt. Auf Grund des Programms trifft das System Entscheidungen, die
                auf der Verknüpfung von Eingaben mit dem jeweiligen Zustand des Systems
                beruhen, und Ausgaben zur Folge haben.}
  }

  \statement{Nennen Sie die zwei wesentlichen Merkmale eines Automaten}{\page{2}}
  {
    \catchword{Grundstruktur aus einer kausalen Folge von Eingabe, Verarbeitung und Ausgabe}
    \catchword{Die Verarbeitung beinhaltet Entscheidungen zwischen verschiedenen Möglichkeiten}
  }

  \question{Welche Ziele hat die Automatisierung}{\page{2}}
  {
    \catchword{Vereinfachung des Prozesses}
    \catchword{Menschen für Arbeiten mit höherem Anspruch freisetzen}
    \catchword{Höhere Produktivität}
  }

  \statement{Nennen Sie die fünf typischen Bereiche,
             in denen der Mensch Entlastung erfahren kann}{\pages{2}{4}}
  {
    \catchword{Arbeiten, die regelmäßig und monoton wiederkehren}
    \catchword{Arbeiten, die hohe Anforderungen an die Konzentration stellen. Mit wechsel von Zeiten
               hoher Aktivität und hoher Anforderung an die Aufmerksamkeit und Zeiten geringer Aktivität,
               die mit Langeweile ausgefüllt sind.}
    \catchword{Arbeiten, die eine hohe Arbeitsgeschwindigkeit verlangen}
    \catchword{Arbeiten, die eine große Datenmenge liefern}
    \catchword{Arbeiten, die in gesundheitsgefährdender Umgebung stattfinden}
  }

  \question{Welche Bereiche umfasst eine höhere Produktivität}{\page{4}}
  {
    \catchword{Größere Menge (Quantität)}
    \catchword{Bessere Qualität}
  }

  \statement{Nennen Sie die Grenzen der Automatisierung}{\page{4}}
  {
    \catchword{Ein automatisches System kann nicht auf alle vorkommenden Ereignisse reagieren}
    \catchword{Kleine Fehler können katastrophale Folgen haben}
  }

  \question{In welchen Bereichen ist die Erkenntnisfähigkeit des Menschen notwendig}{\page{4}}
  {
    \catchword{Spracherkennung}
    \catchword{Bildverarbeitung}
  }

  \question{Was ist die Besonderheit bei der PDV}{\page{5}}
  {
    \normaltext{Die Lenkung eines technischen Prozesses}
  }

  \question{Welche besonderen Anforderungen werden an den Rechner gestellt}{\page{5}}
  {
    \catchword{Einhaltung der zeitlichen Abfolgen}
    \catchword{Einhaltung der vorgegebenen Bearbeitungszeiten}
    \result{Realzeitbedingungen einhalten}
  }

  \question{Wie ist der Begriff Prozess definiert}{\page{6}}
  {
    \normaltext{Der Vorgang zur Umformung, zum Transport oder zur Speicherung von Materie, 
    Enerie oder Information. Ein technischer Prozess ist ein Prozess, dessen Zustandsgrößen
    mit technischen Mitteln erfasst und beeinflusst werden können.}
  }


  \question{Welche Anforderungen stellt das Einhalten der zeitlichen Abfolgen}{\page{5}}
  {
    \catchword{Schritthaltende Verarbeitungsgeschwidigkeit}
    \catchword{Viele Entscheidungsmöglichkeiten in einem Programm}
    \catchword{Sicherheit}
  }

  \question{Welche zwei Einheiten kommunizieren in der PDV}{\page{5}}
  {
    \catchword{Rechner}
    \catchword{Prozess}
  }

  \question{Wie lautet die DIN Definition eines Prozesses}{\page{6}}
  {
    \normaltext{Der Vorgang zur Umformatierung, zum Transport oder zur Speicherung von Materie,
                Energie oder Information. Ein technischer Prozess ist ein Prozess, dessen
                Zustandsgrößen mit technischen Mitteln erfasst und beeinflusst werden können.}
  }

  \question{Was bezeichnet man als Verarbeitungsart}{\page{6}}
  {
    \catchword{Umformung}
    \catchword{Transport}
    \catchword{Speicherung}
  }

  \question{Aus welchen zwei Teilen besteht ein Elementarprozess}{\page{6}}
  {
    \catchword{Verarbeitungsart}
    \catchword{Verarbeitungsgut}
  }

  \question{In welche drei Teile lässt sich die Verarbeitungsart unterteilen}{\page{6}}
  {
    \catchword{Transport}
    \catchword{Umformung}
    \catchword{Speicherung}
  }

  \question{In welche drei Teile lässt sich das Verarbeitungsgut unterteilen}{\page{6}}
  {
    \catchword{Materie}
    \catchword{Energie}
    \catchword{Information}
  }

  \question{Mit welchem Prozess ist ein Speicherprozess fast immer gekoppelt}{\page{6}}
  {
    \normaltext{Transportprozess}
  }

  \question{Mit welchem Prozess ist ein Umformunsprozess fast immer gekoppelt}{\page{6}}
  {
    \normaltext{Transport- oder Speicherprozess}
  }

  \newpage
  \section{Prozessrechner Hardware}


  \newpage
  \section{Periphere Geräte}


  \newpage
  \section{Prozessrechner-Software}


  \newpage
  \section{Sonstiges}


\end{enumerate}


\end{document}