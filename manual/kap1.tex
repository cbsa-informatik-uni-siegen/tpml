%% $Id$
%%
%% Kapitel 1 - Einleitung
%%

\chapter{Einleitung}



\section{Systemvoraussetzungen}

\TPML\ wurde vollst"andig in Java entwickelt, und ben"otigt zur Ausf"uhrung lediglich ein JRE oder
JDK in der Version 5.0 oder neuer.
Einige System, wie zum Beispiel Mac OS X Tiger, diverse Linux Distributionen und angepasste Windows
OEM Installation, enthalten bereits Java 5.0. Ansonsten kann die neueste Version der Java Runtime
Environment (optional kombiniert mit dem Java Development Kit) von der Sun Microsystems Webseite
heruntergeladen werden.

\url{http://java.sun.com/}

Bei manueller Installation des JRE oder JDK unter Unix/Linux-Systemen muss darauf geachtet werden, dass
das {\tt java}-Binary anschlie"send im {\tt \$PATH} verf"ugbar ist. Wird das JDK zum Beispiel unter
{\tt /usr/local/jdk1.5.0} installiert, muss anschlie"send im Profil der Shell {\tt /usr/local/jdk1.5.0/bin}
zum {\tt \$PATH} hinzugef"ugt werden. Im Falle der {\tt bash} w"are dies in der Datei {\tt .bashrc}
im Homeverzeichnis ein Eintrag
\begin{verbatim}
  export PATH="/usr/local/jdk1.5.0/bin:$PATH"
\end{verbatim}
und anschlie"send muss entweder die Datei mittels {\tt source .bashrc} neugeladen oder die
Shell neu gestartet werden.

Unter Windows "ubernimmt das Java Installationsprogramm die Aufgabe die entsprechenden Systemeinstellungen
anzupassen. Es sind in der Regel nach der Installation keine weitere Anpassungen mehr notwendig.



\section{Installation}

Das Programm steht als ZIP- und TAR-Archiv auf der \TPML-Webseite zum Download bereit. Hier findet sich
auch immer die jeweils aktuellste Version des Benutzerhandbuchs.

\url{http://tinyurl.com/y6ov4u}

Die Installation ist so einfach wie m"oglich gehalten. Nach dem Download des ZIP- oder TAR-Archivs muss dieses
lediglich entpackt werden. Unter Mac OS X erledigt dies der StuffIt Expander, unter Windows kann in
neueren Versionen entweder die in den Windows Explorer integrierte Unterst"utzung f"ur komprimierte
Ordner oder ein externes Programm wie zum Beispiel WinZIP verwendet werden. Unter Unix/Linux kann
das TAR-Archiv mittels dem {\tt tar}-Kommando entpackt werden\footnote{Oder alternativ nat"urlich
auch "uber grafische Tools wie Xarchiver, File Roller oder Ark.}.
\begin{verbatim}
  bzcat tpml-X.Y.Z.tar.bz2 | tar xf -
\end{verbatim}
Anschlie"send wechselt man in das erzeugte Verzeichnis {\tt tpml-X.Y.Z} und f"uhrt dort das Shellskript
{\tt tpml.sh} aus. Sofern eine geeignete JavaSE 5.0 Installation gefunden wurde, startet nun \TPML,
ansonsten gibt es eine Fehlermeldung, und es sollte "uberpr"uft werden, ob wirklich ein JRE oder JDK
5.0 installiert ist, und sichergestellt werden, dass das {\tt java}-Binary im {\tt \$PATH} liegt. Beim
ersten Start von {\tt tpml.sh} wird das Programm im System registriert, das hei"st im Men"u erscheint
ein Eintrag f"ur \TPML\ und Dateien k"onnen in standardkonformen Dateimanagern (zum Beispiel Thunar
und Nautilus) per Doppelklick ge"offnet werden.

Unter Windows gen"ugt ein Doppelklick auf die Datei {\tt tpml.exe}. Sollte keine JavaSE 5.0
Installation gefunden werden, oder ist Java nicht korrekt im System eingerichtet erscheint
ein Fehlerdialog.

Sollte die Versionskontrolle einmal versagen kann sie auch "ubergangen werden. Dazu muss die JAR-Datei des UI-Paketes manuell gestartet werden.
\begin{verbatim}
  java -jar de.unisiegen.tpml.ui-X.Y.Z.jar -f
\end{verbatim}



\section{Erste Schritte}

Nach dem Start des Programms k"onnen neue Dateien erstellt werden. Dateien sind immer mit einer
Programmiersprache verkn"upft, die durch die Dateiendung identifiziert wird\footnote{In gleicher
Weise wie {\tt .java} eine Java-Datei bezeichnet und {\tt .c} eine C-Datei.}. In \TPML\ wurden
die Sprachen \LZERO\ bis \LFOUR\ realisiert, die den wesentlichen in der Vorlesung behandelten
Sprachen entsprechen, wobei die Namen und der Umfang
einzelner Sprachen von denen in der Vorlesung abweichen kann. Die Sprachen werden in
Kapitel~\ref{DieSprachenImDetail} im Detail behandelt, welches auch als Nachschlagewerk
dienen soll.

Um eine neue Datei zu erstellen w"ahlt man aus dem Men"u {\bf Datei} den Eintrag {\bf Neu...}
worauf sich ein Dialog "offnet, in dem die f"ur die Datei zu benutzende Programmiersprache
ausgew"ahlt werden muss. Da die Vorlesung mit dem ungetypten $\lambda$-Kalk"ul beginnt,
w"ahlen wir also auch hier ein einfaches Programmbeispiel in der Programmiersprache \LZERO. Die Sprachen sind in Kategorien unterteilt, \LZERO\  ist nat"urlich in der ersten ganz oben zu finden. 
Nach der Auswahl der Programmiersprache "offnet sich ein Texteditor, in den der Programmtext
eingegeben werden kann. Hierzu w"ahlen wir das Beispiel der Identit"atsfunktion, angewandt
auf sich selbst.
\begin{verbatim}
  (lambda x.x) (lambda x.x)
\end{verbatim}
Zugegebenerma"sen nicht das beindruckendste Programm, aber f"ur ein erstes Beispiel doch
sehr gut geeignet. Wie man bei der Eingabe des obigen Programms merken wird, unterst"utzt
der Editor Syntaxhighlighting und markiert automatische Fehler im Programmtext, wie man
es in der Regel von integrierten Entwicklungsumgebungen gewohnt ist\footnote{Nichtsdestotrotz
sollte \TPML\ nicht mit einer IDE verwechselt, da es strikt als Lernwerkzeug ausgelegt ist.}.

Anschlie"send kann man nun den big oder small step Interpreter starten und einen Programmablauf
durchspielen, der bei dem obigen Programm verst"andlicherweise nicht sonderlich spektakul"ar
ist. F"ur das Beispiel benutzen wir den small step Interpreter. Dazu w"ahlt man aus dem
Men"u {\bf Beweis} den Eintrag {\bf Small Step}. Es "offnet sich nun ein weiterer mit dieser
Datei verbundener Reiter {\bf Small Step} mit dem aus der Vorlesung bekannten Aussehen einer
small step Herleitung. Der Interpreter erwartet nun die Auswahl einer small step Regel um den
Beweis zu vervollst"andigen. Hierzu klickt man den grauen Button, und w"ahlt dann aus
dem Men"u die n"achste anzuwendende Regel, in unserem Fall die (BETA-V) Regel.

Nach Anwendung der Regel zeigt der Interpreter den n"achsten Beweisschritt, der in diesem
Fall auch schon der letzte ist, da $\lambda x.x$ bereits ein Wert ist, der naturgem"a"s
nicht weiter ausgewertet werden kann.

Dies soll dem Zweck der \emph{ersten Schritte} gen"ugen. Das n"achste Kapitel beschreibt
die Benutzeroberfl"ache und die Beweiswerkzeuge im Detail.



% vi:set syntax=tex ts=2 sw=2 et encoding=UTF-8:
