
\myslide {Die Renderer - Allgemein} {

  \begin{itemgroup}{Das {\tt de.unisiegen.tpml.graphics} Projekt}
    \item graphische Repr"asentation der Kernkomponenten
    \item realisiert den \cemph{Editor} (mit Syntaxhightlighting) und die \cemph{Renderer}
    \item ein Renderer pro Beweiswerkzeug (Small/Bigstep Interpreter, Type Checker)
    \item Renderer sind Schnittstelle zwischen Benutzer und Kernkomponenten
    \item Renderer absolut \cemph{sprachneutral}
  \end{itemgroup}
}



\myslide {Die Renderer - Basiskomponenten I} {

  \graphicxbox{20cm}{images/prettystring.png}
  \begin{itemgroup} {Der {\tt PrettyStringRenderer} }
	  \item zeichnet vom {\tt ProofModel} erzeugte {\tt PrettyStrings}
    \item syntaxhighlighting erfolgt "uber Characterattributes
    \item anpassung an Fensterbreite "uber Zeilenumbr"uche
    \item Zeilenumbr"uche werden "uber Annotations aus Prettystring ermittelt
  \end{itemgroup}
}

\myslide {Die Renderer - Basiskomponenten II} {
  \begin{itemgroup} {Der {\tt EnvironmentRenderer} }
    \item zeichnet vom {\tt ProofModel} erzeugte {\tt Environment}
    \item Zeichnen erfolgt einfach "uber {\tt JLabel}
    \item nur Darstellung des ersten Eintrags, weiter Eintr"age durch {\tt ,...}
    \item Komplette Umgebung im {\it Tooltipp} zu sehen
  \end{itemgroup}
}

\myslide {Die Renderer - Basiskomponenten III} {
  \graphicxbox{20cm}{images/compoundexpression.png}
  \begin{itemgroup} {Der {\tt CompoundExpressionRenderer} }
    \item Kapselung des {\tt PrettyStringRenderer}s und {\tt EnvironmentRenderer}s
    \item Aussehen wird an Typ der Umgebung unterschieden 
    \begin {itemgroup} {}
      \item bei {\tt Small/BigStepper} Umgebung hinter Ausdruck; geklammerte Darstellung
      \item beim {\tt TypeChecker} Umgebung vor Ausdruck
    \end{itemgroup}
  \end{itemgroup}

}

\myslide {Die Renderer - Basiskomponenten IV} {

  \graphicxbox{7cm}{images/menubutton.png}
  \begin{itemgroup}{Der Menubutton}
    \item haupt Interaktionswerkzeug
    \item eigenes Erscheinungsbild
    \item Auswahlm"oglichkeit von Regeln oder Anweisungen ({\it Guess, Typ eingeben, usw...})
    \item Auswahlmenu durch {\tt JPopupMenu} dargestellt
  \end{itemgroup}

}

\myslide {Die Renderer - Beweiswerkzeuge I} {
  \begin{itemgroup}{Allgemeine Struktur der Komponenten}
    \item f"ur jedes Beweiswerkzeug eine \cemph{View} (z.B. {\tt SmallStepView})
    \item View implementiert Schnittstelle zur {\it Haupt}-GUI (Kommandos wie {\it Guess}, usw...)
    \item in jeder View liegt eine \cemph{Component} (z.B. {\tt SmallStepComponent})
    \item die Component entscheidet "uber das Layouting der einzelnen Knoten
    \item repr"asentation der einzelnen Knoten erfolgt durch \cemph{NodeComponents} (z.B. {\tt SmallStepNodeComponent})
    \item NodeComponent als {\it Userobject} an \cemph{ProofNode} gebunden
    
  \end{itemgroup}
}

\myslide {Die Renderer - Beweiswerkzeug II} {

  \begin{itemgroup}{Graphischer Abgleich mit Kernkomponenten}
    \item {\tt ProofModel} informiert "uber "Anderungen (hinzuf"ugen, l"oschen ...)
    \item geringer Bearbeitungsbedarf, da {\tt NodeComponent Userobject} 
    \item im {\tt Normalfall} nur Relayout
  \end{itemgroup}

  \animate{1}{
    \begin{itemgroup}{Anweisungen an Kernkomponenten}
      \item {\tt NodeComponent} informatiert "uber Aktivit"at
      \item {\tt Component} leitet Knoten und Kommando (Regelauswahl, Knoten raten... ) an {\tt ProofModel} weiter
      \item {\tt ProofModel} regiert seinerseits mit "Anderungen
    \end{itemgroup}
  }
}

\myslide {Die Renderer - Beweiswerkzeuge III} {

  \begin{itemgroup}{Bigstepper und Typechecker}
    \item echte Baumstruktur
    \item daher graphisch {\it fast} Identisch
    \item Layouting erfolgt rekursive "uber den {\it ProofTree}
    \item Zeichnen der Baumdekoration durch {\tt TreeArrowRenderer}
  \end{itemgroup}

}

\myslide {Die Renderer - Beweiswerkzeuge IV} {

  \graphicxbox{20cm}{images/smallstepnode.png}
  \begin{itemgroup}{Smallstepper}
    \item obwol Darstellung als Liste, interne Darstellung ebenfalls Baum
    \item {\tt SmallStepNodeComponent} hat versetzte Darstellung, deswegen kein einfaches Layouting
    \item Platzierung und wahl der Dimensionen abh"angig von {\it Vater-} und {\it KindKnoten}
  
  
  \end{itemgroup}
 
}


