\documentclass[fleqn,landscape,titlepage,german]{myslides}
%
% Praesentation der Projektgruppe TPML, 2006
%
% 2006 Benedikt Meurer <benny@xfce.org>
% 2006 Marcell Fischbach <marcellfischbach@gmx.de>
%
 
\usepackage{amssymb}
\usepackage[latin1]{inputenc}
\usepackage{german}
\usepackage{ngerman}

% Einkommentieren, um Handouts zu generieren
%\createhandoutstrue

\begin{document}
  
\presentation[TPML 1.0]{Marcell\,Fischbach,\,Benedikt\,Meurer}{TPML 1.0 - Ein interaktives Lernwerkzeug}
\maketitle

\addoverviewitem{Einleitung}
\addoverviewitem{Die Sprachen}
\addoverviewitem{Die Renderer}
\addoverviewitem{Die Benutzerschnittstelle}
\addoverviewitem{TPML 2.0}
\addoverviewitem{Fazit}

\CenteredGraphics



%%
%% Einleitung
%%

\makeoverviewslide

\myslide{Die Vorlesung ,,Theorie der Programmierung I''}{
  \begin{itemgroup}{,,Theorie der Programmierung I''}
    \item Vermittelt Grundlagen von Programmiersprachen anhand einfacher funktionaler Programmiersprache (orientiert an O'Caml)
    \item Operationelle Semantik anhand von small und big steps
    \item Typsysteme (Polymorphie, Typinferenz)
    \item Pflichtveranstaltung f"ur Informatik-Studenten
  \end{itemgroup}

  \animate{1}{
    {\bf Problem:} Einstieg in den Stoff wird als schwierig empfunden!
  }
}

\myslide{Motivation I}{
  {\bf Frage:} Warum ein Programm zur ,,Theorie der Programmierung''?

  \animate{1}{
    \begin{itemgroup}{Gr"unde}
      \item Erleichterter Zugang zu theoretischen Inhalten.
      \item Regelwerk durch ,,praktisches Arbeiten'' besser zu verstehen.
      \item ,,Gef"uhl'' f"ur Interpreter und Typsysteme vermitteln.
      \item Hilfestellung bei der Pr"ufungsvorbereitung ($\to$ nicht auf Beispiele aus "Ubung/Vorlesung beschr"anken)
      \item Verst"andnis f"ur Zusammenhang von Programmtext und Ausf"uhrung auf einer Maschine.
    \end{itemgroup}
  }
}

\myslide{Motivation II}{
  \begin{itemgroup}{Warum nicht O'Caml?}
    \item ,,Echte Implementierungen'' ungeeignet.
    \item Abl"aufe nicht sichtbar f"ur den Benutzer
    \item Typfehler von O'Caml schwer verst"andlich
    \item Semantik nicht zug"anglich ($\to$ wird in Byte/native Code kompiliert)
    \item Fehlende formale Spezifikation!
  \end{itemgroup}

  \animate{1}{
    {\bf Folgerung:} Spezielles Lernprogramm besser geeignet.
  }
}

\myslide{Ziele}{
  \begin{itemgroup}{Ziele der Entwicklung}
    \item Geeignete Visualierung von Abl"aufen.
    \item Leichter Zugang zum Regelwerk.
    \item Leicht erweiterbar ($\to$ Vorlesung sehr dynamisch).
    \item Plattformunabh"angig einsetzbar.
    \item Flache Lernkurve ($\to$ IDE)
  \end{itemgroup}
}

\myslide{Planung I}{
  {\bf Dazu:} Planung der Ziele im einzelnen.

  \animate{1}{
    \begin{itemgroup}{Geeignete Visualierung von Abl"aufen}
      \item User Interface soll Darstellung aus der Vorlesung bzw. "Ubung\\ entsprechen.
      \item Benutzer kann selbstst"andig t"atig werden.
      \item Programm kann automatisch vervollst"andigen\\ (Schrittweise/Komplett).
    \end{itemgroup}
  }
}

\myslide{Planung II}{
  \begin{itemgroup}{Leichter Zugang zum Regelwerk}
    \item Unterschiedliche Sprachen verf"ugbar (derzeit ${\cal L}_0$ bis ${\cal L}_4$).
    \item Inkrementelles Erlernen komplexerer Sprachfeatures m"oglich.
    \item Benutzer wird nicht direkt mit der M"achtigkeit\\ ($\to$ Un"ubersichtlichkeit) von ${\cal L}_4$ konfrontiert.
  \end{itemgroup}
}

\myslide{Planung III}{
  \begin{itemgroup}{Leicht erweiterbar}
    \item Vorlesung ,,Theorie der Programmierung'' "andert sich j"ahrlich.
    \item Neue Konzepte kommen hinzu.
    \item Programm soll einfach um neue Sprachen erweiterbar sein.
    \item Benutzerschnittstelle soll um neue Features erweiterbar sein \\ ($\to$ Highlighting von Bindungen).
  \end{itemgroup}
}

\myslide{Planung IV}{
  \begin{itemgroup}{Plattformunabh"angig einsetzbar}
    \item Programm sollte ,,out-of-the-box'' funktionieren \\ ($\to$ keine Installation).
    \item M"oglichst unabh"angig von einer speziellen Plattform \\ ($\to$ nur ein Binary/Download).
    \item Wahl zugunsten Java 5.0, O'Caml Framework ungeeignet \\ f"ur die Benutzerschnittstelle.
  \end{itemgroup}

  \animate{1}{
    \begin{itemgroup}{Flache Lernkurve}
      \item Benutzerschnittstelle angelehnt an g"angige IDEs.
      \item M"oglichst keine Einrichtung notwendig.
    \end{itemgroup}
  }
}

\myslide{Realisierung}{
  \begin{itemgroup}{Werkzeuge}
    \item Plattform: JavaSE 5.0
    \item Entwicklungsumgebung: Eclipse 3.1.2
    \item GUI Builder: NetBeans 5.0
    \item Lexer Generator: JFlex 1.4.1
    \item Parser Generator: JavaCUP 11a
    \item Protokollierung: log4j 1.2.14
  \end{itemgroup}

  \begin{itemgroup}{Grundlage}
    \item Vorlesungsinhalt ,,Theorie der Programmierung I'' (2005/2006)
    \item OCaml 3.09.3
  \end{itemgroup}
}



%%
%% Die Sprachen
%%

\makeoverviewslide




\makeoverviewslide


\myslide {Die Renderer - Allgemein} {

  \begin{itemgroup}{Das {\tt de.unisiegen.tpml.graphics} Projekt}
    \item graphische Repr"asentation der Kernkomponenten
    \item realisiert den \cemph{Editor} (mit Syntaxhightlighting) und die \cemph{Renderer}
    \item ein Renderer pro Beweiswerkzeug (Small/Bigstep Interpreter, Type Checker)
    \item Renderer sind Schnittstelle zwischen Benutzer und Kernkomponenten
    \item Renderer absolut \cemph{sprachneutral}
  \end{itemgroup}
}



\myslide {Die Renderer - Basiskomponenten I} {

  \graphicxbox{20cm}{images/prettystring.png}
  \begin{itemgroup} {Der {\tt PrettyStringRenderer} }
	  \item zeichnet vom {\tt ProofModel} erzeugte {\tt PrettyStrings}
    \item syntaxhighlighting erfolgt "uber Characterattributes
    \item anpassung an Fensterbreite "uber Zeilenumbr"uche
    \item Zeilenumbr"uche werden "uber Annotations aus Prettystring ermittelt
  \end{itemgroup}
}

\myslide {Die Renderer - Basiskomponenten II} {
  \begin{itemgroup} {Der {\tt EnvironmentRenderer} }
    \item zeichnet vom {\tt ProofModel} erzeugte {\tt Environment}
    \item Zeichnen erfolgt einfach "uber {\tt JLabel}
    \item nur Darstellung des ersten Eintrags, weiter Eintr"age durch {\tt ,...}
    \item Komplette Umgebung im {\it Tooltipp} zu sehen
  \end{itemgroup}
}

\myslide {Die Renderer - Basiskomponenten III} {
  \graphicxbox{20cm}{images/compoundexpression.png}
  \begin{itemgroup} {Der {\tt CompoundExpressionRenderer} }
    \item Kapselung des {\tt PrettyStringRenderer}s und {\tt EnvironmentRenderer}s
    \item Aussehen wird an Typ der Umgebung unterschieden 
    \begin {itemgroup} {}
      \item bei {\tt Small/BigStepper} Umgebung hinter Ausdruck; geklammerte Darstellung
      \item beim {\tt TypeChecker} Umgebung vor Ausdruck
    \end{itemgroup}
  \end{itemgroup}

}

\myslide {Die Renderer - Basiskomponenten IV} {

  \graphicxbox{7cm}{images/menubutton.png}
  \begin{itemgroup}{Der Menubutton}
    \item haupt Interaktionswerkzeug
    \item eigenes Erscheinungsbild
    \item Auswahlm"oglichkeit von Regeln oder Anweisungen ({\it Guess, Typ eingeben, usw...})
    \item Auswahlmenu durch {\tt JPopupMenu} dargestellt
  \end{itemgroup}

}

\myslide {Die Renderer - Beweiswerkzeuge I} {
  \begin{itemgroup}{Allgemeine Struktur der Komponenten}
    \item f"ur jedes Beweiswerkzeug eine \cemph{View} (z.B. {\tt SmallStepView})
    \item View implementiert Schnittstelle zur {\it Haupt}-GUI (Kommandos wie {\it Guess}, usw...)
    \item in jeder View liegt eine \cemph{Component} (z.B. {\tt SmallStepComponent})
    \item die Component entscheidet "uber das Layouting der einzelnen Knoten
    \item repr"asentation der einzelnen Knoten erfolgt durch \cemph{NodeComponents} (z.B. {\tt SmallStepNodeComponent})
    \item NodeComponent als {\it Userobject} an \cemph{ProofNode} gebunden
    
  \end{itemgroup}
}

\myslide {Die Renderer - Beweiswerkzeug II} {

  \begin{itemgroup}{Graphischer Abgleich mit Kernkomponenten}
    \item {\tt ProofModel} informiert "uber "Anderungen (hinzuf"ugen, l"oschen ...)
    \item geringer Bearbeitungsbedarf, da {\tt NodeComponent Userobject} 
    \item im {\tt Normalfall} nur Relayout
  \end{itemgroup}

  \animate{1}{
    \begin{itemgroup}{Anweisungen an Kernkomponenten}
      \item {\tt NodeComponent} informatiert "uber Aktivit"at
      \item {\tt Component} leitet Knoten und Kommando (Regelauswahl, Knoten raten... ) an {\tt ProofModel} weiter
      \item {\tt ProofModel} regiert seinerseits mit "Anderungen
    \end{itemgroup}
  }
}

\myslide {Die Renderer - Beweiswerkzeuge III} {

  \begin{itemgroup}{Bigstepper und Typechecker}
    \item echte Baumstruktur
    \item daher graphisch {\it fast} Identisch
    \item Layouting erfolgt rekursive "uber den {\it ProofTree}
    \item Zeichnen der Baumdekoration durch {\tt TreeArrowRenderer}
  \end{itemgroup}

}

\myslide {Die Renderer - Beweiswerkzeuge IV} {

  \graphicxbox{20cm}{images/smallstepnode.png}
  \begin{itemgroup}{Smallstepper}
    \item obwol Darstellung als Liste, interne Darstellung ebenfalls Baum
    \item {\tt SmallStepNodeComponent} hat versetzte Darstellung, deswegen kein einfaches Layouting
    \item Platzierung und wahl der Dimensionen abh"angig von {\it Vater-} und {\it KindKnoten}
  
  
  \end{itemgroup}
 
}




\end{document}

