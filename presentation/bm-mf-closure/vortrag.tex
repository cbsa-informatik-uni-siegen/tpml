\documentclass[fleqn,landscape,titlepage,german]{myslides}
%
% Praesentation der Projektgruppe TPML, 2006
%
% 2006 Benedikt Meurer <benny@xfce.org>
% 2006 Marcell Fischbach <marcellfischbach@gmx.de>
%
 
\usepackage{amssymb}
\usepackage[latin1]{inputenc}
\usepackage{german}
\usepackage{ngerman}

% Einkommentieren, um Handouts zu generieren
%\createhandoutstrue

\begin{document}
  
\presentation[TPML 1.0]{Marcell\,Fischbach,\,Benedikt\,Meurer}{TPML 1.0 - Ein interaktives Lernwerkzeug}
\maketitle

\addoverviewitem{Einleitung}
\addoverviewitem{Die Kernkomponenten}
\addoverviewitem{Die Sprachen}
\addoverviewitem{Die Renderer}
\addoverviewitem{Die Benutzerschnittstelle}
\addoverviewitem{TPML 2.0}
\addoverviewitem{Fazit}

\CenteredGraphics



%%
%% Einleitung
%%

\makeoverviewslide

\myslide{Die Vorlesung ,,Theorie der Programmierung I''}{
  \begin{itemgroup}{,,Theorie der Programmierung I''}
    \item Vermittelt Grundlagen von Programmiersprachen anhand einfacher funktionaler Programmiersprache (orientiert an O'Caml)
    \item Operationelle Semantik anhand von small und big steps
    \item Typsysteme (Polymorphie, Typinferenz)
    \item Pflichtveranstaltung f"ur Informatik-Studenten
  \end{itemgroup}

  \animate{1}{
    {\bf Problem:} Einstieg in den Stoff wird als schwierig empfunden!
  }
}

\myslide{Motivation I}{
  {\bf Frage:} Warum ein Programm zur ,,Theorie der Programmierung''?

  \animate{1}{
    \begin{itemgroup}{Gr"unde}
      \item Erleichterter Zugang zu theoretischen Inhalten.
      \item Regelwerk durch ,,praktisches Arbeiten'' besser zu verstehen.
      \item ,,Gef"uhl'' f"ur Interpreter und Typsysteme vermitteln.
      \item Hilfestellung bei der Pr"ufungsvorbereitung ($\to$ nicht auf Beispiele aus "Ubung/Vorlesung beschr"anken)
      \item Verst"andnis f"ur Zusammenhang von Programmtext und Ausf"uhrung auf einer Maschine.
    \end{itemgroup}
  }
}

\myslide{Motivation II}{
  \begin{itemgroup}{Warum nicht O'Caml?}
    \item ,,Echte Implementierungen'' ungeeignet.
    \item Abl"aufe nicht sichtbar f"ur den Benutzer
    \item Typfehler von O'Caml schwer verst"andlich
    \item Semantik nicht zug"anglich ($\to$ wird in Byte/native Code kompiliert)
    \item Fehlende formale Spezifikation!
  \end{itemgroup}

  \animate{1}{
    {\bf Folgerung:} Spezielles Lernprogramm besser geeignet.
  }
}

\myslide{Ziele}{
  \begin{itemgroup}{Ziele der Entwicklung}
    \item Geeignete Visualierung von Abl"aufen.
    \item Leichter Zugang zum Regelwerk.
    \item Leicht erweiterbar ($\to$ Vorlesung sehr dynamisch)
    \item Plattformunabh"angig einsetzbar.
  \end{itemgroup}
}



\end{document}

