\myslide{Klassen - Übersicht}
{
  \textbf{Aufgabenstellung:}\\[2mm]
  Erweiterung der Sprache \LTWOO\ um Klassen mit Vererbung

  \animate{1}
  {
    \begin{itemgroup}{Lösung:}
      \item Einführung einer neuen Sprache \LTWOC
      \item Erweiterung des Scanners und des Parsers
      \item Implementierung neuer Ausdrücke (Klasse, New und Inherit) und
            Typen (Klassentyp und Attribut-Reihentyp)
      \item Erweiterung der Beweiswerkzeuge Small-Step und Big-Step
    \end{itemgroup}
  }
}

\myslide{Klassen - Erweiterung der Produktionen}
{
  \begin{tabular}{lrp{12.0cm}l}
    e      & ::=    & $\ExprClass{\ExprIdentifier{self}}{\TypeTypeVariable{\tau}}{b\ }$
                    & \mbox{Klasse}\\
           & $\mid$ & $\ExprNew{e}$
                    & \mbox{New}\\[5mm]

    b      & ::=    & $\ExprInherit{\ExprIdentifier{$a_1$},\ \ldots\ ,\ExprIdentifier{$a_k$}}{e}{b}$
                    & \mbox{Inherit}\\
           & $\mid$ & r
                    & \mbox{Reihe}\\[5mm]

    $\tau$ & ::=    & $\TypeClassType{\TypeTypeVariable{\tau}}{\TypeTypeVariable{\phi}}$
                    & \mbox{Klassentyp}\\[5mm]

    $\phi$ & ::=    & $\TypeRowType{{\ExprIdentifier{a}}\colon\ {\TypeTypeVariable{\tau}}\ ;}\ \phi_1$
                    & \mbox{Attribut-Reihentyp}
  \end{tabular}

  {\bf Problem:} Methoden- und Attribut-Reihentyp sind durch den Parser nicht unterscheidbar,
                 da dieser nur einen Identifier erkennt.

  \animate{1}
  {
    {\bf Lösung:} Konkrete Syntax:
                  $\TypeRowType{\KeyAttr\ {\ExprIdentifier{a}}\colon\ {\TypeTypeVariable{\tau_1}}\ ;}
                   \TypeRowType{{\ExprIdentifier{m}}\colon\ {\TypeTypeVariable{\tau_2}}\ ;}\ \phi_1$
  }
}