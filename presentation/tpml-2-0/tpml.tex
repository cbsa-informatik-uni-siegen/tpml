%%
%% Needed latex packages
%%

\usepackage{amsmath}
\usepackage{amssymb}
\usepackage{amstext}
\usepackage{color}
\usepackage{ifthen}
\usepackage{longtable}
\usepackage{pst-node}
\usepackage{pstricks}

%%
%% Needed latex instructions
%%

% ColorExpression: color of expression text
\definecolor{ColorExpression}{rgb}{0.0,0.0,0.0}
% ColorKeyword: color of keywords
\definecolor{ColorKeyword}{rgb}{0.5,0.0,0.0}
% ColorConstant: color of constants
\definecolor{ColorConstant}{rgb}{0.0,0.0,0.5}
% ColorIdentifier: color of identifiers
\definecolor{ColorIdentifier}{rgb}{0.0,0.0,0.4}
% ColorBindingId: color of binding identifiers
\definecolor{ColorBindingId}{rgb}{1.0,0.33,0.1}
% ColorBoundId: color of bound identifiers
\definecolor{ColorBoundId}{rgb}{1.0,0.67,0.2}
% ColorType: color of types
\definecolor{ColorType}{rgb}{0.0,0.6,0.0}
% ColorNone: color of normal text
\definecolor{ColorNone}{rgb}{0.0,0.0,0.0}
% ColorRule: color of proof rules
\definecolor{ColorRule}{rgb}{0.0,0.0,0.0}
\newcounter{tree}
\newcounter{node}[tree]
\newlength{\treeindent}
\newlength{\nodeindent}
\newlength{\nodesep}
\newif\ifarrows
\arrowsfalse
% The environment of the type inference nodes
\newenvironment{typeinferencenode}{\begin{tabular}[t]{p{1.7cm}@{}p{21.8cm}@{}}}{\end{tabular}}
% The environment of the type inference rule
\newenvironment{typeinferencerule}{\begin{tabular}[b]{p{2cm}@{}}}{\end{tabular}}
% The environment of the small step rules with the arrow
\newenvironment{smallsteprulearrow}{\begin{tabular}[t]{p{3.5cm}@{}}}{\end{tabular}}
% The environment of the small step rules
\newenvironment{smallsteprules}{\begin{tabular}{p{2.5cm}@{}}}{\end{tabular}}
% The environment of the small step nodes
\newenvironment{smallstepnode}{\begin{tabular}[b]{p{22cm}@{}}}{\end{tabular}}

%%
%% Needed latex commands
%%

% KeyAmperAmper
\newcommand{\KeyAmperAmper}{\textbf{\color{ColorKeyword}{\&\&}}}
% KeyAttr
\newcommand{\KeyAttr}{\textbf{\color{ColorKeyword}{attr}}}
% KeyBarBar
\newcommand{\KeyBarBar}{\textbf{\color{ColorKeyword}{$\|$}}}
% KeyBool
\newcommand{\KeyBool}{\textbf{\color{ColorType}{bool}}}
% KeyClass
\newcommand{\KeyClass}{\textbf{\color{ColorKeyword}{class}}}
% KeyDo
\newcommand{\KeyDo}{\textbf{\color{ColorKeyword}{do}}}
% KeyElse
\newcommand{\KeyElse}{\textbf{\color{ColorKeyword}{else}}}
% KeyEnd
\newcommand{\KeyEnd}{\textbf{\color{ColorKeyword}{end}}}
% KeyFrom
\newcommand{\KeyFrom}{\textbf{\color{ColorKeyword}{from}}}
% KeyIf
\newcommand{\KeyIf}{\textbf{\color{ColorKeyword}{if}}}
% KeyIn
\newcommand{\KeyIn}{\textbf{\color{ColorKeyword}{in}}}
% KeyInherit
\newcommand{\KeyInherit}{\textbf{\color{ColorKeyword}{inherit}}}
% KeyInt
\newcommand{\KeyInt}{\textbf{\color{ColorType}{int}}}
% KeyLambda
\newcommand{\KeyLambda}{\textbf{\color{ColorKeyword}{$\lambda$}}}
% KeyLet
\newcommand{\KeyLet}{\textbf{\color{ColorKeyword}{let}}}
% KeyList
\newcommand{\KeyList}{\textbf{\color{ColorType}{list}}}
% KeyMethod
\newcommand{\KeyMethod}{\textbf{\color{ColorKeyword}{method}}}
% KeyMu
\newcommand{\KeyMu}{\textbf{\color{ColorKeyword}{$\mu$}}}
% KeyNew
\newcommand{\KeyNew}{\textbf{\color{ColorKeyword}{new}}}
% KeyObject
\newcommand{\KeyObject}{\textbf{\color{ColorKeyword}{object}}}
% KeyRec
\newcommand{\KeyRec}{\textbf{\color{ColorKeyword}{rec}}}
% KeyRef
\newcommand{\KeyRef}{\textbf{\color{ColorType}{ref}}}
% KeySolve
\newcommand{\KeySolve}{\textbf{\color{ColorKeyword}{solve}}}
% KeyThen
\newcommand{\KeyThen}{\textbf{\color{ColorKeyword}{then}}}
% KeyUnify
\newcommand{\KeyUnify}{\textbf{\color{ColorType}{unify}}}
% KeyUnit
\newcommand{\KeyUnit}{\textbf{\color{ColorType}{unit}}}
% KeyVal
\newcommand{\KeyVal}{\textbf{\color{ColorKeyword}{val}}}
% KeyWhile
\newcommand{\KeyWhile}{\textbf{\color{ColorKeyword}{while}}}
% KeyZeta
\newcommand{\KeyZeta}{\textbf{\color{ColorKeyword}{$\zeta$}}}
% BigStepProofNode{depth}{id}{e}{store}{result}{proofrule}{space}
\newcommand{\BigStepProofNode}[7]{
             \ifarrows
             \else \refstepcounter{node}
             \noindent\hspace{\treeindent}\hspace{#2\nodeindent}
             \rnode{\thetree.#1}{\makebox[6mm]{(\thenode)}}\label{\thetree.#1}
             $\begin{tabular}[t]{p{#7}}$
               \ifthenelse{\equal{#4}{}}
                 {#3\ \color{ColorNone}{\Downarrow}\ #5}
                 {\color{ColorNone}{(}#3\ \ #4\color{ColorNone}{)}\ \color{ColorNone}{\Downarrow}\ #5}
             $\\$
               \byrule{#6}
             $\end{tabular}$
             \vspace{\nodesep}
             \fi}
% BigStepProofResult{body}
\newcommand{\BigStepProofResult}[1]{#1}
% BigStepProofRule{name}
\newcommand{\BigStepProofRule}[1]{\mbox{\textbf{\color{ColorRule}(#1)}}}
% ExprAnd{e1}{e2}
\newcommand{\ExprAnd}[2]{\color{ColorExpression}#1\ \KeyAmperAmper\ #2}
% ExprApplication{e1}{e2}
\newcommand{\ExprApplication}[2]{\color{ColorExpression}#1\ #2}
% ExprAttribute{a}{e}
\newcommand{\ExprAttribute}[2]{\color{ColorExpression}\KeyVal\ #1\ =\ #2\ ;}
% ExprBinaryOperator{op}
\newcommand{\ExprBinaryOperator}[1]{\textbf{\color{ColorConstant}{$#1$}}}
% ExprClass{self}{tau}{b}
\newcommand{\ExprClass}[3]{\ifthenelse{\equal{#2}{}}
             {\color{ColorExpression}\KeyClass\ (#1)\ #3\ \KeyEnd}
             {\color{ColorExpression}\KeyClass\ (#1\colon\ #2)\ #3\KeyEnd}}
% ExprCoercion{e}{tau1}{tau2}
\newcommand{\ExprCoercion}[3]{\color{ColorExpression}(#1\colon\ #2\ <\colon\ #3)}
% ExprCondition{e0}{e1}{e2}
\newcommand{\ExprCondition}[3]{\color{ColorExpression}\KeyIf\ #1\ \KeyThen\ #2\ \KeyElse\ #3}
% ExprConditionOne{e0}{e1}
\newcommand{\ExprConditionOne}[2]{\color{ColorExpression}\KeyIf\ #1\ \KeyThen\ #2}
% ExprConstant{c}
\newcommand{\ExprConstant}[1]{\mbox{\textbf{\color{ColorConstant}{#1}}}}
% ExprCurriedLet{id (id1: tau1) ... (idn: taun)}{tau}{e1}{e2}
\newcommand{\ExprCurriedLet}[4]{\ifthenelse{\equal{#2}{}}
             {\color{ColorExpression}\KeyLet\ #1\ =\ #3\ \KeyIn\ #4}
             {\color{ColorExpression}\KeyLet\ #1\colon\ #2\ =\ #3\ \KeyIn\ #4}}
% ExprCurriedLetRec{id (id1: tau1) ... (idn: taun)}{tau}{e1}{e2}
\newcommand{\ExprCurriedLetRec}[4]{\ifthenelse{\equal{#2}{}}
             {\color{ColorExpression}\KeyLet\ \KeyRec\ #1\ =\ #3\ \KeyIn\ #4}
             {\color{ColorExpression}\KeyLet\ \KeyRec\ #1\colon\ #2\ =\ #3\ \KeyIn\ #4}}
% ExprCurriedMethod{m (id1: tau1) ... (idn: taun)}{tau}{e}
\newcommand{\ExprCurriedMethod}[3]{\ifthenelse{\equal{#2}{}}
             {\color{ColorExpression}\KeyMethod\ #1\ =\ #3\ ;}
             {\color{ColorExpression}\KeyMethod\ #1\colon\ #2\ =\ #3\ ;}}
% ExprDuplication{a1 = e1 ; ... ; an = en}
\newcommand{\ExprDuplication}[1]{\color{ColorExpression}\{<#1>\}}
% ExprExn{name}
\newcommand{\ExprExn}[1]{\mbox{\color{ColorExpression}{$\uparrow$\ #1}}}
% ExprIdentifier{id}
\newcommand{\ExprIdentifier}[1]{\mbox{\color{ColorIdentifier}{#1}}}
% ExprIdentifierBinding{id}
\newcommand{\ExprIdentifierBinding}[1]{\mbox{\textbf{\color{ColorBindingId}{#1}}}}
% ExprIdentifierBound{id}
\newcommand{\ExprIdentifierBound}[1]{\mbox{\textbf{\color{ColorBoundId}{#1}}}}
% ExprInfixOperation{op}{e1}{e2}
\newcommand{\ExprInfixOperation}[3]{\color{ColorExpression}#2\ #1\ #3}
% ExprInherit{a1, ... , ak}{e}{b}
\newcommand{\ExprInherit}[3]{\color{ColorExpression}\KeyInherit\ #1\ \KeyFrom\ #2\ ;\ #3}
% ExprLambda{id}{tau}{e}
\newcommand{\ExprLambda}[3]{\ifthenelse{\equal{#2}{}}
             {\color{ColorExpression}\KeyLambda#1.#3}
             {\color{ColorExpression}\KeyLambda#1\colon\ #2.#3}}
% ExprLet{id}{tau}{e1}{e2}
\newcommand{\ExprLet}[4]{\ifthenelse{\equal{#2}{}}
             {\color{ColorExpression}\KeyLet\ #1\ =\ #3\ \KeyIn\ #4}
             {\color{ColorExpression}\KeyLet\ #1\colon\ #2\ =\ #3\ \KeyIn\ #4}}
% ExprLetRec{id}{tau}{e1}{e2}
\newcommand{\ExprLetRec}[4]{\ifthenelse{\equal{#2}{}}
             {\color{ColorExpression}\KeyLet\ \KeyRec\ #1\ =\ #3\ \KeyIn\ #4}
             {\color{ColorExpression}\KeyLet\ \KeyRec\ #1\colon\ #2\ =\ #3\ \KeyIn\ #4}}
% ExprList{e1; ... ; en}
\newcommand{\ExprList}[1]{\color{ColorExpression}{[#1]}}
% ExprLocation{name}
\newcommand{\ExprLocation}[1]{\mbox{\color{ColorExpression}{#1}}}
% ExprMethod{m}{tau}{e}
\newcommand{\ExprMethod}[3]{\ifthenelse{\equal{#2}{}}
             {\color{ColorExpression}\KeyMethod\ #1\ =\ #3\ ;}
             {\color{ColorExpression}\KeyMethod\ #1\colon\ #2\ =\ #3\ ;}}
% ExprMultiLambda{id1, ..., idn}{tau}{e}
\newcommand{\ExprMultiLambda}[3]{\ifthenelse{\equal{#2}{}}
             {\color{ColorExpression}\KeyLambda(#1).#3}
             {\color{ColorExpression}\KeyLambda(#1)\colon\ #2.#3}}
% ExprMultiLet{id1, ..., idn}{tau}{e1}{e2}
\newcommand{\ExprMultiLet}[4]{\ifthenelse{\equal{#2}{}}
             {\color{ColorExpression}\KeyLet\ (#1)\ =\ #3\ \KeyIn\ #4}
             {\color{ColorExpression}\KeyLet\ (#1)\colon\ #2\ =\ #3\ \KeyIn\ #4}}
% ExprNew{e}
\newcommand{\ExprNew}[1]{\color{ColorExpression}\KeyNew\ #1}
% ExprObject{self}{tau}{r}
\newcommand{\ExprObject}[3]{\ifthenelse{\equal{#2}{}}
             {\color{ColorExpression}\KeyObject\ (#1)\ #3\ \KeyEnd}
             {\color{ColorExpression}\KeyObject\ (#1\colon\ #2)\ #3\ \KeyEnd}}
% ExprOr{e1}{e2}
\newcommand{\ExprOr}[2]{\color{ColorExpression}#1\ \KeyBarBar\ #2}
% ExprRecursion{id}{tau}{e}
\newcommand{\ExprRecursion}[3]{\ifthenelse{\equal{#2}{}}
             {\color{ColorExpression}\KeyRec\ #1.#3}
             {\color{ColorExpression}\KeyRec\ #1\colon\ #2.#3}}
% ExprRow{epsilon | val a = e; r1 | method m : τ = e ; r1}
\newcommand{\ExprRow}[1]{\color{ColorExpression}#1}
% ExprSend{e}{m}
\newcommand{\ExprSend}[2]{\color{ColorExpression}#1\ \#\ #2}
% ExprSequence{e1}{e2}
\newcommand{\ExprSequence}[2]{\color{ColorExpression}#1;\ #2}
% ExprTuple{e1, ... , en}
\newcommand{\ExprTuple}[1]{\color{ColorExpression}(#1)}
% ExprWhile{e1}{e2}
\newcommand{\ExprWhile}[2]{\color{ColorExpression}\KeyWhile\ #1\ \KeyDo\ #2}
% MinimalTypingExpressionProofNode{depth}{id}{evironment}{expression}{type}{rule}{space}
\newcommand{\MinimalTypingExpressionProofNode}[7]{
             \ifarrows
             \else \refstepcounter{node}
             \noindent\hspace{\treeindent}\hspace{#2\nodeindent}
             \rnode{\thetree.#1}{\makebox[6mm]{(\thenode)}}\label{\thetree.#1}
             $\begin{tabular}[t]{p{#7}}$
               #3\ \color{ColorNone}{\vartriangleright}\ #4\ \color{ColorNone}{::}\ #5
             $\\$
               \byrule{#6}
             $\end{tabular}$
             \vspace{\nodesep}
             \fi}
% MinimalTypingProofRule{name}
\newcommand{\MinimalTypingProofRule}[1]{\mbox{\textbf{\color{ColorRule}(#1)}}}
% MinimalTypingTypesProofNode{depth}{id}{subtype}{rule}{space}
\newcommand{\MinimalTypingTypesProofNode}[5]{
             \ifarrows
             \else \refstepcounter{node}
             \noindent\hspace{\treeindent}\hspace{#2\nodeindent}
             \rnode{\thetree.#1}{\makebox[6mm]{(\thenode)}}\label{\thetree.#1}
             $\begin{tabular}[t]{p{#5}}$
               #3
             $\\$
               \byrule{#4}
             $\end{tabular}$
             \vspace{\nodesep}
             \fi}
% Parenthesis{e}
\newcommand{\Parenthesis}[1]{(#1)}
% RecSubTypingProofNode{depth}{id}{seenTypes}{type}{type2}{rule}{space}
\newcommand{\RecSubTypingProofNode}[7]{
             \ifarrows
             \else \refstepcounter{node}
             \noindent\hspace{\treeindent}\hspace{#2\nodeindent}
             \rnode{\thetree.#1}{\makebox[6mm]{(\thenode)}}\label{\thetree.#1}
             $\begin{tabular}[t]{p{#7}}$
               #3\ \color{ColorNone}{\vdash}\ #4\ \color{ColorNone}{<:}\ #5
             $\\$
               \byrule{#6}
             $\end{tabular}$
             \vspace{\nodesep}
             \fi}
% RecSubTypingProofRule{name}
\newcommand{\RecSubTypingProofRule}[1]{\mbox{\textbf{\color{ColorRule}(#1)}}}
% SeenTypes{E1, ... , En}
\newcommand{\SeenTypes}[1]{\color{ColorNone}{\{}#1\color{ColorNone}{\}}}
% SmallStepArrow{not axiom rules}{axiom rules}
\newcommand{\SmallStepArrow}[2]{\xrightarrow
             [\mbox{\color{ColorRule}{\scriptsize{#2}}}]
             {\mbox{\color{ColorRule}{\scriptsize{#1}}}}}
% SmallStepNewNode
\newcommand{\SmallStepNewNode}{\\[10mm]}
% SmallStepNewRule
\newcommand{\SmallStepNewRule}{\\}
% SmallStepProofModel{model}
\newcommand{\SmallStepProofModel}[1]{\begin{longtable}{p{3.5cm}@{}p{22cm}@{}}#1\end{longtable}}
% SmallStepProofNode{e}{store}
\newcommand{\SmallStepProofNode}[2]{\ifthenelse{\equal{#2}{}}
             {#1}
             {\color{ColorNone}{(}#1\ \ #2\color{ColorNone}{)}}}
% SmallStepProofRule{name}
\newcommand{\SmallStepProofRule}[1]{\mbox{\textbf{\color{ColorRule}(#1)}}}
% SmallStepRulesCompleted
\newcommand{\SmallStepRulesCompleted}{&}
% SolveLeftParen
\newcommand{\SolveLeftParen}{\textbf{\color{ColorNone}{\ \{}}}
% SolveRightParen
\newcommand{\SolveRightParen}{\textbf{\color{ColorNone}{\ \}}}}
% Store{X1: e1, ..., Xn: en}
\newcommand{\Store}[1]{\color{ColorNone}{[}#1\color{ColorNone}{]}}
% SubType{tau1}{tau2}
\newcommand{\SubType}[2]{#1\ \color{ColorNone}{<:}\ #2}
% SubTypingProofNode{depth}{id}{type}{type2}{rule}{space}
\newcommand{\SubTypingProofNode}[6]{
             \ifarrows
             \else \refstepcounter{node}
             \noindent\hspace{\treeindent}\hspace{#2\nodeindent}
             \rnode{\thetree.#1}{\makebox[6mm]{(\thenode)}}\label{\thetree.#1}
             $\begin{tabular}[t]{p{#6}}$
               #3\ \color{ColorNone}{<:}\ #4
             $\\$
               \byrule{#5}
             $\end{tabular}$
             \vspace{\nodesep}
             \fi}
% SubTypingProofRule{name}
\newcommand{\SubTypingProofRule}[1]{\mbox{\textbf{\color{ColorRule}(#1)}}}
% TypeArrowType{tau1}{tau2}
\newcommand{\TypeArrowType}[2]{\color{ColorExpression}#1\ \to\ #2}
% TypeBooleanType
\newcommand{\TypeBooleanType}{\KeyBool}
% TypeCheckerExpressionProofNode{depth}{id}{evironment}{expression}{type}{rule}{space}
\newcommand{\TypeCheckerExpressionProofNode}[7]{
             \ifarrows
             \else \refstepcounter{node}
             \noindent\hspace{\treeindent}\hspace{#2\nodeindent}
             \rnode{\thetree.#1}{\makebox[6mm]{(\thenode)}}\label{\thetree.#1}
             $\begin{tabular}[t]{p{#7}}$
               #3\ \color{ColorNone}{\vartriangleright}\ #4\ \color{ColorNone}{::}\ #5
             $\\$
               \byrule{#6}
             $\end{tabular}$
             \vspace{\nodesep}
             \fi}
% TypeCheckerProofRule{name}
\newcommand{\TypeCheckerProofRule}[1]{\mbox{\textbf{\color{ColorRule}(#1)}}}
% TypeCheckerTypeProofNode{depth}{id}{type}{type2}{rule}{space}
\newcommand{\TypeCheckerTypeProofNode}[6]{
             \ifarrows
             \else \refstepcounter{node}
             \noindent\hspace{\treeindent}\hspace{#2\nodeindent}
             \rnode{\thetree.#1}{\makebox[6mm]{(\thenode)}}\label{\thetree.#1}
             $\begin{tabular}[t]{p{#6}}$
               #3\ \color{ColorNone}{<:}\ #4
             $\\$
               \byrule{#5}
             $\end{tabular}$
             \vspace{\nodesep}
             \fi}
% TypeClassType{tau}{phi}
\newcommand{\TypeClassType}[2]{\color{ColorExpression}\KeyZeta(#1\ \colon\ #2)}
% TypeEnvironment{id1: tau1, ..., idn: taun}
\newcommand{\TypeEnvironment}[1]{\color{ColorNone}{[}#1\color{ColorNone}{]}}
% TypeEquationListTypeChecker{teqn1, ... , teqnn}
\newcommand{\TypeEquationListTypeChecker}[1]{\color{ColorNone}{\{}#1\color{ColorNone}{\}}}
% TypeEquationListTypeInference{teqn1, ... , teqnn}
\newcommand{\TypeEquationListTypeInference}[1]{\color{ColorNone}{\{}#1\color{ColorNone}{\}}}
% TypeEquationTypeChecker{tau1}{tau2}
\newcommand{\TypeEquationTypeChecker}[2]{#1\ \color{ColorNone}{=}\ #2}
% TypeEquationTypeInference{tau1}{tau2}
\newcommand{\TypeEquationTypeInference}[2]{#1\ \color{ColorNone}{=}\ #2}
% TypeInferenceEqual
\newcommand{\TypeInferenceEqual}{\mbox{\centerline{\LARGE=}}}
% TypeInferenceNewFormula
\newcommand{\TypeInferenceNewFormula}{\\}
% TypeInferenceNewNode
\newcommand{\TypeInferenceNewNode}{\\[10mm]}
% TypeInferenceProofModel{model}
\newcommand{\TypeInferenceProofModel}[1]{\begin{longtable}{p{2cm}@{}p{23.5cm}@{}}#1\end{longtable}}
% TypeInferenceProofNode{body}
\newcommand{\TypeInferenceProofNode}[1]{#1}
% TypeInferenceRule{rule}
\newcommand{\TypeInferenceRule}[1]{\mbox{\centerline{\scriptsize{#1}}}}
% TypeInferenceRulesCompleted
\newcommand{\TypeInferenceRulesCompleted}{&}
% TypeInferenceSubstitutionBegin
\newcommand{\TypeInferenceSubstitutionBegin}{\multicolumn{2}{p{23.5cm}}}
% TypeIntegerType
\newcommand{\TypeIntegerType}{\KeyInt}
% TypeJudgement{env}{e}{tau}
\newcommand{\TypeJudgement}[3]{#1\ \color{ColorNone}{\vartriangleright}\ #2\ \color{ColorNone}{::}\ #3}
% TypeListType{tau}
\newcommand{\TypeListType}[1]{\color{ColorExpression}#1\ \KeyList}
% TypeObjectType{phi}
\newcommand{\TypeObjectType}[1]{\color{ColorExpression}<\ #1\ >}
% TypePolyType{forall tvar1, ..., tvarn}{tau}
\newcommand{\TypePolyType}[2]{\ifthenelse{\equal{#1}{}}
             {\color{ColorExpression}#2}
             {\color{ColorExpression}#1.#2}}
% TypeRecType{t}{tau}
\newcommand{\TypeRecType}[2]{\color{ColorExpression}\KeyMu#1.#2}
% TypeRefType{tau}
\newcommand{\TypeRefType}[1]{\color{ColorExpression}#1\ \KeyRef}
% TypeRowType{epsilon | attr a : tau ; phi1 | m : tau ; phi1}
\newcommand{\TypeRowType}[1]{\color{ColorExpression}#1}
% TypeSubType{tau1}{tau2}
\newcommand{\TypeSubType}[2]{#1\ \color{ColorNone}{<:}\ #2}
% TypeSubstitution{tau}{tvar}
\newcommand{\TypeSubstitution}[2]{#1\color{ColorNone}{/}#2}
% TypeSubstitutionList{tsub1, ... , tsubn}
\newcommand{\TypeSubstitutionList}[1]{\color{ColorNone}{\{}#1\color{ColorNone}{\}}}
% TypeTupleType{tau1 * ... * taun}
\newcommand{\TypeTupleType}[1]{\color{ColorExpression}#1}
% TypeTypeName{t}
\newcommand{\TypeTypeName}[1]{\mbox{\color{ColorIdentifier}{#1}}}
% TypeTypeNameBinding{id}
\newcommand{\TypeTypeNameBinding}[1]{\mbox{\textbf{\color{ColorBindingId}{#1}}}}
% TypeTypeNameBound{id}
\newcommand{\TypeTypeNameBound}[1]{\mbox{\textbf{\color{ColorBoundId}{#1}}}}
% TypeTypeVariable{tvar}
\newcommand{\TypeTypeVariable}[1]{\textbf{\color{ColorType}{$#1$}}}
% TypeUnifyType
\newcommand{\TypeUnifyType}{\KeyUnify}
% TypeUnitType
\newcommand{\TypeUnitType}{\KeyUnit}
% arrow{angle}{from}{to}
\newcommand{\arrow}[3]{\ifarrows
             \ncangle[angleA=-90,angleB=#1]{<-}{\thetree.#2}{\thetree.#3}
             \else
             \fi}
% byrule{rule}
\newcommand{\byrule}[1]{\hspace{-5mm}\mbox{\scriptsize\ #1}}
% mktree{tree}
\newcommand{\mktree}[1]{\stepcounter{tree} #1 \arrowstrue #1 \arrowsfalse}
