\myslide{Minimal Typing}
{
  \textbf{Minimal Typing ist ein Beweiswerkzeug, um den Typ für einen Ausdruck zu bestimmen} \\[6mm]
  \begin{itemgroup}{Unterschiede zu Type Checker und Type Inference}
    \item Es gibt keine Typvariablen mehr
      \catchword {Es werden keine Typen mehr geraten}
    \item Es muss ein Typ für die Identifier angegeben werden
      \catchword $\TypeEnvironment{}\ {\vartriangleright}\ {\ExprLet{\ExprIdentifier{x}}{}
         {\ExprInfixOperation{\ExprBinaryOperator{+}}{\ExprConstant{5}}{\ExprConstant{5}}}
         {\ExprInfixOperation{\ExprBinaryOperator{+}}{\ExprIdentifier{x}}{\ExprConstant{2}}}}$
      $\to$ {\color{red}{falsch}}
      \catchword $\TypeEnvironment{}\ {\vartriangleright}\ {\ExprLet{\ExprIdentifier{x}}{\TypeIntegerType}
         {\ExprInfixOperation{\ExprBinaryOperator{+}}{\ExprConstant{5}}{\ExprConstant{5}}}
         {\ExprInfixOperation{\ExprBinaryOperator{+}}{\ExprIdentifier{x}}{\ExprConstant{2}}}}$
      $\to$ {\color{green}{richtig}}
    \item Es wird überprüft ob der angegebene Typ und der berechnete Typ übereinstimmt
      \catchword{Subtyprelation}
    
  \end{itemgroup}
}