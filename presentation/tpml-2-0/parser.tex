\myslide{Parser-Autovervollst"andigung}
{
  \begin{itemgroup}{"Ubersicht}
    \item Die Parser waren sehr un"ubersichtlich strukturiert
    \item Pro Ausdruck und Typ wurde ein eigenes \glqq non terminal\grqq eingef"uhrt
    \item Zus"atzlich ein eigenes \glqq error non terminal\grqq, welches sich um
          die eingef"uhrte Fehlerbehandlung k"ummert
    \item Jedem geparsten Ausdruck wird seine Source Code Position mitgegeben, um
          Fehler anzeigen zu k"onnen und, um den Source Code durch die Outline
          markieren zu lassen
    \item Eine automatische Vervollst"andigung wurde eingef"uhrt, um vom Benutzer
          eingegebene, noch unvollst"andige Ausdr"ucke zu vervollst"andigen
    \item Fehlermeldungen die mehrere Stellen betreffen wurden eingef"uhrt
  \end{itemgroup}
}

\myslide{Parser-Autovervollst"andigung}
{
  {\bf Fehlermeldungen}\\[5mm]
  Gibt der Anwender \glqq \ExprInfixOperation{+}{1}{\KeyLet\ \ExprIdentifier{x} =}\grqq\ 
  ein, wird der Teil \glqq \KeyLet\ \ExprIdentifier{x} =\grqq\ hervorgehoben und der Tooltip
  verr"at dem Anwender, dass er noch den Ausdruck \glqq $e_1$\ \KeyIn\ $e_2$\grqq\ eingeben muss,
  um \glqq{\bf Let}\grqq\ zu vervollst"andigen.
}