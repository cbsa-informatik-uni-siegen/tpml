\documentclass[fleqn,landscape,titlepage,german]{myslides}
%
% Praesentation der Projektgruppe TPML 2.0, 2007
%
% 2007 Christian Fehler
% 2007 Christoph Fehling
% 2007 Benjamin Mies
% 2007 Michael Oeste
%

\usepackage{amssymb}
\usepackage[latin1]{inputenc}
\usepackage{german}
\usepackage{ngerman}

% TP Latex Macros
% packages
\usepackage{color}
\usepackage{ifthen}

% tpml
\newcommand{\TPML}{\textsf{\textmd{TPML}}}
\newcommand{\TPMLTWOZERO}{\textsf{\textmd{TPML 2.0}}}

% tp
\newcommand{\TPONE}{Theorie der Programmierung I}
\newcommand{\TPTWO}{Theorie der Programmierung II}

% languages
\newcommand{\LZEROCBN}{$\mathcal{L}_0^{CBN}$}
\newcommand{\LONECBN}{$\mathcal{L}_1^{CBN}$}
\newcommand{\LTWOCBN}{$\mathcal{L}_2^{CBN}$}
\newcommand{\LTWOO}{$\mathcal{L}_2^{O}$}
\newcommand{\LTWOC}{$\mathcal{L}_2^{C}$}

% color
\definecolor{ColorExpression}{rgb}{0.0,0.0,0.0}
\definecolor{ColorKeyword}{rgb}{0.5,0.0,0.0}
\definecolor{ColorConstant}{rgb}{0.0,0.0,0.5}
\definecolor{ColorIdentifier}{rgb}{0.0,0.0,0.4}
\definecolor{ColorType}{rgb}{0.0,0.6,0.0}
\definecolor{ColorNone}{rgb}{0.0,0.0,0.0}
\definecolor{ColorRule}{rgb}{0.0,0.0,0.0}

% big step
\newcommand{\regel}[2]{\begin{array}{@{}c@{}} #1 \\ \hline #2 \end{array}}
\newcommand{\eval}{\Downarrow}

% keyword
\newcommand{\KeyLet}{\textbf{\color{ColorKeyword}{let}}}
\newcommand{\KeyIn}{\textbf{\color{ColorKeyword}{in}}}
\newcommand{\KeyLambda}{\textbf{\color{ColorKeyword}{$\lambda$}}}

% identifier
\newcommand{\id}{\ExprIdentifier{id}}

% expression
\newcommand{\ExprApplication}[2]{\color{ColorExpression}#1\ #2}
\newcommand{\ExprConstant}[1]{\mbox{\textbf{\color{ColorConstant}{#1}}}}
\newcommand{\ExprExn}[1]{\mbox{\color{ColorExpression}{$\uparrow$\ #1}}}
\newcommand{\ExprLambda}[3]{\ifthenelse{\equal{#2}{}}
             {\color{ColorExpression}\KeyLambda#1.#3}
             {\color{ColorExpression}\KeyLambda#1\colon\ #2.#3}}
\newcommand{\ExprLet}[4]{\ifthenelse{\equal{#2}{}}
             {\color{ColorExpression}\KeyLet\ #1\ =\ #3\ \KeyIn\ #4}
             {\color{ColorExpression}\KeyLet\ #1\colon\ #2\ =\ #3\ \KeyIn\ #4}}
\newcommand{\ExprIdentifier}[1]{\mbox{\color{ColorIdentifier}{#1}}}
\newcommand{\ExprInfixOperation}[3]{\color{ColorExpression}#2\ #1\ #3}
\newcommand{\Parenthesis}[1]{(#1)}

% Einkommentieren, um Handouts zu generieren
%\createhandoutstrue

\begin{document}

\presentation[TPML 2.0] {Christian\ Fehler,\ Christoph\ Fehling,\ Benjamin\ Mies,\ Michael\ Oeste}
                        {TPML 2.0 - Ein interaktives Lernwerkzeug}

\maketitle

\addoverviewitem{Einleitung}
\addoverviewitem{Benutzerfreundlichkeit: Outline, Parser-Autovervollst"andigung}
\addoverviewitem{Weitere Sprachen: Call by Name, Objekte und Klassen}
\addoverviewitem{\LaTeX-Export}
\addoverviewitem{Beweise: Type Inference, Minimal Typing und Subtyping}
\addoverviewitem{Fazit}

\CenteredGraphics

%%
%% Einleitung
%%
\makeoverviewslide
%%
%% Einleitung
%%

\chapter{Einleitung}

Narf, \cite[50f]{Remy02}.


% vi:set ts=2 sw=2 et ai syntax=tex:


%%
%% Outline
%%
\makeoverviewslide
\myslide{Outline}
{
  \begin{itemgroup}{Outline}
    \item Blablabla ...
  \end{itemgroup}
}

%%
%% Parser
%%
\myslide{Parser-Autovervollst"andigung}
{
  \begin{itemgroup}{"Ubersicht}
    \item Die Parser waren sehr un"ubersichtlich strukturiert
    \item Pro Ausdruck und Typ wurde ein eigenes \glqq non terminal\grqq eingef"uhrt
    \item Zus"atzlich ein eigenes \glqq error non terminal\grqq, welches sich um
          die eingef"uhrte Fehlerbehandlung k"ummert
    \item Jedem geparsten Ausdruck wird seine Source Code Position mitgegeben, um
          Fehler anzeigen zu k"onnen und, um den Source Code durch die Outline
          markieren zu lassen
    \item Eine automatische Vervollst"andigung wurde eingef"uhrt, um vom Benutzer
          eingegebene, noch unvollst"andige Ausdr"ucke zu vervollst"andigen
    \item Fehlermeldungen die mehrere Stellen betreffen wurden eingef"uhrt
  \end{itemgroup}
}

\myslide{Parser-Autovervollst"andigung}
{
  {\bf Fehlermeldungen}\\[5mm]
  Gibt der Anwender \glqq \ExprInfixOperation{+}{1}{\KeyLet\ \ExprIdentifier{x} =}\grqq\ 
  ein, wird der Teil \glqq \KeyLet\ \ExprIdentifier{x} =\grqq\ hervorgehoben und der Tooltip
  verr"at dem Anwender, dass er noch den Ausdruck \glqq $e_1$\ \KeyIn\ $e_2$\grqq\ eingeben muss,
  um \glqq{\bf Let}\grqq\ zu vervollst"andigen.
}

%%
%% Call by Name
%%
\makeoverviewslide
\myslide{Call by Name I}
{
  \begin{itemgroup}{Call by Name}
    \item Sprachen \LZEROCBN, \LONECBN und \LTWOCBN  mit Call by Name Semantik
    \item Ausdr"ucke werden nicht erst ausgewertet, sondern direkt eingesetzt,
          dadurch kann es zu unterschiedlichen Ergebnissen kommen
    \item "Anderung der Small Step Semantik
    \item "Anderung der Big Step Semantik
    \item Einf"uhrung eigener Dateiendungen
    \item Relevant f"ur die Pr"ufung \glqq \TPONE \grqq
  \end{itemgroup}
}


\myslide{Call by Name II}
{
  {\bf "Anderungen:} Small Step Semantik\\[5mm]
  \begin{tabular}{ll}
     \mbox{(BETA-V)}      & nicht vorhanden \\[3mm]
     \mbox{(BETA)}        & $(\abstr{id}{e_1})\,e_2 \to e_1[e_2/\id]$ \\[3mm]
     \mbox{(APP-RIGHT)\ } & $\regel{e \to e'}
                                   {v\,e \to v\,e'}$ \ 
                                   falls ${v}$ nicht von der Form $\abstr{id}{e_0}$ \\[5mm]
     \mbox{(LET-EVAL)\  } & nicht vorhanden \\[3mm]
     \mbox{(LET-EXEC)}    & $\bli{\id}{e_1}{e_2} \to e_2[e_1/\id]$ \\[3mm]
  \end{tabular}
}


\myslide{Call by Name III}
{
  {\bf "Anderungen:} Big Step Semantik\\[5mm]
  \begin{tabular}{ll}
     \mbox{(BETA-V)}      & nicht vorhanden \\[3mm]
     \mbox{(BETA)}        & $\regel{e_1[e_2/\id] \eval v}
                                   {(\abstr{id}{e_1})\,e_2 \eval v}$ \\[5mm]
     \mbox{(APP)}         & nicht vorhanden \\[3mm]
     \mbox{(APP-LEFT)}    & $\regel{e_1 \eval v_1 \quad v_1\,e_2 \eval v}
                                   {e_1\,e_2 \eval v}$ \\[5mm]
     \mbox{(APP-RIGHT)}   & $\regel{e_2 \eval v_2 \quad v_1\,v_2 \eval v}
                                   {v_1\,e_2 \eval v}$ \ 
                                   falls ${v_1}$ nicht von der Form $\abstr{id}{e}$ \\[5mm]
     \mbox{(LET)}         & $\regel{e_2[e_1/\id] \eval v}
                                   {\bli{\id}{e_1}{e_2} \eval v}$ \\[5mm]
  \end{tabular}
}


\myslide{Call by Name IV}
{
  {\bf Beispiel:} Unterschied von \glqq Call by Name\grqq\ zu \glqq Call by Value\grqq\\[5mm]
  \begin{tabular}{ll}
     Call by Value $\quad$      & $\bli{x}{\infix{/}{1}{0}}{2} \ \ \to \ \ divide\_by\_zero$ \\[3mm]
     Call by Name $\quad$       & $\bli{x}{\infix{/}{1}{0}}{2} \ \ \to \ \ 2$ \\[3mm]
  \end{tabular}
}

%%
%% Objekte
%%
\myslide{Objekte}
{
  \begin{itemgroup}{"Ubersicht}
    \item Blablabla ...
  \end{itemgroup}
}

%%
%% Klassen
%%
\myslide{Klassen - Übersicht}
{
  \begin{itemgroup}{}
    \item Blablabla ...
  \end{itemgroup}
}

\myslide{Klassen - Erweiterung der Produktionen}
{
  \bproduction
    e \is \ExprClass{\ExprIdentifier{self}}{\TypeTypeVariable{\tau}}{b\ }   & & \mbox{Klasse}
       \al \ExprNew{e}   & & \mbox{New}\\

    b \is \ExprInherit{\ExprIdentifier{$a_1$},\ \ldots\ ,\ExprIdentifier{$a_k$}}{e}{b}   & & \mbox{Inherit}
       \al r   & & \mbox{Reihe}\\

    \tau \is \TypeClassType{\TypeTypeVariable{\tau}}{\TypeTypeVariable{\phi}}
             & & \mbox{Klassentyp}\\

    \phi \is \TypeRowType{{\ExprIdentifier{a}}\colon\ {\TypeTypeVariable{\tau}}\ ;}\ \phi_1
             & & \mbox{Attribut-Reihentyp}
  \eproduction

  {\bf Problem:} Methoden- und Attribut-Reihentyp sind durch den Parser nicht unterscheidbar,
                 da dieser nur einen Identifier erkennt.

  \animate{1}
  {
    {\bf Lösung:} Konkrete Syntax:
                  $\TypeRowType{\KeyAttr\ {\ExprIdentifier{a}}\colon\ {\TypeTypeVariable{\tau}}\ ;}
                   \TypeRowType{{\ExprIdentifier{m}}\colon\ {\TypeTypeVariable{\tau}}\ ;}\ \phi_1$
  }
}

%%
%% LaTeX Export
%%
\makeoverviewslide
\myslide{\LaTeX-Export}
{
  \begin{itemgroup}{\LaTeX-Export}
    \item Blablabla ...
  \end{itemgroup}
}

%%
%% Type Inference
%%
\makeoverviewslide
\myslide{Type Inference}
{
  \begin{itemgroup}{Type Inference}
    \item Blablabla ...
  \end{itemgroup}
}

%%
%% Minimal Typing
%%
\myslide{Minimal Typing}
{
  \begin{itemgroup}{Minimal Typing}
    \item Blablabla ...
  \end{itemgroup}
}

%%
%% SubTyping
%%
\myslide{Subtyping}
{

  \textbf{Mit dem Subtyping Beweiswerkzeug können Subtyprelationen überprüft werden} \\[6mm]
  \begin{itemgroup}{Beispiele:}
    \item Primitive Subtyprelation \\  $\TypeSubType {\TypeIntegerType} {\TypeIntegerType} $
    \item Rekursive Subtyprelation \\  $\TypeSubType{ \TypeRecType{\TypeTypeName{t} } {\TypeArrowType
       {\TypeIntegerType }{\TypeArrowType{\TypeTypeName{t}}{\TypeTypeName{t}}}}}
       {\TypeArrowType{\TypeIntegerType}{\TypeRecType{\TypeTypeName{t}}{\TypeArrowType
       {\TypeIntegerType}{\TypeArrowType{\TypeTypeName{t}}{\TypeTypeName{t}}}}}} $
    \item Objekt Subtyprelation \\     $\TypeSubType{\TypeObjectType{\TypeRowType{{\ExprIdentifier{a}}
       \colon\ {\TypeIntegerType}\ ;\ {\ExprIdentifier{b}}\colon\  \TypeBooleanType}\ ;}} 
       {\TypeObjectType{\TypeRowType{{\ExprIdentifier{b}}\colon\ {\TypeBooleanType}\ ;\ {\ExprIdentifier{c}}
       \colon\ {\TypeArrowType{\TypeIntegerType}{\TypeIntegerType}}\ ;\ {\ExprIdentifier{a}}\colon\ {\TypeIntegerType}
       \ ;}}}$
    \end{itemgroup}
}
\myslide{Subtyping - Implementierung}
{
  \begin{itemgroup}{Probleme und Lösungen}
    \item Eingabe der Typen
    \catchword{neue GUI für Subtyping}
    \item Es dürfen nur Typen und keine Expression eingeben werden
    \catchword{Eingabe wird direkt geparst}
    \item Speichern der Dateien
    \catchword{neue Sprachen LxSUB}
    \item Für Subtype Eingaben machen die Beweiswerkzeuge für Expressions keinen Sinn
    \catchword{Ausblenden von Beweiswerkzeugen die nicht sinnvoll sind}
    \end{itemgroup}
}


\myslide{Subtyping - GUI Sourcecode}
{
  \begin{center}
    \includegraphics[height=14cm]{images/subtype.png}
  \end{center}
}

%%
%% Fazit
%%
\makeoverviewslide
\myslide{Fazit}
{
  {\bf Fazit:} Projektgruppe TPML 2.0
  \begin{itemize}
    \item Blablabla ...
  \end{itemize}
}

\end{document}