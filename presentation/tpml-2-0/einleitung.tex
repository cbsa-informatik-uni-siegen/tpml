\myslide{Die Vorlesungen}{
  \begin{itemgroup}{\glqq\TPONE \grqq}
    \item Vermittelt Grundlagen von Programmiersprachen anhand einfacher funktionaler Programmiersprache (orientiert an O'Caml)
    \item Operationelle Semantik anhand von small und big steps
    \item Rekursion
    \item Pflichtveranstaltung für Informatik-Studenten
  \end{itemgroup}
}

\myslide{Die Vorlesungen}{
	\begin{itemgroup}{\glqq\TPTWO \grqq}
	 \item Erweiterung um zusätzliche Konzepte:
	 \item Polymorphie
	 \item Typinferenz
	 \item Objekte und Klassen
	 \item bald: Subtyping
	 \item Wahlveranstaltungen im Bereich theoretische Informatik
	\end{itemgroup}
}

\myslide{Motivation I}{
  {\bf Frage:} Warum ein Programm zur \glqq Theorie der Programmierung\grqq?

  \animate{1}{
    \begin{itemgroup}{Gründe}
      \item Erleichterter Zugang zu theoretischen Inhalten
      \item Regelwerk durch \glqq praktisches Arbeiten\grqq\ besser verständlich
      \item \glqq Gefühl\grqq\ für Interpreter und Typsysteme vermitteln
      \item Hilfestellung bei der Prüfungsvorbereitung ($\to$ nicht auf Beispiele aus Übung/Vorlesung beschränken)
      \item Verständnis für Zusammenhang von Programmtext und Ausführung auf einer Maschine
    \end{itemgroup}
  }
}

\myslide{Motivation II}{
  \begin{itemgroup}{Warum nicht O'Caml?}
  % Voller Funktionsumfang von Anfang an Sichtbar
  % Es ist vor allem eine schrittweise, interaktive Beweisführung gewünscht. Diese ist mit O'Caml nicht möglich
  % Weiterhin ist ein Commandozeilentool nicht intuitiv genug. Das Erstellen von GUIs für O'Caml allerdings noch nicht sehr komfortabel.
    \item \glqq Echte Implementierungen\grqq\ ungeeignet
    \item Abläufe nicht sichtbar für den Benutzer
    \item Typfehler von O'Caml schwer verständlich
    \item Semantik nicht zugänglich\\ ($\to$ wird in Byte/native Code kompiliert)
    \item Fehlende formale Spezifikation
  \end{itemgroup}

  \animate{1}{
    {\bf Folgerung:} Spezielles Lernprogramm besser geeignet.
  }
}

\myslide{Die Sprachen I}{
  \begin{itemgroup}{TPML modular/erweiterbar}
  % Genau wie die Vorlesung soll in TPML mit einfachen Sprachen begonnen werden,
  % die Schritt für Schritt erweitert werden.
  % Bei jedem Leistungsstand soll das Tool gleichermaßen einfach bedienbar sein.
    \item Beginn mit  einfachen Sprachen
    \item Inkrementelle Erweiterung
    \item Bis zu imperativen und objekt-orientierten Konzepten
    \item Lernprogramm soll diese Lernkurve wiederspiegeln
	\item Soll mit jedem Wissensstand benutzbar sein
  \end{itemgroup}
}

\myslide{Die Sprachen II}{
  \begin{itemgroup}{Die Sprachen im Überblick}
% Liste der verfügbaren Sprachklassen
    \item ${\cal L}_0$ - ungetypter $\lambda$-Kalkül
    \item ${\cal L}_1$ - einfach getypter $\lambda$-Kalkül
    \item ${\cal L}_2$ - ${\cal L}_1$ mit Rekursion
    \item ${\cal L}_3$ - polymorph getypte Sprache mit Listen und Tupeln
    \item ${\cal L}_4$ - imperative Konzepte (Speicher, Schleifen)
  \end{itemgroup}
}

\myslide{Die Sprachen III}
{
% Wo sinnvoll existiert für jede Sprache eine "Untersprache" für diese Erweiterungen
	\begin{itemgroup}{Spracherweiterungen}
	\item{Call-by-Name Semantik}
	\item{Objekte und Klassen}
	\item{Subtyping: Nachweis von Subtyprelationen}
	\end{itemgroup}
}

\myslide{Die Sprachen IV}{
  \begin{itemgroup}{Realisierung}
  % Durch Auswahl der Sprache legt der Benutzer die Komplexität des Tools implizit fest.
  % So ist gewährleistet,dass die Komplexität des Tools mit der Kompetenz des Users wächst. 
    \item Benutzer wählt beim Erstellen neuer Datei eine Sprache
    \item Sprache bestimmt dann Programmverhalten (Typregeln,\\ Semantikregeln, Syntaxhighlighting, \ldots)
	\item Ausgewählte Sprache beeinflusst verfügbare Beweiswerkzeuge
  \end{itemgroup}

}


\myslide{Projektgruppe TPML 2.0.0}
{
  % Bei TPML 1.0 nur die Sprachklasse aswählbar.
  % nicht kompletter Vorlesungsstoff abgedeckt
  % persistenz fehlt
	\begin{itemgroup}{Ausgangspunkt TPML 1.0.0}
	\item Unterstützt ${\cal L}_0$ - ${\cal L}_4$
	\item keine Unterstützung der Spracherweiterungen
	\item große Ausdrücke teilweise unübersichtlich
	\item kein Export / Drucken der geführten Beweise
	\end{itemgroup}
}

%ZIELE

\myslide{Projektgruppe TPML 2.0.0}
{
% Ziel: Tool insbesondere dann sinnvoll wenn der komplette Stoff unterstützt wird
% sonst doch eher frustrierend
  \begin{itemgroup}{Ziel: Abdeckung des Vorlesungsstoffes}
    \item Vollständige Unterstützung von \glqq\TPONE \grqq
	\catchword{Call-by-Name}
	\catchword{Typinferenz}
	\catchword{Objekte}
    \item Erweiterung um die Inhalte von \glqq\TPTWO \grqq
	\catchword{Klassen}
	\catchword{Subtyping}
  \end{itemgroup}
}

\myslide{Projektgruppe TPML 2.0.0}
{
  \begin{itemgroup}{Ziel: Übersichtlichkeit}
	\item Darstellung von Variablenbindungen
	%Visualisierung bei Mouse Over
	%Eine der ersten Sachen, die in der Vorlesung besprochen werden.
	\item intelligenteres Umbrechen von langen Ausdrücken
	%bisher maximal ein Umbruch
	%Variablenbindungen sonst nicht sichbar
	\item Darstellung des Syntaxbaumes
	%Strukturiertere Darstellung von großen Ausdrücken
	%Einfacheres Erkennen der nun zu verwendenden Regel
	%Erkennen wie ein konkreter Ausdruck aus Items der Abstrakten Syntax aufgebaut ist.
  \end{itemgroup}
}

\myslide{Projektgruppe TPML 2.0.0}
{
  \begin{itemgroup}{Ziel: Komfortfunktionen}
	\item Persistenz der geführten Beweise
	\catchword{PDF Export}
	%target: dau, der drucken will was er sieht
	\catchword{\LaTeX\ Export}
	%target: übungsgruppenleiter, klausurerstellung, da ausgabe bearbeitet werden kann
	\item Autovervollständigung von Ausdrücken
	%Erleichtert schreiben von Ausdrücken
	\item Verbesserung der Errormessages
  \end{itemgroup}
}

\myslide{Projektgruppe TPML 2.0.0}{
	\begin {itemgroup}{Aufgabenverteilung}
    \item Weiterentwicklung der Kernkomponenten - Christian Fehler, Benjamin Mies
    \item Visuelle Darstellung der Beweiswerkzeuge - Michael Oeste
    \item Benutzerinterface und PDF Export - Christoph Fehling
  \end{itemgroup}
}

