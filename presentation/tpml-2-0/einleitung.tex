\myslide{Die Vorlesungen}{
  \begin{itemgroup}{\glqq\TPONE \grqq}
    \item Vermittelt Grundlagen von Programmiersprachen anhand einfacher funktionaler Programmiersprache (orientiert an O'Caml)
    \item Operationelle Semantik anhand von small und big steps
    \item Rekursion
    \item Pflichtveranstaltung für Informatik-Studenten
  \end{itemgroup}
}

\myslide{Die Vorlesungen}{
	\begin{itemgroup}{\glqq\TPTWO \grqq}
	 \item Erweiterung um weitere Konzepte
	 \item Polymorphie
	 \item Typinferenz
	 \item Objekte und Klassen
	 \item Wahlveranstaltungen im Bereich theoretische Informatik
	\end{itemgroup}
}

\myslide{Die Sprachen I}{
  \begin{itemgroup}{TPML modular/erweiterbar}
    \item Beginn mit  einfachem Sprachen
    \item Inkrementelle Erweiterung
    \item Bis zu imperativen und objekt-orientierten Konzepten
    \item Lernprogramm soll diese Lernkurve wiederspiegeln
	\item und bei jedem Wissensstand benutzbar sein
  \end{itemgroup}
}

\myslide{Die Sprachen II}{
  \begin{itemgroup}{Die Sprachen im Überblick}
    \item ${\cal L}_0$ - ungetypter $\lambda$-Kalkül.
    \item ${\cal L}_1$ - einfach getypter $\lambda$-Kalkül.
    \item ${\cal L}_2$ - ${\cal L}_1$ mit Rekursion.
    \item ${\cal L}_3$ - polymorph getypte Sprache mit Listen und Tupeln.
    \item ${\cal L}_4$ - imperative Konzepte (Speicher, Schleifen).
  \end{itemgroup}
}

\myslide{Die Sprachen III}
{
	\begin{itemgroup}{Spracherweiterungen}
	\item{Call-by-Name Semantik}
	\item{Objekte und Klassen}
	\item{Subtyping: Nachweis von Subtyprelationen}
	\end{itemgroup}
}

\myslide{Die Sprachen IV}{
  \begin{itemgroup}{Realisierung}
    \item Benutzer wählt beim Erstellen neuer Datei eine Sprache.
    \item Sprache bestimmt dann Programmverhalten (Typregeln,\\ Semantikregeln, Syntaxhighlighting, \ldots).
	\item Ausgewählte Sprache beeinflusst verfügbare Beweiswerkzeuge.
  \end{itemgroup}

}

\myslide{Motivation I}{
  {\bf Frage:} Warum ein Programm zur \glqq Theorie der Programmierung\grqq?

  \animate{1}{
    \begin{itemgroup}{Gründe}
      \item Erleichterter Zugang zu theoretischen Inhalten.
      \item Regelwerk durch \glqq praktisches Arbeiten\grqq\ besser verständlich.
      \item \glqq Gefühl\grqq\ für Interpreter und Typsysteme vermitteln.
      \item Hilfestellung bei der Prüfungsvorbereitung ($\to$ nicht auf Beispiele aus Übung/Vorlesung beschränken)
      \item Verständnis für Zusammenhang von Programmtext und Ausführung auf einer Maschine.
    \end{itemgroup}
  }
}

\myslide{Motivation II}{
  \begin{itemgroup}{Warum nicht O'Caml?}
    \item \glqq Echte Implementierungen\grqq\ ungeeignet.
    \item Abläufe nicht sichtbar für den Benutzer
    \item Typfehler von O'Caml schwer verständlich
    \item Semantik nicht zugänglich\\ ($\to$ wird in Byte/native Code kompiliert)
    \item Fehlende formale Spezifikation!
  \end{itemgroup}

  \animate{1}{
    {\bf Folgerung:} Spezielles Lernprogramm besser geeignet.
  }
}

\myslide{Projektgruppe TPML 2.0.0}
{
	\begin{itemgroup}{TPML 1.0.0}
	\item Unterstützt ${\cal L}_0$ - ${\cal L}_4$
	\item keine Unterstützung der Spracherweiterungen
	\item große Ausdrücke teilweise unübersichtlich
	\item kein Export / Drucken der geführten Beweise
	\end{itemgroup}
}

\myslide{Projektgruppe TPML 2.0.0}
{
  \begin{itemgroup}{Ziele der Entwicklung}
    \item Erweiterung der Funktionalität anhand des noch verbliebenen
        Stoffs der Vorlesung \glqq\TPONE \grqq
    \item Erweiterung der Funktionalität anhand von Stoff der 
        Vorlesungen \glqq\TPTWO \grqq
    \item Studierenden ein komplettes Werkzeug zum besseren Verständnis und zur 
        Prüfungsvorbereitung zur Verfügung stellen.
    \item Unterstützung des Dozenten und der Übungsleiter bei Vorbereitung 
        der Vorlesungen und Übungen.
	\item Export zu PDF und Latex
  \end{itemgroup}
}

\myslide{Projektgruppe TPML 2.0.0}{
	\begin {itemgroup}{Aufgabenverteilung}
    \item Weiterentwicklung der Kernkomponenten - Christian Fehler, Benjamin Mies
    \item Visuelle Darstellung der Beweiswerkzeuge - Michael Oeste
    \item Benutzerinterface und PDF Export - Christoph Fehling
  \end{itemgroup}
}

\myslide{Projektgruppe TPML 2.0.0}{
  \textbf{Praktische Vorführung}\\[10cm]
}

