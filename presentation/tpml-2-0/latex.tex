\myslide{\LaTeX-Export}
{
  \begin{itemgroup}{\LaTeX-Export}
    \item Alle Komponenten implementieren eine toLatexString()-Methode
    \item Jede Komponente liefert die benötigten Commands und Imports
    \catchword{Im Kopf der exportierten \LaTeX-Datei}
    \catchword{Als seperate \LaTeX-Datei}
    \item \LaTeX-Export für Sourcecode ist nur möglich, wenn es sich bei
      der Eingabe um einen gültigen Ausdruck handelt.

  \end{itemgroup}
}

\myslide{\LaTeX-Export}
{
  \begin{itemgroup}{Wo findet der \LaTeX-Export Verwendung}
    \item Erstellen von Übungszetteln
    \item Erstellen von Musterlösungen
    \item Zum Einbau von Ausdrücken in z.B. wissenschaftlichen Arbeiten
    \item Innerhalb der Projektgruppe zum Schreiben des Handbuchs und der Präsentation
    \item Konvertierung in pdf-Format
    \catchword{Anpassung des Dokuments nach eigenen Vorstellungen bezüglich der Farben und
    Formatierungen}
  \end{itemgroup}
}

\myslide{\LaTeX-Export}
{
  \begin{itemgroup}{Einstellungen}
    \item Allgemeine Einstellungen
    \catchword{Name der zu erstellenden \LaTeX-Datei}
    \catchword{Commands und Imports in seperate Datei}
    \catchword{Auswahl des Zielordners}
    \item Extra Einstellungen für Models der Baumstruktur
    \catchword{Überlappung der einzelnen Seiten \\ (Wert zwischen $0mm$ und $50mm$)}
    \catchword{Anzahl der Seiten \\ (Maximale Anzahl 13 Seiten)}

  \end{itemgroup}
}