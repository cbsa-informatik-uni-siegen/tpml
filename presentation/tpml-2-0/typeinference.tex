\myslide{Type Inference - Implementierung}
{ 
    \begin{itemgroup}{Der Typinferenzalgorithmus arbeitet auf einer Liste von TypeFormulas. \\[5mm] 
	TypeFormula:}
    \item \textbf{Typeequation} 	$\to$	$\SeenTypes{}\ \vdash\ \TypeEquationTypeInference
	{\TypeTypeVariable{\alpha3}} {\TypeIntegerType} $ 
    \item \textbf{Typejudgement} $\to$  	 $\TypeJudgement {\TypeEnvironment{}} {\ExprConstant{1}}
    	{\TypeTypeVariable{\alpha5}} $ 
    \item \textbf{Subtype}	$\to$	$\TypeSubType {\TypeIntegerType} {\TypeBooleanType} $  \\[-3mm]
    \item \textbf{Typeequation} 
	\catchword{zur Auswertung von Typgleichungen}
	\catchword{ensteht durch Anwendung der Regeln auf eine Expression}
	\catchword{eigenes Regelwerk des Typinferenzalgorithmus}
    \end{itemgroup}
}


\myslide{Type Inference - Implementierung}
{ 
	\begin{itemgroup}{}
	\item \textbf{Subtype}	
	\catchword{zur Auswertung von Subtyprelationen}
	\catchword{ensteht durch Auswertung einer Expression vom Typ Coercion}
	\catchword{Regelwerk entspricht Sub Typing}
    	\item \textbf{Typejudgement}
	\catchword{zur Auswertung von Expressions}
	\catchword{vom Benutzer vorgeben oder aus einer anderen Expression entstanden}
	\catchword{Es wird intern eine Type Checker Knoten angelegt, die entsprechende Regel angewendet,
	und das die enstehenden Knoten der Liste von TypeFormulas hinzugefügt}
	\catchword{Regelwerk entspricht Type Checker}
     \end{itemgroup}
}