\myslide{Type Inference - Implementierung}
{ 
    \begin{itemgroup}{Der Typinferenzalgorithmus arbeitet auf einer Liste von TypeFormulas. \\[8mm] 
	TypeFormula:\\[-6mm]}
    
    \item \textbf{Typeequation} 	$\to$	$\SeenTypes{}\ \vdash\ \TypeEquationTypeInference
	{\TypeTypeVariable{\alpha3}} {\TypeIntegerType} $ 
    \item \textbf{Typejudgement} $\to$  	 $\TypeJudgement {\TypeEnvironment{}} {\ExprConstant{1}}
    	{\TypeTypeVariable{\alpha5}} $ 
    \item \textbf{Subtype}	$\to$	$\TypeSubType {\TypeIntegerType} {\TypeBooleanType} $  \\[-3mm]

    \end{itemgroup}

  \begin{itemgroup}{Jeder Knoten enthält eine Menge mit den bereits gesammelten Type Substitutions\\[-6mm]}
    \item \textbf{Type Substitution}	$\to$ $\TypeSubstitution{\TypeIntegerType}{\TypeTypeVariable{\alpha5}}$
  \end{itemgroup}
}

\myslide{Type Inference - Implementierung}
{ 
    \begin{itemgroup}{Typeequation}
     
	\item zur Auswertung von Typgleichungen
	\item entsteht durch Anwendung der Regeln auf eine Expression 
	\item eigenes Regelwerk des Typinferenzalgorithmus  \\
    \end{itemgroup}
    \begin{itemgroup}{Subtype}	
	\item zur Auswertung von Subtyprelationen
	\item entsteht durch Auswertung einer Expression vom Typ Coercion
	\item Regelwerk entspricht Sub Typing
     \end{itemgroup}
}

\myslide{Type Inference - Implementierung}
{ 
	\begin{itemgroup}{Typejudgement}
	\item zur Auswertung von Expressions 
	\item vom Benutzer vorgeben oder aus einer anderen Expression entstanden
	\item Es wird intern ein Type Checker Knoten angelegt, die entsprechende Regel angewendet,
	   und die enstehenden Knoten der Liste von TypeFormulas hinzugefügt
	\item Regelwerk entspricht Type Checker
     \end{itemgroup}
}