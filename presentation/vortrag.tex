\documentclass[fleqn,landscape,titlepage,german]{myslides}
%
% Praesentation der Projektgruppe TPML, 2006
%
% 2006 Benedikt Meurer <benny@xfce.org>
%
 
\usepackage{amssymb}
\usepackage[latin1]{inputenc}
\usepackage{german}
\usepackage{ngerman}

% Einkommentieren, um Handouts zu generieren
%\createhandoutstrue

\begin{document}
  
\presentation[TPML 1.0]{Benedikt Meurer}{TPML 1.0 - Ein "Uberblick}
\maketitle

\addoverviewitem{Einleitung}
\addoverviewitem{Die Kernkomponenten}
\addoverviewitem{Die Sprachen}
\addoverviewitem{Die Renderer}
\addoverviewitem{Die Benutzerschnittstelle}
\addoverviewitem{TPML 2.0}
\addoverviewitem{Fazit}

\CenteredGraphics



%%
%% Einleitung
%%

\makeoverviewslide

\myslide{Ziele}{
  \begin{itemgroup}{Ziele der Entwicklung}
    \item Unterst"utzung der autodidaktischen Erarbeitung von Inhalten der ,,Theorie der Programmierung I''
    \item Studierenden ein Gef"uhl f"ur Interpreter und Typsystem vermitteln
    \item Praktisches Arbeiten mit dem Regelwerk hilft beim Verst"andnis
    \item Besseres Verst"andnis der Zusammenh"ange von Programmtext und Ausf"uhrung auf einer Maschine
    \item Vorbereitung und Motivation f"ur weitergehende Veranstaltungen\\(,,Theorie der Programmierung II'' und ,,Konzepte h"oherer Programmiersprachen'')
  \end{itemgroup}
}

\myslide{Realisierung - ,,Divide and Conquer''}{
  {\bf Leitsatz f"ur Ingenieure:} Hast du ein Problem, mach viele draus!

  \animate{1}{
    \begin{itemgroup}{Zerlegung in drei Teilprojekte}
      \item {\tt de.unisiegen.tpml.core}      - Die Kernkomponenten Typechecker, Interpretern, Parsern, Lexern, Pretty Printern, \ldots
      \item {\tt de.unisiegen.tpml.graphics}  - Untere Ebene der Benutzeroberfl"ache mit Texteditor, big und small step und Typechecker Oberfl"ache
      \item {\tt de.unisiegen.tpml.ui}        - Obere Ebene der Benutzeroberfl"ache die alles zusammenf"ugt
    \end{itemgroup}
    \graphicybox{5cm}{images/realisierung.pdf}
  }
}

\myslide{Realisierung - Arbeitsaufteilung}{
  \begin{itemgroup}{Vorgehen}
    \item Teilprojekte \cemph{unabh"angig} voneinander entwickelt
    \item Kommunikation "uber \cemph{wohldefinierte} Schnittstellen
  \end{itemgroup}

  \begin{itemgroup}{Entwickler (Ansprechpartner f"ur nachfolgende Gruppen)}
    \item {\tt de.unisiegen.tpml.core}      - Benedikt Meurer
    \item {\tt de.unisiegen.tpml.graphics}  - Marcell Fischbach
    \item {\tt de.unisiegen.tpml.ui}        - Christoph Fehling
  \end{itemgroup}
}

\myslide{Realisierung - Entwicklung}{
  \begin{itemgroup}{Werkzeuge}
    \item Plattform: JavaSE 5.0
    \item Entwicklungsumgebung: Eclipse 3.1.2
    \item GUI Builder: NetBeans 5.0
    \item Lexer Generator: JFlex 1.4.1
    \item Parser Generator: JavaCUP 11a
    \item Protokollierung: log4j 1.2.14
  \end{itemgroup}

  \begin{itemgroup}{Grundlage}
    \item Vorlesungsinhalt ,,Theorie der Programmierung I'' (2005/2006)
    \item OCaml 3.09.3
  \end{itemgroup}
}



%%
%% Die Kernkomponenten
%%

\makeoverviewslide

\myslide{"Uberblick I}{
  \begin{itemgroup}{Komponenten}
    \item Abstract Syntax Tree
    \item Small step Interpreter
    \item Big step Interpreter
    \item Type Checker
    \item Parser/Lexer
    \item Pretty Printer
  \end{itemgroup}

  \animate{1}{
    \begin{itemgroup}{Ziele}
      \item \cemph{Generische} Komponenten (Big/Small Stepper, Type Checker)
      \item Sprachabh"angige Teile \cemph{isolieren}
      \item Einziger Abstract Syntax Tree
    \end{itemgroup}
  }
}

\myslide{"Uberblick II}{
  {\bf "Uberblick:} Datenflu"s zwischen den Kernkomponenten

  \graphicxbox{24cm}{images/overview.pdf}

  \begin{itemize}
    \item grobe Strukturierung
    \item einige Komponenten mehrfach vorhanden (z.B. Parser/Lexer)
  \end{itemize}
}

\myslide{"Uberblick III}{
  \begin{itemgroup}{Voraussetzung:}
    \item Unterschiedliche Sprachen (nur teilweise aufeinander aufbauend)
    \item Parser, Lexer, abstrakte Syntax (und damit Pretty Printer), Big/Small Stepper, Type Checker augenscheinlich sprachabh"angig
  \end{itemgroup}

  {\bf Problem:} Jede Komponente pro Sprache realisieren w"are zu aufwendig/nicht wartbar

  {\bf L"osung:} Aufteilung in sprachabh"angige/-unabh"angige Teile
}

\myslide{"Uberblick IV}{
  {\bf Aufteilung:} Sprachabh"angige/-unabh"angige Teile

  \graphicxbox{18cm}{images/separierung.pdf}

  \begin{itemize}
    \item AST, PrettyPrinter, Parser/Lexer klar
    \item Interpreter, Type Checker aufspalten
  \end{itemize}
}

\myslide{Die abstrakte Syntax I}{
  \begin{itemgroup}{Abstrakte Syntax}
    \item Darstellung m"oglichst nahe an der Vorlesung
    \item Realisierung durch \cemph{Vererbung} in Java (einfache Hierarchie)
  \end{itemgroup}

  \graphicybox{10cm}{images/ast.pdf}
}

\myslide{Die abstrakte Syntax II}{
  {\bf Problem I:} Zugeh"origkeit zur Menge $Val$ (operationelle Semantik)?

  \begin{itemize}
    \item Klar bei {\tt Value} und abgeleiteten Klassen
    \item Aber {\tt Tuple} {\tt extends} {\tt Value} oder {\tt List} {\tt extends} {\tt Value}?
  \end{itemize}

  \animate{1}{
    {\bf L"osung:} Werte nicht ,,durch Vererbung auszumachen''

    Statt
    \[
      e \in Val \Leftrightarrow e \texttt{ instanceof } \texttt{Value}
    \]
    eine Methode {\tt isValue()} in {\tt Expression}, statisch f"ur {\tt Value}, dynamisch f"ur {\tt List}/{\tt Tuple}
  }
}

\myslide{Die abstrakte Syntax III}{
  {\bf Problem II:} Unterschiedliche Substitution f"ur {\tt Let} und {\tt LetRec}

  Substitution und Prettyprinting unterschiedlich f"ur
  \[
    \textbf{let}\,id=e_1\,\textbf{in}\,e_2
  \]
  und
  \[
    \textbf{let rec}\,id=e_1\,\textbf{in}\,e_2
  \]
  aber {\tt LetRec} {\tt extends} {\tt Let}.

  \animate{1}{
    {\bf L"osung:} Bisher keine, {\tt substitute()} und "ahnliche Methoden m"ussen neuimplementiert werden in diesen
    F"allen ($\to$ schlechtes Codereuse, doppelte Fehler)
  }
}

\myslide{Die abstrakte Syntax IV}{
  \begin{itemgroup}{Syntaktischer Zucker}
    \item eigene Klassen f"ur syntaktischen Zucker
    \item z.B. {\tt LetRec} f"ur $\textbf{let rec}\,id=e_1\,\textbf{in}\,e_2$
    \item oder {\tt CurriedLet} f"ur $\textbf{let}\,id\ id_1\,\ldots\,id_n = e_1\,\textbf{in}\,e_2$
  \end{itemgroup}

  \animate{1}{
    "Ubersetzung in Kernsyntax durch sprachabh"angigen {\tt LanguageTranslator}
    \begin{itemize}
      \item erm"oglicht Konstrukte sp"ater in Kernsyntax zu "ubernehmen
      \item z.B. {\bf let} syntaktischer Zucker f"ur $\mathcal{L}_1$ aber Kernsyntax f"ur $\mathcal{L}_1^{typed}$
    \end{itemize}

    Zwei Modi f"ur {\tt LanguageTranslator}: {\em outermost only} und {\em recursive}
  }
}

\myslide{Die konkrete Syntax}{
  \begin{itemgroup}{Konkrete Syntax:}
    \item Definitiv \cemph{sprachabh"angig}
    \item Eigene Lexer/Parser pro Sprache
    \item Lexer f"ur \cemph{Syntaxhighlighting} im Editor benutzen
  \end{itemgroup}

  \animate{1}{
    \begin{itemgroup}{Dazu:}
      \item Gemeinsame Schnittstellen {\tt LanguageScanner} und {\tt LanguageParser} f"ur alle Lexer/Parser.
      \item Parser durch {\tt Java CUP}, Lexer durch {\tt JFlex} generieren.
    \end{itemgroup}

    {\bf Beachte:} Konkrete Syntax nicht 100\%ig gleich abstrakter Syntax, z.B. {\tt lambda id:'a\,->\,'b.e} statt $\lambda id:\alpha\to\beta.e$, etc.
  }
}

\myslide{Der Pretty Printer}{
  \begin{itemgroup}{Pretty Printing}
    \item \cemph{Sprachunabh"angig}, basierend auf abstrakter Syntax
    \item ,,Umkehrung'' des Parsings, Prinzip der \cemph{,,Priority Grammars''} zur Klammerung
    \item {\tt PrettyString} als Repr"asentation mit zus"atzlichen Informationen
    \item {\tt PrettyPrintable} implementiert in {\tt Expression} und {\tt Type}
    \item {\tt PrettyStringBuilder} zum Konstruieren von {\tt PrettyString}s
  \end{itemgroup}
}

\myslide{ProofModels}{
  {\bf Idee:} Big/Small Stepper und Type Checker aufteilen in sprachabh"angige/unabh"angige Teile

  \animate{1}{
    \begin{itemgroup}{Sprachunabh"angiger Teil}
      \item Bietet zwei Schnittstellen:
      \item zum sprachabh"angigen Teil
      \item zum User Interface
    \end{itemgroup}

    {\bf Schnittstelle zur Sprache:} Regeln ({\tt ProofRuleSet}s) in das {\tt ProofModel} laden

    {\bf Schnittstelle zum UI:} Baumdarstellung basierend auf {\tt TreeModel} (Swing)
  }
}

\myslide{ProofRuleSets}{
  \begin{itemgroup}{Schnittstelle zur Sprache}
    \item Sprachen bieten Regeln, z.B. {\tt BigStepProofRule}s
    \item Regeln angeordnet in {\tt ProofRuleSet}s
    \item Regeln interagieren mit spezifischem {\tt ProofModel} "uber {\tt ProofContext}s
  \end{itemgroup}

  \animate{1}{
    \begin{itemgroup}{Big step und Type Checker Regeln:}
      \item {\tt apply()} zum Anwenden auf Knoten
      \item {\tt update()} f"ur z.B. {\bf (LET)}
    \end{itemgroup}

    Small stepper hingegen ,,wei"s'' Regeln exakt und vergleicht lediglich Benutzerauswahl ($\to$ Eindeutigkeit des small steps)
  }
}

\myslide{ProofContexts}{
  {\bf Frage:} Regeln interagieren "uber {\tt ProofContext}s statt direkt mit {\tt ProofModel}?

  \animate{1}{
    \begin{itemize}
      \item {\tt ProofContext} als \cemph{Protokollierungsschicht} zwischen Regeln und Model
      \item kann Schritte f"ur \cemph{Undo}/\cemph{Redo} bestimmen
      \item Undo/Redo damit transparent f"ur Model und Regeln, keine explizite Programmierung erforderlich
    \end{itemize}
  }
}

\myslide{TreeModel}{
  \begin{itemgroup}{Schnittstelle zum UI}
    \item grunds"atzlich alle drei \cemph{baumartige} Strukturen
    \item basierend auf {\tt javax.swing.tree.TreeModel}
    \item da existierendes Framework ($\to$ well-tested code)
    \item leichter zu debuggen, z.B. mittels {\tt JTree}
  \end{itemgroup}

  {\bf Small stepper:} Darstellung als degenerierter Baum
}

\myslide{ProofModels revisited}{
  {\bf Zusammenfassung:} Big/Small step und Type Checker {\tt ProofModel}s

  \begin{itemize}
    \item Sprachunabh"angiger, generischer Kern
    \item Schnittstelle zum Regelwerk (sprachabh"angig)
    \item Schnittstelle zum UI (sprachneutral)
    \item Beispielhaft: das {\tt BigStepProofModel}
  \end{itemize}

  \graphicybox{4cm}{images/proofmodels.pdf}
}



%%
%% Die Sprachen
%%

\makeoverviewslide

\myslide{Die Sprachen I}{
  \begin{itemgroup}{Zielsetzung:}
    \item Unterst"utzung f"ur unterschiedliche Sprachen
    \item dynamisch erweiterbar
  \end{itemgroup}

  {\bf Grundlage:} zuvor geschilderte Aufteilung der Kernkomponenten

  {\bf dynamisch erweiterbar:} ja, aber bisher nicht von ,,extern'' (z.B. JAR-File/Plugin Mechanismus)
}

\myslide{Die Sprachen II}{
  \begin{itemgroup}{Realisierte Sprachen (TPML 1.0)}
    \item $\mathcal{L}_0$ - reiner ungetypter $\lambda$-Kalk"ul
    \item $\mathcal{L}_1$ - angewandter $\lambda$-Kalk"ul mit einfachem Typsystem
    \item $\mathcal{L}_2$ - rekursive Ausdr"ucke
    \item $\mathcal{L}_3$ - polymorphes Typsystem, Listen, Tupel
    \item $\mathcal{L}_4$ - imperative Konzepte
  \end{itemgroup}

  \begin{itemgroup}{Nicht realisiert}
    \item Records mit Subtyping (wird in TP II ersetzt durch Objekte)
  \end{itemgroup}
}

\myslide{Die Sprachen III}{
  \begin{itemgroup}{Ein Blick in die Zukunft}
    \item Objekte (Reihen und Reihentypen)
    \item Klassen
    \item Subtyping
    \item rekursive Typen
    \item Pattern Matching 
    \item Aufz"ahlungstypen
  \end{itemgroup}

  Teilweise "Anderungen an sprachunabh"angigem Teil (z.B. \cemph{Sorted Unification})
}



%%
%% Die Renderer
%%

\makeoverviewslide

\myslide{Die Renderer I}{
  \begin{itemgroup}{Das {\tt de.unisiegen.tpml.graphics} Projekt}
    \item oberhalb der Kernkomponenten
    \item realisiert den \cemph{Editor} (mit Syntaxhighlighting) und die \cemph{Renderer}
    \item ein Renderer pro Beweiswerkzeug (Small/Big step Interpreter, Type Checker)
    \item Renderer sind Schnittstelle zwischen Benutzer und den Kernkomponenten
    \item Renderer absolut \cemph{sprachneutral}
  \end{itemgroup}
}

\myslide{Die Renderer II}{
  \begin{itemgroup}{Funktionsweise}
    \item Renderer arbeiten grunds"atzlich auf {\tt ProofModel}s, mit jeweiligen Eigenheiten
    \item grob aufgeteilt in drei Klassen {\tt NodeComponent}, {\tt Component} und {\tt View}
    \item {\tt NodeComponent} als Widget per Knoten im Beweisbaum
    \item {\tt Component} sorgt f"ur Layouting der {\tt NodeComponent}s
    \item {\tt View} enth"alt ein {\tt Component}, dient als Schnittstelle zum dar"uberliegenden UI
  \end{itemgroup}
}

\myslide{Die Renderer III}{
  \begin{itemgroup}{Zuk"unftige Entwicklungen}
    \item Visualisierung von Bindungen (z.B. bei $\textbf{let}$, $\lambda$, etc)
    \item Darstellung des abstrakten Syntaxbaums eines Ausdrucks
    \item Schrittweise Durchf"uhrung einer Substitution
    \item Schrittweise Durchf"uhrung des Unifikationsalgorithmus
    \item Schrittweise Durchf"uhrung des Subtypingalgorithmus\\($\to$ Objekte)
    \item Druckfunktion in Kombination mit dem UI
  \end{itemgroup}
}



%%
%% Die Benutzerschnittstelle
%%

\makeoverviewslide

\myslide{Die Benutzerschnittstelle I}{
  \begin{itemgroup}{Die Benutzerschnittstelle}
    \item Bindeglied zwischen den Teilprojekten
    \item Basierend auf AWT/Swing f"ur m"oglichst einfache Installation
    \item Orientiert an bekannten IDEs (Eclipse, NetBeans)
    \item Ein-Fenster-Anwendung mit einem Tab pro Datei
    \item Pro Tab {\tt Source}, {\tt Big Step}, {\tt Small Step} und {\tt Type Checker}
    \item Verbindung Datei $\leftrightarrow$ Sprache
  \end{itemgroup}
}

\myslide{Die Benutzerschnittstelle II}{
  \begin{itemgroup}{Details}
    \item GUI Entwurf mit NetBeans
    \item Verschiedene \cemph{Schichten}: {\tt MainWindow} $\to$ {\tt EditorPanel} $\to$ {\tt EditorComponent}
    \item Kommunikation zu h"oheren Schichten "uber \cemph{Bean Properties} (z.B. Zustand der Undo/Redo Button)
    \item Session Management (Fensterposition/gr"o"se und Dateien merken)
  \end{itemgroup}

  \begin{itemgroup}{Zusatzfeatures}
    \item Starter f"ur Unix/Linux und Windows, mit Pr"ufung auf JavaSE 5.0
    \item Integration in das Betriebssystem "uber MIME und Application Database (bisher nur unter Unix/Linux)
  \end{itemgroup}
}

\myslide{Die Benutzerschnittstelle III}{
  \begin{itemgroup}{Neulich in der Glaskugel$\ldots$}
    \item Druckfunktion und/oder PDF/PS Export
    \item Weitere Strukturierung, Trennung von NetBeans GUI und Funktionalit"at
    \item OS-Integration f"ur Windows/OSX
    \item Installer, optional als Bundle mit JRE 5.0 (Fehlerursache \#1)
    \item {\tt log4j}-Integration f"ur Fehlerberichte
    \item \emph{Outline} (abstrakte Syntax) f"ur Editor, ala Eclipse
  \end{itemgroup}
}



%%
%% TPML 2.0
%%

\makeoverviewslide

\myslide{TPML 2.0}{
  {\bf Die Zukunft:} TPML 2.0/3.0

  \begin{itemize}
    \item einige Ziele f"ur Teilprojekte schon abgesteckt
    \item bereits Stoff f"ur (fast) zwei weitere PGs
    \item Vorlesung ,,Theorie der Programmierung'' sehr variabel
    \item Userfeedback wird weitere Baustellen aufzeigen
  \end{itemize}

  \animate{1}{
    {\bf Ansprechpartner:} Christoph Fehling und Benedikt Meurer

    {\bf Erfahrungsgem"a"s:} Ziele/Anspr"uche nicht zu hoch stecken
  }
}



%%
%% Fazit
%%

\makeoverviewslide

\myslide{Fazit}{
  {\bf Fazit:} Stichpunkte im R"uckblick auf ein Jahr TPML 1.0

  \begin{itemize}
    \item Kleine Projektgruppe (2-3 Teilnehmer) \quad\quad \includegraphics[width=1cm]{images/wmdz.png}
    \item Gute Arbeitsatmosph"are
    \item Flexible Einteilung, (fast) keine fixen Termine
    \item ,,Extreme Prototyping''
    \item Flexible Zielsetzung
    \item Formale Grundlage (in der Vorlesung fehlende Regeln vorher festklopfen)
  \end{itemize}
}



\end{document}

